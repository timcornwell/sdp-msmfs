\message{ !name(msmfs_aa.tex)}%                                                                 aa.dem
% AA vers. 6.1, LaTeX class for Astronomy & Astrophysics
% demonstration file
%                                                 (c) Springer-Verlag HD
%                                                revised by EDP Sciences
%-----------------------------------------------------------------------
%
%\documentclass[referee]{aa} % for a referee version
%\documentclass[onecolumn]{aa} % for a paper on 1 column  
%\documentclass[longauth]{aa} % for the long lists of affiliations 
%\documentclass[rnote]{aa} % for the research notes
%\documentclass[letter]{aa} % for the letters 
%
\documentclass[structabstract]{aa}  
%\documentclass[traditabstract]{aa} % for the abstract without structuration 
                                   % (traditional abstract) 
%
\usepackage{graphicx}
%%%%%%%%%%%%%%%%%%%%%%%%%%%%%%%%%%%%%%%%
\usepackage{txfonts}
%%%%%%%%%%%%%%%%%%%%%%%%%%%%%%%%%%%%%%%%
\usepackage{epsfig}
\newcommand{\nuno}{{\left(\frac{\nu}{\nu_0}\right)}}
%%\newcommand{\nuno}{{({\nu}/{\nu_0})}}
\newcommand{\dnuno}{{\left(\frac{\nu-\nu_0}{\nu_0}\right)}}
%\newcommand{\dnuno}{{({\nu}/{\nu_0} - 1)}}
\newcommand{\F}{{F}}
\newcommand{\Fd}{{F^\dag}}
\newcommand{\E}{{E}}
\newcommand{\Ed}{{E^\dag}}
\newcommand{\G}{{G}}
\newcommand{\Gnu}{{G_{\nu}}}
\newcommand{\Gddnu}{{G_{\nu}^{dd}}}
\newcommand{\Gdddnu}{{{G_{\nu}^{dd}}^{\dag}}}
\newcommand{\Gd}{{G^\dag}}
\newcommand{\Gdnu}{{G_{\nu}^{\dag}}}
\newcommand{\Sa}{{S}}
\newcommand{\Sd}{{S^\dag}}
\newcommand{\Sna}{{S_{\nu}}}
\newcommand{\Snd}{{S_{\nu}^\dag}}
\newcommand{\Wim}{{W^{im}}}
\newcommand{\Wimd}{{{W^{im}}^\dag}}
\newcommand{\Wimn}{{W^{im}_{\nu}}}
\newcommand{\Wimnd}{{{W^{im}_{\nu}}^\dag}}
\newcommand{\wnt}{{w_{\nu}^t}}
\newcommand{\wnp}{{w_{\nu}^p}}
\newcommand{\wntp}{{w_{\nu}^{t+p}}}
\newcommand{\Wnt}{{W^{mfs}_t}}
\newcommand{\Wntd}{{{W^{mfs}_t}^\dag}}
\newcommand{\Wntn}{{w^{mfs}_{t,\nu}}}
\newcommand{\Wntnd}{{{w^{mfs}_{t,\nu}}^\dag}}
\newcommand{\Wnp}{{W^{mfs}_p}}
\newcommand{\Wnpd}{{{W^{mfs}_p}^\dag}}
\newcommand{\Wnpn}{{W^{mfs}_{p,\nu}}}
\newcommand{\Wnpnd}{{{W^{mfs}_{p,\nu}}^\dag}}
\newcommand{\Pnu}{{P_{\nu}}}
%\usepackage{amsmath}
%%%%%%%%%%%%%%%%%%%%%%%%%%%%%%%%%%%%%%%%
%
\begin{document}

\message{ !name(msmfs_aa.tex) !offset(-3) }

%\tableofcontents
%\newpage
%
   \title{A multi-scale multi-frequency deconvolution algorithm
for synthesis imaging in radio interferometry}

%   \subtitle{I. Overviewing the $\kappa$-mechanism}

   \author{U. Rau
          \inst{1}
          \and
          T.J. Cornwell \inst{2} %\fnmsep \thanks{Just to show the usage of the elements in the author field}
          }

   \institute{New Mexico Institute of Mining and Technology,Socorro, NM,USA\\
%              National Radio Astronomy Observatory, Socorro, NM, USA\\
%              Australia Telescope National Facility, CSIRO, Sydney, AU\\
              \email{rurvashi@aoc.nrao.edu}
         \and
             Australia Telescope National Facility, CSIRO, Sydney, AU \\
             \email{tim.cornwell@atnf.csiro.au}
%             \thanks{blah.}
             }

   \date{Received XXX XX, 2009; accepted XXX XX, 2009}
   %%\date{Received December 15, 2008; accepted XXX XX, 2009}

% \abstract{}{}{}{}{} 
% 5 {} token are mandatory
 
  \abstract
  % context heading (optional)
  % {} leave it empty if necessary  
   {The use of broad-band receivers in radio interferometery impacts standard continuum
    synthesis imaging in three ways. 
    First, correlations measured at different frequencies increase the uv-coverage of
    the synthesis array by sampling different spatial frequencies.
    Second, the sky brightness can change significantly across the large frequency 
    range that these receivers are sensitive to, violating a basic assumption of the
    Van Cittert Zernike theorem.  
    Third, the field-of-view of each array element is frequency-dependent, and this
    affects the spectral characteristics of the measured sky brightness during wide-field
    continuum imaging.
    Existing algorithms designed to account for some of the above, can be shown to be
    insufficient to achieve target image fidelity and dynamic range over the wide bands that 
    most new instruments are being built with.}
  % aims heading (mandatory)
   {This paper describes a multi-scale multi-frequency deconvolution algorithm for
    synthesis imaging that combines the information measured about the source spectrum
    with the additional uv-coverage offered by multi-frequency-synthesis to 
    reconstruct source structure and spectra at multiple spatial scales, %.
    while accounting for a frequency-dependent field-of-view.}
  % methods heading (mandatory)
   {The MS-MFS algorithm discussed in this paper extends the Sault-Wieringa MF-Clean 
    algorithm to reconstruct spectral curvature in addition to spectral index,
    and combines it with a  multi-scale deconvolution approach similar to that of MS-Clean.
    We also discuss a simpler hybrid of spectral-line and continuum imaging methods
    that may suffice in certain situations, and a more generic parametric approach.
    Frequency-dependent fields-of-view are corrected via existing wide-field imaging
    algorithms.
   }
  % results heading (mandatory)
   {We show via simulations and application to wideband (E)VLA data (1 GHz to 2 GHz),
    that it is possible to reconstruct both spatial and spectral structure of 
    compact and extended emission across wide fields of view.
    We also present an error analysis to show
    under what conditions these techniques are feasible and give accurate results. %(give numbers)
   }
  % conclusions heading (optional), leave it empty if necessary 
   {}

   \keywords{multi-frequency-synthesis --
                multi-scale-deconvolution --
                spectral index --
		spectral curvature --
               }

   \authorrunning{U.Rau and T.J.Cornwell}
   \titlerunning{Multi-Scale Multi-Frequency Synthesis Imaging in Radio Interferometry}
   \maketitle
%
\section{Introduction}

Instruments like the EVLA and e-MERLIN, are among a new 
generation of broad band radio interferometers designed to provide continuum sensitivities and
high dynamic range imaging capabilities superior to that of existing radio interferometers.
The wide bandwidths will allow detailed measurements of the spectral structure of astrophysical
sources, but to achieve the desired continuum sensitivities, imaging algorithms need to be 
sensitive to the effects of combining measurements from a large range of frequencies 
(namely varying ranges of sampled spatial scales, varying antenna fields-of-view, and
 the spectral structure of the sky brightness distribution).

%New algorithms are required to take full advantage of new broad-band 
%receivers for telescopes like the EVLA where broad-band sensitivity is the primary gain,
%as well as for arrays with fewer antennas like the VLBA, ATCA or Merlin where wide
%bandwidths are the key to measuring enough Fourier samples to reconstruct extended
%emission accurately.

Multi-Frequency-Synthesis (MFS) is the technique of combining measurements at multiple
discrete receiver frequencies during synthesis imaging.
MFS was initially done (REF) to increase the aperture-plane coverage
of sparse arrays by using narrow-band receivers and switching frequencies during the observations.
Wide bandwidth systems ( ~10\% ) presented the problem of bandwidth smearing, which was
eliminated (REF) by splitting the wide band into narrow-band channels and mapping them onto
their correct spatial frequencies during imaging. It was still assumed that
at the receiver sensitivities of the time, 
the sky brightness was constant across the observed bandwidth.
The next step was to consider a frequency-dependent sky brightness distribution.
Conway,Cornwell and Wilkenson,1990 describe a Double Deconvolution algorithm based on
spectral psfs, the instruments response to a series of spectral basis function,
in this case a Taylor series.
Sault,Wieringa,1994 describe a CLEAN-based MFS algorithm (MF-CLEAN) which models an image 
as a collection of point sources with power law spectra, and uses a linear approximation
to these spectra along with the above mentioned spectral psfs. A spectral index map
was produced as a by-product of the deconvolution.

So far, these CLEAN-based MFS deconvolution algorithms used 
point-source (zero-scale) flux components to model the sky emission,
a choice not well suited for extended emission.
We show in this paper that for MFS, deconvolution errors that occur with a point-source
model are enhanced in the spectral index image because of error propagation effects, and
that the use of a multi-scale technique can minimize this.
%In (REF1,REF1), XXX show that for high dynamic range imaging, point-source models
%are not well suited for extended emission, and describe several multi-scale
%techniques that are more capable at reconstructing complex extended emission accurately.
Multiscale deconvolution techniques model images using flux
components of varying scale size, and are more accurate at deconvolving large scale emission
features. Bhatnagar,Cornwell,2004 describe the ASP-CLEAN algorithm that explicitly fits 
for the parameters of Gaussian flux components and uses scale size to aid the separation of
signal from noise. Cornwell,Holdaway,2004 describe the MS-CLEAN algorithm which performs
matched filtering using templates constructed from the instrument response to various large 
scale flux components.

For high dynamic range imaging across wide fields-of-view,
direction-dependent instrumental effects that need to be accounted for are 
the w-term and the antenna power pattern, both of which are frequency-dependent. 
The XXX algorithm described by Bhatnagar,Cornwell,Golap 2007 describes the correction of
direction-dependent effects during imaging for the narrow-band case.

%Some other work related to this problem include a Spatio-spectral MEM (xxx).

   The MS-MFS algorithm discussed in this paper extends the Sault-Wieringa MF-Clean 
    algorithm to reconstruct spectral curvature in addition to spectral index,
    and combines it with a  multi-scale deconvolution approach similar to that of MS-Clean.
    Frequency-dependent wide-field instrumental effects are accounted for
    along with MFS as an extension of the AProjection and WProjection methods.
%    We also discuss a simpler hybrid of spectral-line and continuum imaging methods
%    that may suffice in certain situations.
   We show via simulations and application to (E)VLA observations of Cygnus-A,
    that for wide bandwidths (50\% up to xxx\%)
    it is possible to reconstruct both spatial and spectral structure from the
    combined measurements.
    We use the example of the 3C286 field to demonstrate
    frequency-dependent correction of the primary beam for wide-field imaging.
    We also present an error analysis to show
    under what conditions these techniques are feasible. %(give numbers)



%%Section \ref{PRIOR WORK} describes existing methods and their limitations.
%Section \ref{HYBRID} describes a simple MF algorithm for arrays with dense single-frequency
%uv-coverage.
%Section \ref{MS-MFS} describes the details of the multi-scale multi-frequency deconvolution algorithm 
%for arrays that use MFS to increase their uv-coverage. 
%Section \ref{SIMULATIONS} compares images of continuum flux, spectral index and curvature, obtained
%by the above two algorithms.
%Section \ref{DATA} compares results from both algorithms on real data (and compares to
%		previously existing algorithms).

%__________________________________________________________________

\section{Wide-Band Imaging}\label{IMAGING}
%__________________________________________________________________

%\subsection{Discussion}
%The primary use of broadband receivers in radio interferometry is to enhance the instruments
%continuum sensitivity and increase the range of spatial frequencies measured instantaneuosly.
%At these wide bandwidths, the frequency-dependence of the sky brightness distribution becomes
%relevant and must be taken into account during image reconstruction.  
%%Multi-frequency-synthesis (REF CCW/SW) is the technique of combining visibility 
%%measurements made at multiple receiver/sky frequencies to construct an image of the
%%sky and its frequency dependence.

We begin with a discussion of how well we can reconstruct both spatial and 
spectral information about the source
from incomplete measurements at multiple frequencies. 
The spatial frequency range per channel is given by 
$u_{\{min,max\}}(\nu) = \vec{b_{\{min,max\}}} \nu/c$, where
$\vec{b_{\{min,max\}}}$ represents the shortest and longest projected baseline vector on
the plane perpendicular to the direction of the source. 
The maximum spatial frequency measured at each frequency defines the angular resolution of
the instrument at that frequency as $\theta_{\nu} = 1/u_{max}(\nu)$.
The range of spatial frequencies between $u_{min}$ at $\nu_{max}$ and  $u_{max}$ at $\nu_{min}$
represents the region that is sampled at all frequencies in the band. In this region, 
both spatial and spectral information is measured in comparable detail. 
The spatial frequencies outside this region are sampled only by a fraction of the band, and 
the accuracy of the reconstruction depends on how well the spectral and spatial structure are
constrained by the choice of a flux model and the rest of the measurements.

%Multi-frequency-synthesis (REF CCW/SW) is the technique of combining visibility 
%measurements made at multiple receiver/sky frequencies to construct a single continuum image.
%The main advantages of MFS are the additional sensitivity and fidelity that comes with filling in 
%the spatial frequency plane and increasing the number of independant constraints
%on the spatial and spectral structure of the source, the increased angular resolution for
%imaging, and the measurement of the source spectrum across a wide bandwidth.

For a flat-spectrum source, measurements at multiple frequencies sample the same sky brightness
distribution at different ranges of spatial scales, increasing the signal-to-noise of the
measurements in regions of overlap, and providing better uv-coverage and angular resolution.
Therefore standard deconvolution algorithms applied to measurements combined via MFS will suffice
to reconstruct source structure across the full range of spatial scales measured across the band.

For sources with spectral structure, different frequencies measure different sky brightness 
distributions, and a direct combination of frequencies via MFS would generate spurious spatial
structure. The traditional method of imaging has been to make separate images at each frequency and
and derive spectral structure after tapering the measurements from higher frequencies to match
the angular resolution of the lowest frequency. 
The imaging sensitivity and fidelity is however restricted to be that of a single channel and its 
uv-coverage, and in most cases, it is possible to do better.

A compact, unresolved source with spectral structure is measured as a point
source at all frequencies, and $u_{max}$ at $\nu_{max}$ gives the maximum angular resolution at
which this source can be imaged. Since the visibility function of a point source is flat across
the entire spatial frequency plane, its spectrum is adequately sampled by the multi-frequency measurements,
and it is possible to completely reconstruct both spatial and spectral structure for such a source.

%%%???
For resolved sources with spectral structure, the accuracy of the reconstruction
across all spatial scales between $u_{min}$ at $\nu_{min}$ and  $u_{max}$ at $\nu_{max}$
depends on an appropriate the choice of flux model, and the constraints that it provides.

For example, a source with a synchrotron spectrum can be described by a brightness 
distribution at a particular frequency, and a power-law spectrum for each location. 
The angular resolution of the images are given by $u_{max}$ at $\nu_{max}$, with the 
assumption that different frequencies probe the same structure but different amplitudes. 
Measurements at the highest frequencies constrain the angular size of the flux 
components, and the combined spectra are constrained by the rest of the measurements.
This constraint is strong enough to correctly reconstruct even moderately resoloved sources
that are completely unresolved at the low end of the band, but resolved at the higher end.

On the other hand, a source whose structure itself changes across the band would break the
assumption that structure seen at the highest resolution is what is present all across the band.
One example is with multi-frequency observations of solar magnetic loops
where the different frequencies probe different layers in the xxx-sphere and can have
very different structures. In this case, a complete reconstruction would be possible only
between $u_{min}$ at $\nu_{min}$ and  $u_{max}$ at $\nu_{max}$, unless the flux model
includes constraints that bias the solution towards one appropriate for solar magnetic loops.
The imaging would still however operate at the continuum sensitivity level, and is therefore
better than single-channel imaging.

The lower end of the frequency range presents a different problem. 
The size of the central hole in the uv-coverage
increases with frequency, and source spectra are not measured adequately for
extended emission whose visibility function is non-zero only in
the range below $u_{min}$ at $\nu_{max}$. A flat-spectrum large-scale source is 
indistinguishable from a relatively smaller source with a steep spectrum, and vice versa.
For scale sizes in this range, additional constraints in the form 
of total-flux values may be required for each frequency.
% or spectrum is simple enough that half the band is enough to get it.

In addition to the sky flux being frequency dependent, the instrumental effects are also 
frequency dependent.
The HPBW of the primary beam of each array element scales as $HPBW = \frac{c}{\nu D}$. 
The reconstructed image will therefore have artificial spatial and spectral structure that can be
measured and removed to undo these effects.

\section{MS-MFS Algorithm}\label{ALGORITHMS}
The design of an image reconstruction algorithm begins with an appropriate choice of
parameterization of the sky brightness distribution and various instrumental effects.
This parameterization is then folded into the transfer function of the instrument
whose measurement process we are trying to invert. An optimization process is then
followed to obtain best-fit estimates for the parameters which can then be 
interpreted physically. This section describes each
of these stages in detail for one approach to broad-band synthesis imaging 
for radio interferometry.
The following is an extension of MF-CLEAN, combined with MS-CLEAN and AProjection.

\subsection{Parameterization of spatial structure}
For a multi-scale model, an image can be written as a linear combination of images
at different angular resolutions (Ref MS-CLEAN).
A multiscale representation of an image is given by 
\begin{equation}
I^m = \left\{ \sum_{s=0}^{N_s} \left[ I^m_{s,\delta}\star I^{blob}_s \right] \right\}
~~~ where ~~~ I^{blob}_s = XXX
\label{EQN_MS}
\end{equation}
and $N_s$ is the number of discrete spatial scales used to represent the image, and
$I^m_{s,\delta}$ represents a collection of $\delta$-functions that describe the locations
and integrated amplitudes of flux components of scale $s$ in the image.
$I^{blob}_s$ is a tapered truncated parabola of width proportional to $s$.
The symbol $\star$ denotes convolution.

\subsection{Parameterization of spectral structure}
The spectrum of each flux component is modeled by a polynomial in frequency
( a Taylor series expansion about $\nu_0$ ).
\begin{equation}
I^{obs}_{\nu} = \sum_{t=0}^{N_t} \dnuno^t I^m_t 
\label{EQN_TS}
\end{equation}
where $N_t$ is the order of the Taylor series expansion, and 
the $I^m_t$ represent multi-scale Taylor coefficient images (moment maps ??? ).
This decomposition is linear in the coefficients as well as in the basis functions.

These Taylor coefficients can be interpreted by choosing an astro-physically appropriate
spectral model and performing a Taylor expansion to derive expressions that each coefficient
maps to.
One choice for a spectral model is a power law with a varying index, represented by a 
second-order polynomial in $log(I)~vs~log\nuno$ space.
The variation of the spectral index with frequency is described by an average spectral
index $\alpha$ and a curvature term $\beta$.
\begin{equation}
I_{\nu}^{obs} = I_{\nu_0}^{obs} \nuno^{\alpha + \beta log \nuno}
\label{EQN_POWERLAW1}
\end{equation}
The main reason behind this choice is the fact that continuum synchrotron emission is 
usually modeled (and observed) as a power law distribution with frequency. Across the wide
frequency ranges that new receivers are now sensitive to, 
spectral breaks, steepening and turnovers need to be factored into
models, and the simplest way to include them and ensure smoothness, is spectral curvature.
(FIGURE).
%% Figure Caption
%% (1) Synchrotron spectrum with steepening, turnover, break.
(Wideband imaging algorithms describes in CCW/SW use a fixed spectral index across the band,
and handle slight curvature by performing multiple rounds of imaging after removing the 
dominant/average $\alpha$ at each stage. 
They also suggest using higher order polynomials to handle spectral curvature.)

A Taylor expansion of Eqn \ref{EQN_POWERLAW1} yields the following expressions for the first
three coefficients ($t=0,1,2$ in Eqn \ref{EQN_TS}) from which the spectral index $\alpha$ and curvature $\beta$ images 
can be computed algebraically.
\begin{equation}
I^m_0 = I^{obs}_{\nu_0} ~~;~~~ I^m_1 = \alpha I^{obs}_{\nu_0} ~~;~~~ I^m_2 = \left(\frac{\alpha(\alpha-1)}{2} + \beta\right) I^{obs}_{\nu_0}
\label{EQN_COEFFS}
\end{equation}
Note that with this choice of parameterization, 
we are using a polynomial to model an exponential, and this has caveats, etc which
are discussed in the section on errors.
Also, note that there can be other interpretations of the Taylor coefficients %poln ?
and other expansions. 
A power-series expansion about $\alpha$ and $\beta$ will yield a logarithmic
expansion i.e. I vs log nu. (CHECK if/when this is better)
It is however impractical to work directly in Log I and Log nu space because that
involves taking logs and exp of image pixel amplitudes and this is highly unstable numerically.


\subsubsection{Frequency-dependant instrumental effects}
Several instrumental effects are also frequency dependent, and need to be modeled/measured and
corrected for.
The field-of-view of the telescope depends on the physical size of the collector
in units of signal wavelength
(given by $\lambda/D$) and is therefore frequency dependent. Figure (FIGURE) shows the
the shape of the element response function (primary beam) of an EVLA antenna at multiple
frequencies. 
%% Figure Caption 
%% (1) HPBW at 1 GHz is near the 5% point at 2 GHz.
%% (2) Continuum sensitivity has no nulls until quite far out.
%% (3) HPBW of ref freq has a spectral index of about 1.4 (also derived in SW)
For a source away from the pointing-centre of these beams, the coefficient images now 
represent the sky and its spectrum, multiplied by the frequency dependent primary beam.  
The artificial spectral structure introduced by the antenna response pattern can also be
modeled by a polynomial in frequency, and divided out of coefficient images to correct 
them. A polynomial model for the primary beam can be either measured, or pre-computed.

%The continuum, spectral index and curvature images computed from Eqn \ref{EQN_COEFFS}
%can then be interpreted as 
%\begin{equation}
%I^{obs}_{\nu_0} = I^{sky}_{\nu_0}P_{\nu_0}~~; ~~~~ \alpha = \alpha_{sky} + \alpha_{pb}~~; ~~~~ 
%\beta = \beta_{sky} + \beta_{pb}
%\end{equation}
%where $P_{\nu_0}, \alpha_{pb}, \beta_{pb}$ describe the 
%effective spectrum of each direction on the sky due to the primary beam and
%can be measured and/or pre-computed. 
%The only reason for this choice is
%that since it is the same as the sky flux, the correction of spectral effects becomes linear
%and this controls error propagation.
%and the results can be corrected after the Minor cycle of deconvolution.
Other frequency-dependent instrumental effects include the w-term, and beam squint for 
off-axis feed systems, all of which are phase ramps in the aperture illumination pattern
and therefore correctable during the imaging process.

%%%%% Implementation

\subsection{Imaging Equations}

The {\it Measurement Equation\footnote
{
The measurement equation of an imaging instrument describes its transfer function (the effect of
the measurement process on the input signal).
For an ideal interferometer (a perfect spatial frequency filter, with no instrumental gains),
it can be written in matrix notation as follows. 
Let $I^{sky}_{m\times 1}$ be a pixelated image of the sky and let $V^{obs}_{n\times 1}$
be a vector of $n$ visibilities. 
Let $S_{n\times m}$ be a projection operator that describes
the uv-coverage as a mapping of $m$ discrete spatial 
frequencies (pixels on a grid) to $n$ visibility samples (usually $n>m$).
Let $F_{m\times m}$ be the Fourier transform operator.
Then, $[{\Sa}_{n\times m}] [F_{m\times m}] \vec{I}^{sky}_{m\times 1} = \vec{V}^{obs}_{n\times 1} $
}
}for a sky brightness distribution parameterized by Eqns.
\ref{EQN_MS}\&\ref{EQN_TS} is given by 
\begin{eqnarray}
V^{obs}_{\nu} &=& \sum_{t=0}^{N_t}\wnt [\Sna][\F] \left\{\sum_{s=0}^{N_s} I^{blob}_s \star I^{m}_{t,s,\delta} \right\} \\
		&=& \sum_{t=0}^{N_t} \sum_{s=0}^{N_s} \wnt [\Sna][T_s] [\F] I^{m}_{t,s,\delta}
%%V^{corr}_{\nu} &=& \sum_{t=0}^{N_t}\wnt [\Sna][\F] [\Pnu] \left\{\sum_{s=0}^{N_s} I^{blob}_s \star I^{m}_{t,s,\delta} \right\} \\
%%		&=& \sum_{t=0}^{N_t} \sum_{s=0}^{N_s} \wnt [\Sna][\Gnu][T_s] [\F] I^{m}_{t,s,\delta}
%%V^{corr}_{nc \times 1} = \sum_{t=0}^{N_t} \sum_{s=0}^{N_s} [\Wnt][\Sa][\F][\Fd T_s \F]I^{sky}_{t,s,\delta}
\end{eqnarray}
where $w_{\nu} = \dnuno$ and $\Sna$ represents the uv-coverage of the synthesis array at frequency $\nu$.
The image-domain convolution with $I^{blob}_s$ is written as a spatial-frequency
Taper function $[T_s]_{m\times m} = \F I^{blob}_s $. 
The image parameters to be solved for are the Taylor coefficients at each spatial scale
($I^{model}_{t,s,\delta}$ for all $t,s$), and data from all frequencies are used simultaneously
in the solution process.

The {\it Normal Equations\footnote
{
The Normal Equations are the linear system of equations whose solution 
gives a weighted least-squares estimate of a set of parameters in a model  
($\chi^2$ minimization). 
For an ideal interferometer, it is given by 
$[\Fd \Sd W \Sa \F ] I^{sky}_{m\times 1} = [\Fd \Sd W] V^{obs}_{n\times 1} = I^{dirty}_{m\times 1}$
where $W_{n\times n}$ is a diagonal matrix of signal-to-noise based measurement weights
and $\Sd$ denotes the mapping of measured visibilities onto a spatial frequency grid.
The Hessian (matrix on the LHS) is a {\it convolution} operator with 
$I^{psf}_{m\times 1} = diag[\Fd \Sd W \Sa]$ in each row. 
The {\it dirty} image on the RHS is therefore a result of a convolution of $I^{sky}_{m\times 1}$ with $I^{psf}$
and these equations can be solved by a {\it deconvolution}. 
}
}for each scale size $s$, and Taylor term $t$ are given by
\begin{equation}
\sum_{p=0}^{N_t}\sum_{q=0}^{N_s} H_{t,s,p,q} I^{m}_{p,q,\delta} = I^{dirty}_{s,t}
\label{HESS}
\end{equation}
\begin{eqnarray}
\label{HESS1}
%%%\mathrm{where} ~~~~ H_{t,s,p,q} &=& [\Fd T_s] \left\{\sum_{\nu} \wntp [\Snd\Wimn\Sna] \right\} [T_q \F] \\
\mathrm{where} ~~~~ H_{t,s,p,q} &=& \sum_{\nu} \wntp [\Fd T_s \Snd\Wimn\Sna T_q \F] \\
%%where ~~~~ H_{t,s,p,q} &=& \sum_{\nu} \wntp [\Fd T_s \Gdnu\Snd\Wimn\Gnu\Sna T_q \F] \\
%	&=& I^{blob}_s \star I^{psf}_{im,t,p} \star I^{blob}_q \\
\label{HESS2}
%%%\mathrm{and} ~~~~ I^{dirty}_{s,t} &=& [\Fd T_s] \left\{ \sum_{\nu}\wnt[\Snd\Wimn] V^{corr}_{\nu} \right\}
I^{dirty}_{s,t} &=&  \sum_{\nu}\wnt[\Fd T_s \Snd\Wimn] V^{corr}_{\nu} 
%%I^{dirty}_{s,t} &=&  \sum_{\nu}\wnt[\Fd T_s \Gdnu \Snd\Wimn] V^{corr}_{\nu} 
%	&=& I^{blob}_s \star I^{dirty}_t
\end{eqnarray}
%where $\Wimn_{n\times n}$ is a diagonal matrix of signal-to-noise based measurement weights for channel $\nu$
%and $\Snd$ denotes the mapping of measured visibilities onto a spatial frequency grid.
%and $I^{dirty}_{s,t} = [\Fd T_s \F] [\Fd\Sd\Wntd\Wim] V^{corr}$


%\begin{figure}[!t]\label{NEqn With PB}
%\epsfig{figure=pics/NEqn.cropped.pb.yes.eps,scale=0.4}
%\caption{Normal Equations for MFS Imaging with a Frequency Dependent Field of View}
%\label{FIG_NEQN_2}
%\end{figure}

Fig.\ref{FIG_NEQN_1} represents the Normal Equations for a three-term Taylor series
$N_t = 3$ and one spatial scale $N_s = 1$. 
	$I^{sky}_{3m\times 1}$ is a 1-D model of two point sources on an empty sky. 
	The three segments correspond to Taylor coefficients $I^{m}_p$ for $p=0,1,2$.
	The Hessian/Beam matrix on the LHS is comprised of $3 \times 3$ blocks each
	of size $m\times m$. Each block is a Toeplitz operator (constructed from spectral PSFs)
	that implements the shift/multiply/add sequence of convolution
	when applied to the corresponding segment of $I^{m}_t$.
	%The RHS vector represents the dirty image formed from direct Fourier inversion
	%of Taylor-weighted measured visibilities. 
	These equations show how the {\it dirty} image on the RHS can be
	written as a linear combination of convolutions of Taylor coefficient images with
	spectral PSFs. Each segment on the RHS is given by
	$I^{dirty}_t = \sum_{p=0}^{3} I^{psf}_{t+p} \star I^{m}_p$.
	Note : The equations corresponding to $t=0,p=0$ represent the standard convolution
	equation in synthesis imaging.

When all scales and Taylor terms are combined, 
the LHS of the Normal equations is made up of the full Hessian matrix with
$N_t N_s \times N_t N_s$ blocks each of size $m\times m$ and 
containing information from all frequency channels, 
and $N_t$ Taylor coefficient images each of size $m\times 1$.
Each block described by Eqn.\ref{HESS1} is a convolution operator with a {\it spectral} PSF of order $t+p$ 
in each row. 
The Spectral PSF $ I^{psf}_{t} = \sum_{\nu} \wnt I^{psf}_{im,\nu}$
describes the instrument response functions to a point source with Taylor-polynomial spectra of order $t$.
Similarly, the spatial PSF $ I^{psf}_s = I^{psf}_{im} \star I^{blob}_s $ is the instruments 
response function to a spatial scale $s$ and unit integrated flux.
%This taper gives lower spatial frequencies a relatively high weight compared to higher spatial frequencies,
%to tune the instrument's response towards a spatial scale larger than the
%angular resolution of the telescope.
Off-diagonal blocks represent the degree of coupling (non-orthogonality)
between the various spatial and spectral basis functions.
The RHS of this system of equations is a collection of {\it dirty} images obtained by
direct Fourier inversion of Taylor-weighted visibilities.

\begin{figure}[]
\epsfig{figure=pics/NEqn.cropped.pb.no.eps,scale=0.4}
\caption{A 1-D representation of the Normal Equations for MFS imaging with a $2^{nd}$ order
	Taylor expansion in frequency($N_t=3$) and a point-source flux model (no multi-scale).
%	$I^{sky}_{3m\times 1}$ is a 1-D model of two point sources on an empty sky. 
%	The three segments correspond to Taylor coefficients $I^{m}_p$ for $p=0,1,2$.
%	The Hessian/Beam matrix on the LHS is comprised of $3 \times 3$ blocks each
%	of size $m\times m$. Each block is a Toeplitz operator (constructed from spectral PSFs)
%	that implements the shift/multiply/add sequence of convolution
%	when applied to the corresponding segment of $I^{m}_t$.
%	%The RHS vector represents the dirty image formed from direct Fourier inversion
%	%of Taylor-weighted measured visibilities. 
%	These equations show how the {\it dirty} image on the RHS can be
%	written as a linear combination of convolutions of Taylor coefficient images with
%	spectral PSFs. Each segment on the RHS is given by
%	$I^{dirty}_t = \sum_{p=0}^{3} I^{psf}_{t+p} \star I^{m}_p$.
%	Note : The equations corresponding to $t=0,p=0$ represent the standard convolution
%	equation in synthesis imaging.
}
\label{FIG_NEQN_1}
\end{figure}

The Normal Equations are solved in two stages, following a CLEAN-type approach.
The parameters of the flux components are first solved for one component at a time.
Then, the Normal equations are used to calculate the contribution of these flux components
to the dirty image, which is then subtracted out.

For the purpose of finding flux components, several assumptions can be made about the Hessian.
First, each scale can be treated separately as in MS-CLEAN, and Hessian blocks that couple
different spatial scales are ignored. Each scale then has $[N_t \times N_t]$ blocks of size
$m\times m$ as shows in Fig.\ref{FIG_NEQN_1}.
Second, pixels are assumed to be independant of each other and a 
diagonal approximation to the Hessian blocks yields a single $N_t \times N_t$ matrix 
that applies to all pixels. The $t,s$ elements of this matrix come from the values of
$I^{psf}_{t+p}$, read off at the location of the peak of $I^{psf}_0$. These numbers
correspond to the sum of weights $\sum_{\nu} \wntp$.
These N.E. are the result of autocorrelations of Spectral PSFs and are 
(usually) guaranteed to be non-singular, and exactly solvable via an LU decomposition.
Taylor coefficients for each location in the image are obtained by
solving this $N_t \times N_t$ matrix for every component location.

The Normal equations are then evaluated for the components with the dominant Taylor0
coefficient (or picked out some other way), and the effect of that component is
subtracted out of the RHS vector. This is the update step in a formal iterative
optimization.

%\item When direction-dependant effects are included, the blocks in $H$ are no longer 
%convolution operators, and each row is the PSF scaled and multiplied by the PB.\\ 
%$H_{i,j} = P(i-i_{mid}) P(j-j_{mid}) I^{psf}(j-j_{mid}-i)$. 
%Therefore, can still compute only one PSF per block, and multiply and scale appropriately
%when updating the Residuals in the Minor Cycle.

The relation between this implementation and that of the SW MF-CLEAN is described in Appendix\ref{APP_A}.

\subsection{Algorithm Steps}
\subsubsection{Pre-compute Hessian}  
All $N_s N_t \times N_s N_t$ terms in the Hessian ($t,p \in [0,N_t]; s,q \in [0,N_s]$) are 
evaluated as in Eqn.\ref{HESS}.
%\begin{equation}
%[\Fd T_s T_q \F]I^{psf}_{im,t,p} = I^{blob}_{s} \star I^{psf}_{im,t,p} \star I^{blob}_{q} 
%\end{equation}
%where 
\begin{eqnarray}
H_{t,s,p,q} &=& I^{blob}_s \star \left\{ \sum_{\nu} \wntp I^{psf}_{im,\nu} \right\} \star I^{blob}_q \\
 I^{psf}_{im,\nu} &=& [I^{wt}_{pc}]^{-1} [ \Fd \G^{pc} \Snd \Wimn]  V^{1}
\end{eqnarray}
where $I^{wt}_{pc} = \Fd P_s$, where $P_s$ is an anti-aliasing function,
and $\G^{pc} = [\F I^{wt}_{pc}\Fd]$ is the corresponding gridding convolution operator.

Since each block is a Toeplitz matrix constructed from spectral and spatial PSFs, only one
row per block need be computed.

The values at the pixel at which the
{\it peak} of $I^{psf}_{im}$ occurs,
are separately recorded as $N_t\times N_t$ matrices ($H^{peak}_{t,s,p,s}$) for each scale $s$.

Note : Normalize all by the $w_{sum}$ of Taylor 0.

\subsubsection{Pre-compute Primary Beam} 
Taylor coefficients for a polynomial expansion of the PB across frequency are obtained
by applying the inverse of $H^{peak}$ to RHS functions computed as follows.
\begin{eqnarray}
\sum_{p=0}^{N_t} H_{t,p} P^{dd}_{p} &=& I^{wt}_{dd,t} \\
I^{wt}_{dd,t} = \sum_{\nu} \wnt I^{wt}_{dd,\nu} &=& \sum_{\nu} \wnt{ [\Fd \G_{\nu}^{{dd}^{\dag}} \Wimn \G_{\nu}^{dd} F] }
\end{eqnarray}
where $G_{\nu} = [\F P_{\nu} \Fd]$.
The effective $\alpha$ and $\beta$ in each direction on the sky, are computed
from Taylor coefficients. 
\begin{enumerate}
\item Note that $P^{dd}_{\nu_0},\alpha_{pb},\beta_{pb}$ as computed
directly from the coefficients have $PB^2$ in it, so take the sqrt of the PB and
divide the alpha and beta by two.
\item The PB and its frequency dependence are computed via weighted averages, using
the same imaging weights as used in computing the dirty/residual images. Also, baseline-dependent 
PBs at different parallactic angles can also be accounted for in this weighted average.
Therefore, $P^{dd}_{\nu_0},\alpha_{pb},\beta_{pb}$ represent the collective effect on the
final image.
\end{enumerate}

\subsubsection{Major and Minor Cycles}
The Normal Equations are solved iteratively by repeating steps \ref{stepRHS} to \ref{stepPredict} until 
some termination criterion is reached.
Steps \ref{stepRHS} and \ref{stepPredict} form one Major Cycle, and repetitions of Steps \ref{stepFind} and \ref{stepUpdate} form the Minor Cycle.

\begin{enumerate}
\item\label{stepRHS} {\bf Compute RHS} : Residual images for each 
$s\in[0,N_s]$ and $t\in[0,N_t]$ are computed as in Eqn.\ref{HESS2}.
\begin{eqnarray}
I^{dirty}_{s,t} &=& I^{blob}_s \star \left\{ \sum_{\nu} \wnt I^{dirty}_{\nu} \right\} \\
I^{dirty}_{\nu} &=& [\Pnu]^{-1} [I^{wt}_{pc}]^{-1} [ \Fd \G_{\nu}^{{dd}^{\dag}} \G^{pc} \Snd \Wimn ]  V^{residual}_{\nu}
\end{eqnarray}
where $V^{residual}_{\nu} = V^{corr}_{\nu}$ for the first Major cycle iteration.

%The normalization by $\Pnu$ takes out one instance of the PB
%and is equivalent to grid-correction.

$\G^{pc}$ and $\Gddnu$ are convolution operators used during gridding.
$\G^{pc}$ applies an anti-aliasing function, and $\Gddnu$ corrects for direction-dependent
phase effects (w-projection, pointing offsets and squint).
%Different versions are used in different
%variants of this process as discussed in section \ref{VARIANTS}.

%%\item\label{stepGridCorrect} {\bf Grid correction} : to take out one PB.


\item\label{stepFind} {\bf Find a Flux Component} :
For each scale $s$, the $N_t\times N_t$ Hessian matrix given by $H_{t,p,s,s}$
is inverted and applied to all pixels of the corresponding RHS vector 
($I^{dirty}_{t,s};s\in[0,N_s]$), to generate
solution sets $\{I^{model}_{p,\delta}\};p\in[0,N_t]$ for each $s\in[0,N_s]$. 
The $N_t$ element solution set of Taylor coefficients
with the dominant $p=0$ component across all scales and pixel locations,
is chosen the current flux component. 
Let the scale size for this set be $q$. 
The chosen solution set is given by $\{I^{model}_{p,q,\delta}\};p\in[0,N_t]$.

\item\label{stepUpdate} {\bf Updates }: 
Multiscale model images for each Taylor coefficient, and dirty images (RHS,residuals) are updated as 
\begin{equation}
I^{model}_p += I^{model}_{p,q,\delta} \star I^{blob}_q ~~~~~~~~ \forall p\in[0,N_t]
\end{equation}
\begin{equation}
I^{dirty}_{t,s} -= \sum_{p=0}^{N_t} \left[ I^{blob}_{s} \star I^{psf}_{im,t,p} \star I^{blob}_{q} \star I^{model}_{p,q,\delta} \right] ~~~~~~~~ \forall t,s
%= \sum_{p=0}^{N_t} \left[[\Fd T_s T_q \F]I^{psf}_{im,t,p}\right]\star I^{model}_{q,p,\delta}
\end{equation}

\item\label{stepCorrectPB} {\bf Correct for PB}: 
The model returned from the Minor cycle contains one instance of the PB along with its
frequency dependence, and this needs to be corrected before prediction. 
The model image is divided by the P0, the alpha and beta are corrected by subtracting
the alpha and beta of the PB.  The taylor coefficients are then recomputed using the
corrected PB0, alpha and beta.
\begin{eqnarray}
I^{skymodel}_0 &=& I^{model}_0 / P_{\nu_0} \\
I^{skymodel}_1 &=& I^{sky}_0 \left(\frac{I^{model}_1}{I^{model}_0} - \alpha_{PB}\right)\\
\cdots & & \forall p\in[0,N_t]
\end{eqnarray}

\item\label{stepPredict} {\bf Predict }: Model visibilities are computed as 
\begin{equation}
V^{model}_{\nu} = \sum_{p=0}^{N_t} \wnt [\Sna] [\G_{\nu}^{{dd}}] [\F]I^{skymodel}_{p}
%V^{model}_{nc\times 1} = \sum_{t=0}^{N_t}[\Wnt][\Sa \G^{pc}\F][I^{pc}]^{-1} I^{model}_{t}
\end{equation}
The PB-corrected models get back a PB in the predicted visibilities, 
    making them read to compare with the data.
Residual visibilities are computed as $V_{\nu}^{residual} = V_{\nu}^{corr} - V_{\nu}^{model}$ and
then processed as in Step \ref{stepRHS} to construct the residual image for the next Major cycle.

\subsubsection{Restoration / Trivial Solution}
Model images are usually smoothed by the {\it Clean Beam} and the residuals are added back in, to
create the final restored image. With MFS however, this step is different. 
Taylor coefficients need to be constructed from the residual images before they are added back in.

This is also relevant for the trivial solution - 
the case where deconvolution is not done, and where the dirty images
computed in step \ref{stepRHS} can be used as is, to compute Taylor coefficients. 
The $N_t \times N_t$ matrix formed from the Hessian peaks for the zero-scale $H^{peak}_{t,0,p,0}$,
is inverted and applied to all pixels of the images computed in step \ref{stepRHS}.
This step needs to be followed by a further correction by PB and its alpha and beta.

When there is no deconvolution, the Hessian blocks are diagonal, and the full hessian with
the PB is equivalent to multiplying the image with PB square..

\end{enumerate}

\subsection{Error Analysis}
\begin{enumerate}
\item Magnitude of diff Taylor terms :
Are they visible in image as errors (vs) SNR needed to solve.
\item Error bars on alpha and beta due to SNR
\item Error due to using too few Taylor terms to fit the exp with a poly.
CCW comment on a bias that occurs with a 2-term T-exp. This is just the use of insufficient terms of a polynomial to model an exponential.
\item Error due to too many Taylor terms for low SNR.
\item Expansion about $\nuno$ (logarithmic PSFs)
vs expand about $\alpha$ and $\beta$ (Taylor PSFs).
% NOTE : Can model it as a polynomial in I vs log(nu) space too.
% Cannot model it as a poly in log(I) vs log(nu) space because computing log(I) is a pain.
\item Spectrum of extended emission... artificially steepened due to short-spacing problem.
-- error is a certain amount, based on the percentage of flux measured at each freq.... 
\item Sources that are unresolved at one end of the band, but resolved at the other.
\item Wide-Band flux calibration : required accuracy.
\item Direction dependent sensitivity (Weight Image) (sources can pass through a Null at some freq and have less continuum sensitivity.)
\item Error in alpha due to uncorrected freq-dep primary beam.
\item Freq-dep PB correction with non-homogenous arrays : the Weight image is a weighted average
across all baselines,times, frequencies and represents a collective effect.
\end{enumerate}


\subsection{Variants}\label{VARIANTS}
The discussions in the preceding sections were based on a single pointing observation,
and a data-access pattern that treats frequencies independently (suitable for
data-partitioning by frequency in a parallel implementation). Also, the Minor cycle
operates on residual images that satisfy the convolution equation $PSF \star (P I^{sky})$, 
so that the update step in the Minor cycle is accurate. This is a flat-noise model, in that
the residual image is a signal-to-noise map and the model image represents $(P I^{sky}$.

The following are variants of this approach.
\begin{enumerate}
\item Using a Multi-Freq UV grid (P and 2 alpha) - This is an approximation in that the minor cycle update cannot be accurate, but the major cycle will be accurate. Perhaps useful when you need to touch all frequencies at once (mem-access-patterns).
\item Flat Sky vs Flat Noise (Mosaicing, Non-homogeneous arrays (Different PB for each antenna))
\end{enumerate}

\section{Hybrid Algorithm}\label{HYBRID}
This section describes a simple procedure for wide-band imaging as an extension of the traditional
single-channel imaging, and discusses situations in which this may be appropriate.

The simplest way to deal with a frequency-dependent sky is to image each frequency channel
separately, and then compute spectral characteristics. The advantage is that is does not depend
on any spectral flux model, can handle arbitrary spectral shapes, and is straightforward to
implement in code and parallelize. 
However, the accuracy of single-channel imaging depends on the uv-coverage obtained at one
single frequency (or a range of frequencies over which the sky is spectrally flat), so this method
is appropriate only for interferometers where the narrow-band coverage is sufficient to sample
all relevant spatial scales of a source.  
Further, the narrow-band imaging sensitivity can be 
significantly less than what the broadband instrument is capable of, but 
in some cases, a hybrid approach between single-channel and continuum imaging can be used to
overcome this problem. 

\subsection{Procedure and Error Analysis}
\begin{enumerate}
\item Deconvolve each channel separately, and Clean down only to the single-channel sensitivity limit
$\sigma_{chan}$. 
\item Compute the continuum residual image, 
assume a flat spectrum, and clean down to the continuum sensitivity limit 
$\sigma_{cont} = {\sigma_{chan}}/{\sqrt{N_{chan}}}$.  
\end{enumerate}
When does the second stage work ? 
For bright sources visible in the single-channel maps, the spectral structure is detected and
removed by the end of Step 1, leaving flat-spectrum residuals that justify the assumption in
Step 2. 
Weak sources whose flux values lie below $\sigma_{chan}$ but above $\sigma_{cont}$ 
are detected only in Step 2, and the flat-spectrum assumption may be a problem for their
deconvolution. However, if these spectral errors are less than $\sigma_{cont}$, it will not
matter. CCW have shown that for a spectral index of -1.0 across a 2:1 band, spectral errors
are visible beyond a dynamic range of 1000. Therefore, as long as 
$\sigma_{chan}/\sigma_{cont} < 1000$,
the errors should not be visible. This implies $\sqrt{N_{chan}} < 1000$ which will almost
always be satisfied.

This method is straightforward for simple fields of compact sources where the single-channel
uv-coverage is sufficient to fully reconstruct the source structure. 
For complicated fields, or insufficient uv-coverage, this method is highly susceptible to
deconvolution errors that leave single-channel residuals that neither add up coherently in
Step 2, nor are random enough to average down by $\sqrt{N_{chan}}$. CCW also comment on this.

If there is sufficient uv-coverage per channel, then low-level undeconvolved flux can add up
coherently, and continuum residuals will satisfy a convolution equation that Step 2 can solve.
This idea has been tested on EVLA simulations, but it yet to be verified on real wide-band data
where all frequencies are measured simultaneously.

Since this is an inherently single-channel approach, the spectral structure can be obtained only
at the angular resolution allowed by the lowest frequency. The field of view is not a restriction,
but the continuum sensitivity degrades beyond the field of view given by the highest frequency.



%__________________________________________________________________

\section{Wide-band ASP-Clean}
\begin{enumerate}
\item Add a frequency dependence to each ASP. Simplest is a polynomial amplitude variation of each Gaussian in the image fit.
\item Can analyse N-unknowns and N-equations.
\item Need at least 2 measurements per Gaussian, and 2 freqs - to fit its scale and spectral params. This implies that it can work for very sparse sampling, eMerlin, VLBI, etc..
%For example, consider a flux parameterization of Gaussians with amplitudes given 
%by a polynomial spectrum. For extended emission, the spatial and spectral parameters 
%of a Gaussian can be fit as long as an adequate number of measurements ($>$ N parameters)
%are made in the spatial frequency range where the visibility function of the Gaussian is non-zero
%at at least two frequencies.
\item For moderately resolved sources, it gets multiple high-res Gaussians that satisfy the measured SUM at the lower ends.
\item Any limiting case for this ?? 
\item Some cases can get ambigious.
\item Very high dynamic range is difficult - since the fitting is done in the visibility domain. Good for extended emission. Difficult to get weak point source on top of extended emission.
\end{enumerate}

%__________________________________________________________________

\section{Simulations}\label{SIMULATIONS}
\begin{enumerate}
\item True Image : Range of spatial scales and 2nd order log spectra 
\item Data : Full range of scales is sampled only by multi-freq uv-coverage. One source is moderately resolved.
\item Without primary-beam : Comparison between MF-Clean (with 3terms), MS-MFS, Hybrid  in terms of 
image fidelity, angular resolution of spectral images and error-bars on alpha/beta.
\item With primary beam : Comparison between MS-MFS without and with primary-beam correction,
	sensitivity across field-of-view.  Use Data simulated with primary beam where
	there are sources outside the hpbw of the highest frequency, 
	but within the hpbw of the lowest frequency.  
	Also have one source near the null at some frequency.
\end{enumerate}

%__________________________________________________________________

\section{Data}\label{DATA}
\subsection{Narrow-field CygA, no Primary Beam Correction}
\begin{enumerate}
\item Data : Sparse single-freq uv-coverage at 9 frequencies.
\item Calibration (wideband)
\item Imaging Results : Hybrid vs MS-MFS (Continuum Image, Spectral Index Image,
Curvature ?) %for high SNR parts)
\end{enumerate}
\subsection{Wide-field 3C286, Primary Beam Correction}
\begin{enumerate}
\item Data : wide field of unresolved sources.
\item Calibration : PBMos + WProj + wideband model of 3C286
\item Imaging results : Continuum and spectral index images, without and with Primary Beam Correction
\end{enumerate}

%__________________________________________________________________

\section{Discussion}
\begin{enumerate}
\item { Summary : Can do this'n'that (image and spectrum reconstruction + astrophysical parameters)}
...at the angular resolution defined at the highest measured frequency.
\item Single-Channel Imaging Hybrid vs MS-MFS
\item Astrophysics : A variety of astrophysical observations could gain from these new instruments due to the 
increased sensitivity and large instantaneous bandwidth. Synchrotron spectra can be measured
within the instantaneous bandwidth, to detect/measure turnovers/breaks.
Snapshot imaging of highly variable sources 
(sunspots, supernovae, AGN) can be improved with the increased uv-coverage of mfs while
deriving the instantaneous spectra at multiple spatial scales.
Weak supernova remnants could be detected against a continuum background by their spectral signature.
Spectral information can be obtained at the highest angular resolution. This is especially 
significant for moderately resolved sources. 
The atmospheres of stars have an angular size of xxx and are resolvable at xxx GHz in the middle
of the EVLA frequency coverage. MFS with data between xx and yy GHz could yield wideband spectra
at the angular resolution defined by yy GHz.

Sunspot flares - magnetic loops - frequency probes different depths - band-limited signals.

VLBI position offsets as a function of frequency - can be resolved (defined) via MFS.

Multiple arrays can be combined by matching spatial frequency coverage across wide frequency ranges.
For example, EVLA-D between 5 and 50 GHz would match with EVLA-C in L and S bands... (as has always
been done, but it can be imaged together - not sure what final advantage this will bring though..
Snapshot imaging with EVLA subarrays could also achieve this.

\item Software (CASA and ASKAPSOFT), Cost and Parallelization (Hybrid vs MSMFS)
\item { Limitations }
- moderately resolved sources will need simultaneous deconvolution of multipel components (ASP)
- unresolved at lower freq, resolved at higher - 1D simulations show that it is possible to reconstruct, if at least XXX fraction of the bw has resolved info.
\item { Future Directions }
Out into the null of the PB (alternate parameterisations).
Full-Stokes imaging (ref SW)... parameterize differently in lm and freq but same idea.
\end{enumerate}



%__________________________________________________________________

\begin{acknowledgements}
      Thank you ! RJS, SB, RN, DW, FO, BC, 
\end{acknowledgements}

\begin{thebibliography}{}

  \bibitem[1994]{MFCLEAN} R.J. Sault and M.H. Wieringa... "
\end{thebibliography}

\begin{appendix}
\section{Relation to MF-CLEAN}\label{APP_A}


MF-CLEAN is formulated as a least-square method, but its implementation follows a matched-filtering method
where the Hessian and RHS are computed as post-gridding convolutions of images with point-spread functions.
(CHECK) . The two methods are equivalent only when mf-gridding is ignored.
%, and there too, the convo is with sqrt(imwt) in each PSF.
By computing the H and RHS as the equations dictate, 
there is a notable improvement in stability of the algorithm, and normalizations work out exactly.
%Also - this is reflected in a different way of computing Hessian elements as conv of PSFs vs a single PSF with the summed taylor-indices.

MF-Clean does the following : $I^{res}_t = I^{res}_0 \star B_t$ where $B_t = [\Fd]\left\{ \sum_{\nu} \wnt [\Snd][\Wimnd]V^{1} \right\} $. Also, the Hessian elements are computed as $H_{t,p} = B_t \star B_p$. 
Computed this way, $B_t \star B_p = B_{t+p}$ {\bf only} when $\Wimn$ is the 
Identity matrix (natural weighting) and 
$\Sna$ is a pseudo-unitary Boolean operator, or the UV-coverage of the array
is mutually independent across frequencies. 
This can be understood by thinking about when the following could be true 
$\left\{ \sum_{\nu} \wnt [\Snd\Wimnd] \right\}\left\{ \sum_{\nu} \wnp [\Snd\Wimnd]\right\} = \left\{ \sum_{\nu} \wntp [\Snd\Wimnd]\right\}$.

\end{appendix}



\end{document}



\message{ !name(msmfs_aa.tex) !offset(-962) }
