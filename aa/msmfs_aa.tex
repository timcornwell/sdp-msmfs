
%                                                                 aa.dem
% AA vers. 6.1, LaTeX class for Astronomy & Astrophysics
% demonstration file
%                                                 (c) Springer-Verlag HD
%                                                revised by EDP Sciences
%-----------------------------------------------------------------------
%
%\documentclass[referee]{aa} % for a referee version
%\documentclass[onecolumn]{aa} % for a paper on 1 column  
%\documentclass[longauth]{aa} % for the long lists of affiliations 
%\documentclass[rnote]{aa} % for the research notes
%\documentclass[letter]{aa} % for the letters 
%
\documentclass[structabstract]{stylefiles/aa}  
%\documentclass[traditabstract]{aa} % for the abstract without structuration 
                                   % (traditional abstract) 

\usepackage[boxruled]{stylefiles/algorithm2e}
\usepackage[dvips]{graphicx, color}
\usepackage{epsfig}
\usepackage{mdwlist}
%\usepackage{amsmath}
%\usepackage{mathptmx}
\usepackage{txfonts}
\setcounter{secnumdepth}{3}

\usepackage{stylefiles/fmtcount}
\usepackage{natbib}
\bibpunct{[}{]}{;}{a}{}{;}
\renewcommand{\cite}{\citep}



\usepackage{varioref}


\newcommand{\nuno}{{\left(\frac{\nu}{\nu_0}\right)}}
\newcommand{\dnuno}{{\left(\frac{\nu-\nu_0}{\nu_0}\right)}}

\newcommand{\dg}{^\dag}
\newcommand{\X}{\vec{x}}
\newcommand{\Xd}{\vec{{x}^\dag}}
\newcommand{\B}{\vec{b}}
\newcommand{\Bd}{\vec{b^\dag}}
\newcommand{\A}{{\tens{A}}}
\newcommand{\Ad}{{\tens{A^\dag}}}
\newcommand{\F}{{\tens{F}}}
\newcommand{\Fd}{{\tens{F^\dag}}}
\newcommand{\He}{{\tens{H}}}
\newcommand{\Sa}{{\tens{S}}}
\newcommand{\Sd}{{\tens{S^\dag}}}
\newcommand{\Sna}{\tens{{S_{\nu}}}}
\newcommand{\Snd}{\tens{{S_{\nu}^\dag}}}
\newcommand{\T}{{\tens{T}}}
\newcommand{\W}{{\tens{W}}}
\newcommand{\Wd}{{\tens{W^\dag}}}
\newcommand{\Pb}{{P_b}}

\newcommand{\Wim}{{\tens{W^{im}}}}
\newcommand{\Wimd}{{\tens{{W^{im}}^\dag}}}
\newcommand{\Wnt}{{\tens{W^{mfs}_t}}}
\newcommand{\Wntd}{{\tens{{W^{mfs}_t}^\dag}}}
\newcommand{\Wnp}{{\tens{W^{mfs}_p}}}
\newcommand{\Wnpd}{{\tens{{W^{mfs}_p}^\dag}}}
\newcommand{\Wnq}{{\tens{W^{mfs}_q}}}
\newcommand{\Wnqd}{{\tens{{W^{mfs}_q}^\dag}}}
\newcommand{\Wimn}{{\tens{W^{im}_{\nu}}}}
\newcommand{\Wimnd}{{\tens{{W^{im}_{\nu}}^\dag}}}

\newcommand{\wnt}{{w_{\nu}^t}}
\newcommand{\wnq}{{w_{\nu}^q}}
\newcommand{\wntq}{{w_{\nu}^{t+q}}}
\newcommand{\Wntn}{{\tens{w^{mfs}_{t,\nu}}}}
\newcommand{\Wntnd}{{\tens{{w^{mfs}_{t,\nu}}^\dag}}}
\newcommand{\Wnpn}{{\tens{W^{mfs}_{p,\nu}}}}
\newcommand{\Wnpnd}{{\tens{{W^{mfs}_{p,\nu}}^\dag}}}

\newcommand{\pd}{{\partial}}
\newcommand{\mi}{{m_{I}}}
\newcommand{\R}{{R}}
\newcommand{\Rd}{{R^\dag}}
\newcommand{\I}{{\vec{I}}}


%\usepackage{amsmath}
%%%%%%%%%%%%%%%%%%%%%%%%%%%%%%%%%%%%%%%%
%
\begin{document}
%\tableofcontents
%\newpage
%
   \title{A multi-scale multi-frequency deconvolution algorithm
for synthesis imaging in radio interferometry}

%   \subtitle{I. Overviewing the $\kappa$-mechanism}

   \author{U. Rau
          \inst{1}
          \and
          T.J. Cornwell \inst{2} %\fnmsep \thanks{Just to show the usage of the elements in the author field}
          }

   \institute{ National Radio Astronomy Observatory, Socorro, NM, USA\\
              \email{rurvashi@aoc.nrao.edu}
         \and
             Australia Telescope National Facility, CSIRO, Sydney, AU \\
             \email{tim.cornwell@atnf.csiro.au}
%             \thanks{blah.}
             }

   \date{Received XXX XX, 2009; accepted XXX XX, 2009}
   %%\date{Received December 15, 2008; accepted XXX XX, 2009}

% \abstract{}{}{}{}{} 
% 5 {} token are mandatory
 
  \abstract
  % context heading (optional)
  % {} leave it empty if necessary  
   {The use of broad-band receivers in radio interferometery impacts standard continuum
    synthesis imaging in three ways. 
    First, correlations measured at different frequencies sample the visibility function of the
   sky brightness at different spatial frequencies.
    Second, the sky brightness can change significantly across the large frequency 
    range that these receivers are sensitive to.
    Third, direction-dependent instrumental effects such as the element response pattern
     also vary with frequency, and this 
    affects the spectral characteristics of the measured sky brightness during wide-field
    continuum imaging.
    Existing algorithms designed to account for some of the above, can be shown to be
    insufficient to achieve target image fidelity and dynamic range over the wide bands that 
    most new instruments are being built with.}
  % aims heading (mandatory)
   {This paper describes a multi-scale multi-frequency deconvolution algorithm for
    the minor-cycle of 
synthesis imaging that combines the information measured about the source spectrum
    with the additional spatial-frequency-coverage offered by multi-frequency-synthesis to 
    reconstruct source structure.} %, while accounting for frequency-dependent antenna primary-beams.}
  % methods heading (mandatory)
   {
The MS-MFS (multi-scale multi-frequency synthesis) algorithm discussed in this
paper extends the MF-Clean algorithm to reconstruct spectral curvature in 
addition to spectral index and combines it with a multi-scale deconvolution approach
similar to MS-Clean.
%The MS-MFS algorithm discussed in this paper models the wide-band sky-brightness
%     distribution as a linear combination of spatial and spectral basis functions, 
%      and combines a linear-least squares approach along with iterative chi-sq minimization.
%    This method can be used in conjunction with existing wide-field imaging algorithms.
   We also discuss a simpler hybrid of spectral-line and continuum imaging methods
   that may suffice in certain situations.
%    Frequency-dependent fields-of-view are corrected via existing wide-field imaging
%    algorithms.
   }
  % results heading (mandatory)
   {We show via simulations and application to wideband (E)VLA data (1 GHz to 2 GHz),
    that it is possible to reconstruct both spatial and spectral structure of 
    compact and extended emission at the continuum sensitivity level and
    angular resolution allowed by the highest sampled frequency.
   We show imaging examples using simulations, multi-frequency VLA data
    and wideband EVLA data,  to illustrate the capabilities of the MS-MFS algorithm
    and conditions under which these techniques are feasible and give accurate results.
   }
  % conclusions heading (optional), leave it empty if necessary 
   {}

   \keywords{multi-frequency-synthesis --
                multi-scale-deconvolution --
                spectral index --
		spectral curvature --
               }

   \authorrunning{U.Rau and T.J.Cornwell}
   \titlerunning{Multi-Scale Multi-Frequency Synthesis Imaging in Radio Interferometry}
   \maketitle
%

\section{Introduction}

%Instruments like the EVLA and e-MERLIN, are among a new 
%generation of broad band radio interferometers designed to provide continuum sensitivities and
%high dynamic range imaging capabilities superior to that of existing radio interferometers.
%The wide bandwidths will allow detailed measurements of the spectral structure of astrophysical
%sources, but to achieve the desired continuum sensitivities, imaging algorithms need to be 
%sensitive to the effects of combining measurements from a large range of frequencies 
% (namely varying ranges of sampled spatial scales, varying antenna fields-of-view, and
% the spectral structure of the sky brightness distribution).

A new generation of broad-band radio interferometers is currently being
designed and built to provide 
high-dynamic-range imaging capabilities superior to that of existing instruments. 
The large instantaneous bandwidths offered by new front-end systems 
increases the raw continuum sensitivity of the instrument,
%The introduction of broad-band receivers into radio interferometry has
%opens up new opportunities for the study of wide-band continuum
%emission from a vast range of astrophysical objects.  In addition to
%increasing the imaging sensitivity of the instrument, such systems
as well as 
greatly enhances our ability to measure the detailed spectral structure 
of the incident radiation across large continuous bandwidths. 
%
One prominent reason for using broad-band receivers on an imaging
interferometer has been to obtain a continuum image that makes 
use of the increased sensitivity and spatial frequency 
coverage offered by combining measurements from multiple frequencies. 
So far, the bandwidths used have been relatively
narrow ($<25\%$), and effects due to spectral structure of the 
sky-brightness distribution have been considered 
only in the context of reducing errors in the continuum image.% \citep{MFCLEAN_CCW,MFCLEAN}.
Any spectral information is obtained only as a by-product, and attention has not 
been paid to their accuracy.
But now, the new bandwidths ($~100\%$) are large enough to allow the spectral 
structure of the sky brightness distribution 
to also be reconstructed to produce a meaningful astrophysical measurement.   
To do so, 
%while taking full advantage of these instruments, we need imaging 
we need imaging 
algorithms that model and reconstruct both spatial and spectral structure simultaneously,
and that are also 
%sensitive to the frequency dependence of the
%instrument as well as the sky brightness distribution. 
sensitive to various effects of combining measurements from a large range
 of frequencies
 (namely varying ranges of sampled spatial scales and 
varying array-element response functions)

%The use of broad-band receivers in radio interferometery impacts standard continuum
%    synthesis imaging in three ways. 
%\begin{enumerate}
%\item   Correlations measured at different frequencies sample the visibility function of the
%   sky brightness at different spatial frequencies. 
%\item  The sky brightness can change significantly across the large frequency 
%    range that these new receivers are sensitive to.
%\item  Instrumental effects such as the element response pattern vary with frequency.
%\end{enumerate}
%%The following discussion describes existing methods designed to account for some of
%the above, and motivates the work that is the focus of this paper.

%    Existing algorithms designed to account for some of the above, can be shown to be
%    insufficient to achieve target image fidelity and dynamic range over the wide bands that 
%    most new instruments are being built with.

% and
% the spectral structure of the sky brightness distribution).

%Continuum emission from most astrophysical radio sources shows 
%significant spectral structure over the frequency ranges that these
%new receivers will instantaneously measure.
%To make a continuum image at the desired sensitivity, 
%it is essential to measure or reconstruct the spectral structure 
%the sky-brightness distribution 
%before constructing an image of the 
%integrated flux, and to do this while accounting for the frequency 
%dependence of the instrument (namely varying ranges of sampled 
%spatial scales, and varying antenna fields-of-view).
%
%While the main goal of wide-band imaging is to obtain a high dynamic-range
%continuum image, the reconstructed spectral structure can also be
%a useful astrophysical measurement. 

The simplest method of wide-band image reconstruction is to image 
each frequency channel separately and combine the results at the end. 
However, single-channel imaging is restricted to the narrow-band
$uv$-coverage and sensitivity of the instrument, and source 
%Also, the angular resolution of the telescope 
%changes with frequency, 
spectra can be studied only at the
angular resolution allowed by the lowest frequency in the sampled range.
For complicated extended emission, the single-frequency $uv$-coverage
may not be sufficient to produce a consistent solution across frequency.
While such imaging may suffice for many science goals, it does not
take full advantage of what an instantaneously wide-band instrument provides,
namely the sensitivity and spatial-frequency coverage obtained by 
combining measurements from multiple receiver frequencies.
%

Multi-Frequency-Synthesis (MFS) is the technique of combining measurements at multiple
discrete receiver frequencies during synthesis imaging.
MFS was initially done (REF) to increase the aperture-plane coverage
of sparse arrays by using narrow-band receivers and switching frequencies during the observations.
Wide bandwidth systems ( ~10\% ) later presented the problem of bandwidth smearing, which was
eliminated (REF) by splitting the wide band into narrow-band channels and mapping them onto
their correct spatial frequencies during imaging. It was assumed that
at the receiver sensitivities of the time, 
the measured sky brightness was constant across the observed bandwidth.
The next step was to consider a frequency-dependent sky brightness distribution.
\citep{MFCLEAN_CCW} describe a double-deconvolution algorithm based on
%and \citep{MFCLEAN} describe multi-frequency
%deconvolution algorithms based on 
the instrument's responses to a series of spectral basis functions, 
in particular, the first two terms of a Taylor series.
A map of the average spectral index is derived from the coefficient maps.
%
\citep{MFCLEAN} describe the MF-Clean algorithm which uses a formulation
similar to double-deconvolution but calculate Taylor-coefficients via a
least-squares solution. MF-Clean is implemented in the miriad software package.
The MF-Clean formulation ignores the process of resampling 
multi-frequency visibilities onto a single grid of spatial frequencies, and for 
arrays with many overlapping spatial-frequency tracks, errors are incurred.
More recently, \citet{LIKHACHEV} re-derive the least-squares method used
in MF-Clean to include more than two series coefficients.
\citep{SPATIOSPECTRALMEM} describe spatio-spectral 
MEM, an entropy based method in which single-channel imaging is done
along with a smoothness constraint applied across frequency.

%
%Both methods model an image as a collection of point 
%sources with linear spectra, solve for their intensities and slopes,
%and derive an average spectral-index for each source from the
%fitted slope. 
%The MF-Clean methods were developed for relatively narrow bandwidths ($<25\%$), 
%and can be shown \citep{EVLAMEMO101} to be insufficient to model 
%typical spectral structure across the large frequency ranges that 
%new wide-band receivers are sensitive to.
%

So far, these CLEAN-based multi-frequency deconvolution algorithms used 
point-source (zero-scale) flux components to model the sky emission.
% choice not well suited for extended emission.
We show in this paper that with the MF-Clean approach, 
deconvolution errors that occur with a point-source
model are enhanced in the spectral index image because of error propagation effects, and
that the use of a multi-scale technique can minimize this.
%
%In (REF1,REF1), XXX show that for high dynamic range imaging, point-source models
%are not well suited for extended emission, and describe several multi-scale
%techniques that are more capable at reconstructing complex extended emission accurately.
%Multiscale deconvolution techniques model images using flux
%components of varying scale size, and are more accurate at deconvolving large scale emission
%features. 
\citep{MSCLEAN} describe the MS-CLEAN algorithm which is a 
matched filtering technique using templates constructed from the instrument 
response to various large scale flux components. 
\citep{ERIC_MSCLEAN} describe a method simular to MS-Clean, and 
\citep{Asp_Clean} describe the ASP-CLEAN algorithm that explicitly fits 
for the parameters of Gaussian flux components and uses scale size to aid the separation of
signal from noise.

For high dynamic range imaging across wide fields-of-view,
direction-dependent instrumental effects that need to be accounted for are 
the w-term and the antenna power pattern, both of which are frequency-dependent. 
\citep{AWProjection}describe an algorithm for the correction of time-variable
wide-field instrumental effects for narrow-band interferometric imaging. 

In this paper, we describe MS-MFS (multi-scale multi-frequency synthesis) as
an algorithm that combines variants of the MF-Clean and MS-Clean
approaches to simultaneously reconstruct both spatial and spectral structure of the 
sky-brightness distribution. 
Frequency-dependent primary-beam correction is considered as a 
post-deconvolution correction step\footnote{The integration of 
direction-dependent correction algorithms such as AW-Projection
with MS-MFS will be discussed in a subsequent paper.}.
In section ZZ, we show imaging examples using simulations, multi-frequency VLA data
 and wideband EVLA data,  to illustrate the capabilities and limits of the
MS-MFS algorithm. 

%__________________________________________________________________

\subsection{Wide-Band Imaging}\label{Sec:WIDEBANDIMAGING}
%__________________________________________________________________


%
%The use of broad-band receivers in radio interferometery impacts standard continuum
%    synthesis imaging in three ways. 
%First, correlations measured at different frequencies sample the visibility function of the
%   sky brightness at different spatial frequencies. 
%Second, the sky brightness can change significantly across the large frequency 
%    range that these new receivers are sensitive to.
% Third, direction-dependent instrumental effects such as the element response pattern
%     also vary with frequency.
%This section describes existing methods designed to account for some of
%the above, and motivates the work that is the focus of this paper.
%


We begin with a discussion of how well we can reconstruct both spatial and 
spectral information from an incomplete set of visibility samples at
multiple observing frequencies.
% and describe how our choice of a flux model influences the 
%image reconstruction process when each observing frequency measures a different 
%set of spatial frequencies.
%Section \ref{Sec:MSMFS-model} then defines a multi-scale multi-frequency image model  
%and section \ref{Sec:MSMFS-eqns} describes a deconvolution algorithm that uses
%this flux model.

The spatial frequencies sampled at each observing frequency $\nu$ are between  
${u}_{min} = \frac{\nu}{c}{b}_{min} $ and ${u}_{max} = \frac{\nu}{c}{b}_{max}$, 
where ${u}$ is used here as a generic label for the $uv$-distance\footnote
{
The $uv$-distance is defined as $\sqrt{u^2+v^2}$ and is the 
radial distance of the spatial frequency measured by the baseline from the
 origin of the $uv$-plane, in units of wavelength $\lambda$.
} and 
$b$ represents the length of the baseline vector (in units of meters) projected onto the plane
perpendicular to the direction of the source.
%The maximum spatial frequency measured at each frequency defines the angular 
%resolution of the instrument ($\theta_{\nu} = 1/u_{max}(\nu)$).
%
The range of spatial frequencies between ${u}_{min}$ at $\nu_{max}$ and
${u}_{max}$ at $\nu_{min}$ represents the region where the visibility function
is sampled at all frequencies in the band, and there is sufficient information
to reconstruct both spatial and spectral structure.
The spatial frequencies outside this region are sampled only by a fraction of the 
band and the accuracy of a broad-band reconstruction
depends on how well the spectral and spatial structure are
constrained by an appropriate choice of a flux model. % and the rest of the measurements.

%\begin{enumerate}
%\item
For a flat-spectrum source, measurements at multiple frequencies sample the 
same sky brightnes distribution at different ranges of spatial scales, increasing 
the signal-to-noise of the measurements in regions of overlap, and providing 
better uv-coverage and angular resolution.
Therefore standard deconvolution algorithms applied to measurements combined 
via MFS will suffice to reconstruct source structure across the full range of 
spatial scales measured across the band.

%For sources with spectral structure, different frequencies measure different sky
% brightness distributions, and a direct combination of frequencies via MFS would 
%generate spurious spatial structure. The traditional method of imaging has been to
% make separate images at each frequency and and derive spectral structure after
% tapering the measurements from higher frequencies to match the angular 
%resolution of the lowest frequency. The imaging sensitivity and fidelity is however
%restricted to be that of a single channel and its uv-coverage, and in most cases,
%it is possible to do better.
%

%\noindent A few examples are used to illustrate the importance of an appropriate flux model.
%\item 
A compact, unresolved source with spectral structure is measured as a point
source at all frequencies, and $u_{max}$ at $\nu_{max}$ gives the maximum angular resolution at
which this source can be imaged. Since the visibility function of a point source is flat across
the entire spatial frequency plane, its spectrum is adequately sampled by the
multi-frequency measurements.
Using a flux model in which each source is a $\delta$-function with a smooth 
polynomial spectrum, 
it is possible to reconstruct the spectral structure of the source
at the maximum possible angular resolution.

%\item 
For resolved sources with spectral structure, the accuracy of the reconstruction
across all spatial scales between $u_{min}$ at $\nu_{min}$ and  $u_{max}$ at $\nu_{max}$
depends on an appropriate choice of flux model, and the constraints that it provides.
For example, a source emitting broad-band synchrotron radiation can be described by
a fixed brightness 
distribution at one frequency with a power-law spectrum associated with each location. 
Images can be made at the maximum angular resolution (given by $u_{max}$ at $\nu_{max}$)
with the 
assumption that different observing frequencies probe the same spatial 
structure but measure different amplitudes (usually a valid assumption). 
%Measurements at the highest frequencies constrain the angular size of the flux 
%components, and the combined spectra are constrained by the rest of the measurements.
This constraint is strong enough to correctly reconstruct even moderately resolved sources
that are completely unresolved at the low end of the band but resolved at the higher end.
On the other hand, a source whose structure itself changes by 100\% in amplitude 
across the band would break the above assumption (band-limited signals).
%that structure seen at the highest resolution is what is present all across the band.
One example is with multi-frequency observations of solar magnetic loops
where the different frequencies probe different layers in the upper chromosphere
and can have very different structures. 
In this case, a complete reconstruction would be possible only in the region of
overlapping spatial frequencies (between $u_{min}$ at $\nu_{min}$ and  $u_{max}$ at $\nu_{max}$),
unless the flux model includes constraints that bias the solution towards one
appropriate for such sources.
%The imaging would still however operate at the continuum sensitivity level, and is therefore
%better than single-channel imaging.

%\item 
The lower end of the spatial frequency range presents a different problem. 
The size of the central hole in the $uv$-coverage
increases with frequency. Spectra are not measured adequately for
%extended 
emission whose visibility function is non-zero only %in
%the range 
below $u_{min}$ at $\nu_{max}$ and a flat-spectrum large-scale source can
be indistinguishable from a relatively smaller source with a steep spectrum.
%For scale sizes in this range,
Additional constraints in the form 
of total-flux values for each frequency may be required for an accurate reconstruction.
% or spectrum is simple enough that half the band is enough to get it.
%\end{enumerate}

Finally, for wide-field imaging, the frequency-dependence of the primary-beam
can introduce artificial spectral effects that result in a 100\% variation of the
flux across the band. To recover both spatial and spectral structure of
the sky brightness across a large field of view, the frequency dependence of
the primary beam must be modeled and removed before or 
during multi-frequency synthesis imaging.  
%In addition to the effects of $uv$-coverage and source structure, various 
%frequency depdendent instrumental effects also need to be accounted for. 
%The angular size of the primary beam of the antenna decreases with
%an increase in observing frequency (see Fig.\ref{WBPB1}, section \ref{Sec:FreqDep}).
%Sources away from the pointing center of the beam
%are attenuated by different amounts across the 
%frequency band and this introduces artificial spectral structure into the
%measurements. To recover both spatial and spectral structure of
%the sky brightness across a large field of view, the frequency dependence of
%the primary beam must be modeled and removed during multi-frequency synthesis
%imaging.  

%In addition to the sky flux being frequency dependent, the instrumental effects are also 
%frequency dependent.
%The HPBW of the primary beam of each array element scales as $HPBW = \frac{c}{\nu D}$%. 
%The reconstructed image will therefore have artificial spatial and spectral structure that can% be
%measured and removed to undo these effects.


To summarize, just as standard interferometric image reconstruction uses 
{\it a priori} information about
the spatial structure of the sky to estimate the
visibility function in unmeasured regions of the $uv$-plane,
multi-frequency image reconstruction
algorithms need to use {\it a priori} information about the spectral
structure of the sky brightness. 
By combining such models 
with the known frequency-dependence of the 
spatial-frequency coverage and element response function, 
it is possible to reconstruct the broad-band sky brightness distribution % across wide fields of view
from incomplete spectral and spatial-frequency sampling.

%\subsection{Outline of this paper}

%Section \ref{Sec:WIDEBANDIMAGING} discusses the information contained in
%multi-frequency measurements from the point of view of a combined spatial and
%spectral reconstruction. 
%Section \ref{Sec:OVERVIEW} summarizes various existing algorithms that can be
%used for wide-band imaging, and points out what improvements are required.
%Section \ref{Sec:ALGORITHM} describes the formulation and implementation of the
%MS-MFS algorithm (multi-scale multi-frequency synthesis) as 
%a linear least-squares technique combined with conventional deconvolution. 
%This is an extension and combination of two existing methods with a few 
%differences that affect the accuracy and stability of the results. 
%Section \ref{Sec:IMAGINGRESULTS} shows some imaging results that demonstrate the capabilities
% and limits of the MS-MFS algorithm, and discusses various sources of error
%inherent in the reconstruction process. 
%Section \ref{Sec:CONCLUSIONS} summarizes these results.
%





\section{Multi-scale Multi-frequency deconvolution}\label{Sec:ALGORITHM}

%The design of an image reconstruction algorithm begins with an appropriate choice of
%parameterization of the sky brightness distribution and various instrumental effects.
%This parameterization is then folded into the transfer function of the instrument
%whose measurement process we are trying to invert. An optimization process is then
%followed to obtain best-fit estimates for the parameters which can then be 
%interpreted physically. 
%
The MS-MFS algorithm described here is based on the iterative image-reconstruction
framework\footnote{A 
steepest-descent chi-square minimization is done by iterating between two steps.
A major cycle computes the RHS of the normal equations, and the
minor cycle performs an approximate inverse of the Hessian matrix to generate
an update direction (image model components). }
described in \citep{RAU_IEEE_2009}. 
Sections xxx to yyy formulate the algorithm used in the minor-cycle of iterative
deconvolution for MS-MFS. 
Differences between the multi-scale and multi-frequency parts of MS-MFS with
the original MS-Clean and MF-Clean approaches are highlighted in sections XX and YY.
The implementation of this algorithm in the CASA package is summarized in
section ZZ.

\subsection{Parameterization of spatial structure}
For a multi-scale model, an image can be written as a linear combination of images
at different angular resolutions (Ref MS-CLEAN).
A multiscale representation of an image is given by 
\begin{equation}
\vec{I}^{model} = \sum_{s=0}^{N_s-1}  \vec{I}^{shp}_{s} \star \vec{I}^{sky,\delta}_s
\label{Eq:ms_model}
\end{equation}
where $N_s$ is the number of discrete spatial scales used to represent the image, and
$\vec{I}^{sky,\delta}_{s}$ represents a collection of $\delta$-functions that describe the locations
and integrated amplitudes of flux components of scale $s$ in the image.
$\vec{I}^{shp}_s$ is a tapered truncated parabola of width proportional to $s$.
The symbol $\star$ denotes convolution. 
%$\T_s=[\F]\vec{I}^{shp}_{s}$ is a 
%spatial-frequency taper equivalent to the convolution kernel $\vec{I}^{shp}_{s}$.
In order to always allow for the modeling of unresolved sources, 
we choose the first scale function $\vec{I}^{shp}_{s=0}$ to be a $\delta$-function.
%Successive basis functions then correspond to inverted parabolas of 
%larger widths (as $s$ increases). 

\subsection{Parameterization of spectral structure}\label{Sec:freqmodel}
The spectrum of each flux component is modeled by a polynomial in frequency
( a Taylor series expansion about $\nu_0$ ).
\begin{eqnarray}
%\label{Eq:mf_model}
\label{Eq:mf_model}
\vec{I}^{model}_{\nu} = \sum_{t=0}^{N_t-1} \wnt \vec{I}^{sky}_{t} ~~~\mathrm{where}~~~ \wnt&=&\dnuno^t 
%\label{Eq:tfunc_2}
%~~~~~\mathrm{or}~~~~ \wnt&=&\left[\log\nuno\right]^t
\end{eqnarray}
where $N_t$ is the order of the Taylor series expansion, and 
the $I^m_t$ represent multi-scale Taylor coefficient images (moment maps ??? ).
This decomposition is linear in the coefficients as well as in the basis functions.

These Taylor coefficients can be interpreted by choosing an astro-physically appropriate
spectral model and performing a Taylor expansion to derive expressions that each coefficient
maps to.
One choice for a spectral model is a power law with a varying index, represented by a 
second-order polynomial in $log(I)~vs~log\nuno$ space.
The variation of the spectral index with frequency is described by an average spectral
index $\alpha$ and a curvature term $\beta$.
\begin{equation}
I_{\nu}^{sky} = I_{\nu_0}^{sky} \nuno^{\alpha + \beta log \nuno}
\label{EQN_POWERLAW1}
\end{equation}
The main reason behind this choice of interpretation is the fact that continuum synchrotron emission is 
usually modeled (and observed) as a power law distribution with frequency. Across the wide
frequency ranges that new receivers are now sensitive to, 
spectral breaks, steepening and turnovers need to be factored into
models, and the simplest way to include them and ensure smoothness, is spectral curvature.
(FIGURE).
%% Figure Caption
%% (1) Synchrotron spectrum with steepening, turnover, break.
(Wideband imaging algorithms describes in CCW/SW use a fixed spectral index across the band,
and handle slight curvature by performing multiple rounds of imaging after removing the 
dominant/average $\alpha$ at each stage. 
They also suggest using higher order polynomials to handle spectral curvature.)

A Taylor expansion of Eqn \ref{EQN_POWERLAW1} yields the following expressions for the first
three coefficients ($t=0,1,2$ in Eqn \ref{EQN_TS}) from which the spectral index $\alpha$ and curvature $\beta$ images 
can be computed algebraically.
\begin{equation}
I^m_0 = I^{sky}_{\nu_0} ~~;~~~ I^m_1 = \alpha I^{sky}_{\nu_0} ~~;~~~ I^m_2 = \left(\frac{\alpha(\alpha-1)}{2} + \beta\right) I^{sky}_{\nu_0}
\label{EQN_COEFFS}
\end{equation}
Note that with this choice of parameterization, 
we are using a polynomial to model an exponential, and this has caveats, etc which
are discussed in the section on errors.
Also, note that there can be other interpretations of the Taylor coefficients %poln ?
and other expansions. 
A power-series expansion about $\alpha$ and $\beta$ will yield a logarithmic
expansion i.e. I vs log nu. (CHECK if/when this is better)
It is however impractical to work directly in Log I and Log nu space because that
involves taking logs and exp of image pixel amplitudes and this is highly unstable numerically.

Across wide fields of view, an approximate correction of the primary-beam and its
frequency dependence can also be folded into this formulation. 
We can write 
\begin{equation}
I_{\nu}^{sky} = P_{\nu}  I_{\nu}^{true} =  P_{\nu_0} I_{\nu_0}^{true} \nuno^{[\alpha_{true}+\alpha_{PB}] + [\beta_{true} + \beta_{PB} ] log \nuno}
\label{EQN_POWERLAW2}
\end{equation}
where $P_{\nu_0}$ is the primary beam at the reference frequency, and
$\alpha_{PB}$ and $\beta_{PB}$ are spectral index and curvature of the frequency
dependence of the primary beam.   An image reconstructed using such a 
formulation can be corrected in a post-deconvolution step, if the primary-beam and
its frequency dependence is known {\it a priori}.  This however corrects only for
an average primary-beam and its frequency dependence, and a more accurate
solution requires the AW-Projection algorithm. 
%%%%% Implementation

\subsection{Multi-scale multi-frequency model}
For multi-scale and multi-frequency deconvolution, the sky-brightness distribution
can be parameterized in a multi-scale basis (\ref{Eq:ms_model}), with the amplitudes
of each component described by a polynomial across frequency (\ref{Eq:mf_model}).  
A region of emission in which the spectrum varies with position
will then be modeled as a sum of wide-band flux components.
The image flux model at each frequency can be written as a linear sum of  
coefficient images at different spatial scales. 
\begin{equation}
\vec{I}^{model}_{\nu} = \sum_{t=0}^{N_t} \sum_{s=0}^{N_s} \wnt \left[ \vec{I}^{shp}_s \star \vec{I}^{sky}_{s\atop t}\right] ~~~~\mathrm{where}~~~\wnt = \dnuno^t 
\label{Eq:msmf_model}
\end{equation}
Here, $N_s$ is the number of discrete spatial scales used to represent the image and  
$N_t$ is the order of the series expansion of the spectrum. 
$\vec{I}^{sky}_{s\atop t}$ represents a collection of $\delta$-functions that describe the locations
and integrated amplitudes of flux components of scale $s$ in the image of the $t^{th}$ series 
coefficient. 

\subsection{Measurement equations}\label{Sec:meqn}
The {Measurement Equations\footnote
{
The measurement equation of an imaging instrument describes its transfer function (the effect of
the measurement process on the input signal).
For an ideal interferometer (a perfect spatial frequency filter, with no instrumental gains),
it can be written in matrix notation as follows. 
Let $I^{sky}_{m\times 1}$ be a pixelated image of the sky and let $V^{obs}_{n\times 1}$
be a vector of $n$ visibilities. 
Let $S_{n\times m}$ be a projection operator that describes
the uv-coverage as a mapping of $m$ discrete spatial 
frequencies (pixels on a grid) to $n$ visibility samples (usually $n>m$).
Let $F_{m\times m}$ be the Fourier transform operator.
Then, $[{\Sa}_{n\times m}] [F_{m\times m}] \vec{I}^{sky}_{m\times 1} = \vec{V}^{obs}_{n\times 1} $
}
}for a sky brightness distribution parameterized by Eqn.\ref{Eq:msmf_model} are

\begin{eqnarray}
\vec{V}^{obs}_{\nu} &=& [\Sna][\F]\vec{I}^{model}_{\nu}
= \sum_{t=0}^{N_t} \sum_{s=0}^{N_s} \wnt [\Sna][T_s][\F] \vec{I}^{sky}_{s\atop t}
\label{Eq:msmfs_meqn}
\end{eqnarray}
where $w_{\nu}$ are Taylor-weights, $[\Sna]$ represents the 
spatial-frequency sampling function for frequency $\nu$, and 
the image-domain convolution with $\vec{I}^{shp}_s$ is written as a spatial-frequency
taper function $[T_s]_{m\times m} = diag([\F] \vec{I}^{shp}_s)$.

Combining measurements from all frequencies together, we can write
\begin{eqnarray}
\vec{V}^{obs} &=& \sum_{t=0}^{N_t} \sum_{s=0}^{N_s} [\Wnt][\Sa][T_s] [\F] \vec{I}^{sky}_{s\atop t}
\label{Eq:msmfs_meqn}
\end{eqnarray}
where $[\Wnt]$ is a diagonal $nN_c\times nN_c$ matrix of weights, comprised of $N_c$ blocks each
of size $n\times n$ for each frequency channel ($\nu$), containing $\wnt$. 
The multi-frequency $uv$-coverage of the synthesis array is represented by $[\Sa_{nN_c\times m}]$.

The full measurement matrix therefore has the shape
$nN_c\times mN_sN_t$, which when multiplied by the set of $N_sN_t$ model sky vectors each
of shape $m\times 1$, produces $nN_c$ visibilities.

\noindent For $N_t=3, N_s=2$ the measurement equations can be written as follows, in
block matrix form. The subscript $p$ denotes the $p^{th}$ spatial scale
and the subscript $q$ denotes the $q^{th}$ Taylor coefficient of the spectrum
polynomial.

{
\begin{equation}
\begin{array}{l}
\left[\begin{array}{llllll} 
\noalign{\medskip}
\noalign{\medskip}
\noalign{\medskip}
\noalign{\medskip}
%   \left[\W^{mfs}_0 \Sa T_0 \F\right] & \left[\W^{mfs}_1 \Sa T_0 \F\right] & \left[\W^{mfs}_2 \Sa T_0 \F\right] & 
%   \left[\W^{mfs}_0 \Sa T_1 \F\right] & \left[\W^{mfs}_1 \Sa T_1 \F\right] & \left[\W^{mfs}_2 \Sa T_1 \F\right] \\  
   \left[A_{{0}\atop{0}}\right] & \left[ A_{{0}\atop{1}} \right] & \left[A_{{0}\atop{2}}\right] & 
   \left[A_{{1}\atop{0}}\right] & \left[A_{{1}\atop{1}}\right] & \left[A_{{1}\atop{2}}\right] \\  
\noalign{\medskip}
\noalign{\medskip}
\noalign{\medskip}
\noalign{\medskip}
   \end{array} \right] \\
\noalign{\medskip}
\noalign{\medskip}
\noalign{\medskip}
\noalign{\medskip}
\noalign{\medskip}
{~\mathrm{where}~~\left[ A_{{p}\atop{q}}\right] = [\W^{mfs}_q][\Sa][T_p][\F]} \\
\noalign{\medskip}
{~\mathrm{for}~p\in\{0,N_s-1\}~\mathrm{and}~q\in\{0,N_t-1\}    }  
\end{array}
\left[\begin{array}{l} 
\noalign{\medskip}
                       \vec{I}^{sky}_{{0}\atop{0}} \\ 
\noalign{\medskip}
                       \vec{I}^{sky}_{{0}\atop{1}} \\ 
\noalign{\medskip}
                       \vec{I}^{sky}_{{0}\atop{2}} \\ 
\noalign{\medskip}
                       \vec{I}^{sky}_{{1}\atop{0}} \\ 
\noalign{\medskip}
                       \vec{I}^{sky}_{{1}\atop{1}} \\ 
\noalign{\medskip}
                       \vec{I}^{sky}_{{1}\atop{2}}\\
\noalign{\medskip}
		       \end{array}\right] =\vec{V}^{obs}%_{nN_c\times 1}
\label{meqn_msmfs_math}
\end{equation}
}



\subsection{Normal equations}\label{Sec:neqn}

The {Normal Equations\footnote
{\label{FN:Neqn}
The Normal Equations are the linear system of equations whose solution 
gives a weighted least-squares estimate of a set of parameters in a model  
($\chi^2$ minimization). 
For an ideal interferometer, it is given by 
$[\Fd \Sd W \Sa \F ] I^{sky}_{m\times 1} = [\Fd \Sd W] V^{obs}_{n\times 1} = I^{dirty}_{m\times 1}$
where $W_{n\times n}$ is a diagonal matrix of signal-to-noise based measurement weights
and $\Sd$ denotes the mapping of measured visibilities onto a spatial frequency grid.
The Hessian (matrix on the LHS) is by construction a circulant { convolution} operator with 
a shifted version of $I^{psf}_{m\times 1} = diag[\Fd \Sd W \Sa]$ in each row. 
The {dirty} image on the RHS (produced by direct Fourier inversion of weighted visibilities)
is therefore the convolution of $I^{sky}_{m\times 1}$ with $I^{psf}$,
and these equations can be solved by a {deconvolution}. 
}
}
can be written in block matrix form, with each block-row 
(for scale size $s$, and Taylor term $t$) is given by
\begin{eqnarray}
\label{Eq:msmfs_neqn_1}
\sum_{p=0}^{N_s-1}\sum_{q=0}^{N_t-1} \left[\He_{{s,p}\atop{t,q}}\right] \vec{I}^{sky}_{{p}\atop{q}} &=& \vec{I}^{dirty}_{{s}\atop{t}}~~~  \forall~ s \in [0,N_s-1], t\in[0,N_t-1]
\end{eqnarray}
where each $\left[\He_{{s,p}\atop{t,q}} \right]$ is an $m\times m$ block of the 
Hessian matrix, and $\vec{I}^{dirty}_{{s}\atop{t}}$ is one of $N_s N_t$ dirty images.
\begin{eqnarray}
\left[\He_{{s,p}\atop{t,q}} \right] &=& \left[A_{{s}\atop{t}}^{\dag}\right][\Wim] \left[A_{{p}\atop{q}}\right] \\
 &=&    [\Fd T_s  \Sd \W^{mfs}_t]  [\Wim^{mfs}]  [\W^{mfs}_q \Sa T_p \F] \\
 &=&    [\Fd T_s \F] [\Fd  \Sd \W^{mfs}_t  \Wim^{mfs}  \W^{mfs}_q \Sa  \F] [\Fd T_p \F]\\
% &=&    [\Fd T_s \F] [\He^{mfs}_{t,q}] [\Fd T_p \F]
 &=& [\Fd T_s \F] \left\{  \sum_{\nu} \wntq [\Fd\Snd\Wimn\Sna\F] \right\} [\Fd T_p \F] \\
 &=& [\Fd T_s \F] \left\{  \sum_{\nu} \wntq [\He_{\nu}\} \right] [\Fd T_p \F]
\end{eqnarray}
%\begin{eqnarray}
% [\He^{mfs}_{t,q}] &=& [\Fd  \Sd \W^{mfs}_t  \Wim^{mfs}  \W^{mfs}_q \Sa  \F]\\
%   &=& \sum_{\nu} \wntq [\Fd\Snd\Wimn\Sna\F] \\
% &=& \sum_{\nu} \wntq [\He_{\nu}]
%\end{eqnarray}
%\begin{eqnarray}
%\label{Eq:Hessian}
% [\He_{\nu}] &=&  [\Fd\Snd\Wimn\Sna\F]
%\end{eqnarray}
$[\He_{\nu}] =  [\Fd\Snd\Wimn\Sna\F]$ is the Hessian matrix formed using only one
 frequency channel, and
is a convolution operator containing a shifted version of the single-frequency 
PSF $\I^{psf}_{\nu} = diag[\Fd \Sd W \Sa]$ in each row (see footnote \vref{FN:Neqn}).
$[\Fd \T_s\F]$ is also a convolution operators with $I^{shp}_s$ as 
the kernel. 
The process of convolution is associative and commutative, and 
therefore, $\left[\He_{{s,p}\atop{t,q}}\right]$  is also a convolution operator 
whose kernel is given by 
\begin{equation}
\label{Eq:msmfs_neqn_2.5}
\vec{I}^{psf}_{{s,p}\atop{t,q}} = I^{shp}_s  \star \left\{ \sum_{\nu} \wntq \vec{I}^{psf}_{\nu} \right\} \star I^{shp}_p 
%&=& I^{shp}_s \star I^{psf}_{t,q} \star I^{shp}_p
\end{equation}
Therefore, to compute the Hessian matrix, it suffices to compute one such kernel
per Hessian block.
%%%%%%%%%%%%%%%%%%%%%%%%%%%%%%%%%%%%%


The dirty images on the RHS of Eqn.\ref{Eq:msmfs_neqn_1} can be written as follows.
\begin{eqnarray}
\label{Eq:msmfs_neqn_3}
\vec{I}^{dirty}_{{s}\atop{t}}  &=& [\Fd T_s \F][\Fd\Sd\Wntd\Wim] \vec{V}^{obs} \\
&=& [\Fd T_s \F] \left\{  \sum_{\nu} \wnt [\Fd\Snd\Wimn] \vec{V}_{\nu}^{obs}  \right\} \\
&=& I^{shp}_s \star \left\{\sum_{\nu} \wnt \vec{I}^{dirty}_{\nu}  \right\} 
%&=& [\Fd T_s \F] \vec{I}^{dirty}_t = I^{shp}_s \star \vec{I}^{dirty}_t\\
\label{Eq:mf_neqn_3a}
%\vec{I}^{dirty}_{t}  &=& [\Fd\Sd\Wntd\Wim] \vec{V}^{corr}\\
% &=& \sum_{\nu} \wnt [\Fd\Snd\Wimn] \vec{V}_{\nu}^{corr} = \sum_{\nu} \wnt \vec{I}^{dirty}_{\nu}\\
%\label{Eq:specPSF}
%\vec{I}^{psf}_{t,q}&=& \sum_{\nu} \wntq [\Fd\Snd\Wimn] \vec{1} = \sum_{\nu} \wntq \vec{I}^{psf}_{\nu}
\end{eqnarray}
where ${\I}_{\nu}^{dirty} = [\Fd \Snd \W_{\nu}] \vec{V}^{obs}_{\nu} $ is the dirty
image formed by direct Fourier inversion of weighted visibilities from one frequency channel.


When all scales and Taylor terms are combined, 
the full Hessian matrix contains
$N_t N_s \times N_t N_s$ blocks each of size $m\times m$, 
and $N_t$ Taylor coefficient images each of size $m\times 1$, 
for all $N_s$ spatial scales.

The normal equations in block matrix form for the example 
in Eqn.\ref{meqn_msmfs_math} for $N_t=3, N_s=2$ is shown
in Eqn.\ref{Eq:msmfs_neqn_matrix}.
The Hessian matrix consists of $N_s\times N_s=2\times 2$ blocks 
(the four quandrants of the matrix), each for one pair of spatial scale $s,p$.
Within each quadrant, the $N_t\times N_t = 3\times 3$
matrices correspond to various pairs of $t,q$ (Taylor coefficient indices).
This layout shows how the multi-scale and multi-frequency aspects of
this imaging problem are combined and illustrates the 
dependencies between the spatial and spectral basis functions.
%Note that the $3\times3$ block in the top left quadrant corresponds to the
%entire Hessian matrix in Eqn.~\ref{Eq:mfs_neqn3} (since $\I^{shp}_0$ is a
%$\delta$-function).


\begin{equation}\small
\left[\begin{array}{llllll} 
\noalign{\medskip}
   \left[H_{{ 0, 0}\atop{ 0, 0}}\right] & \left[H_{{ 0, 0}\atop{ 0, 1}}\right] & \left[H_{{ 0, 0}\atop{ 0, 2}}\right] & \left[H_{{ 0, 1}\atop{ 0, 0}}\right] & \left[H_{{ 0, 1}\atop{ 0, 1}}\right] & \left[H_{{ 0, 1}\atop{ 0, 2}}\right] \\  
\noalign{\medskip}
   \left[H_{{ 0, 0}\atop{ 1, 0}} \right] & \left[H_{{ 0, 0}\atop{ 1, 1}}\right] & \left[H_{{ 0, 0}\atop{ 1, 2}}\right] & \left[H_{{ 0, 1}\atop{ 1, 0}}\right] & \left[H_{{ 0, 1}\atop{ 1, 1}}\right] & \left[H_{{ 0, 1}\atop{ 1, 2}}\right] \\  
\noalign{\medskip}
   \left[H_{{ 0, 0}\atop{ 2, 0}} \right] & \left[H_{{ 0, 0}\atop{ 2, 1}}\right] & \left[H_{{ 0, 0}\atop{ 2, 2}}\right] & \left[H_{{ 0, 1}\atop{ 2, 0}}\right] & \left[H_{{ 0, 1}\atop{ 2, 1}}\right] & \left[H_{{ 0, 1}\atop{ 2, 2}}\right] \\  
\noalign{\medskip}
%   \hline
\noalign{\medskip}
   \left[H_{{ 1, 0}\atop{ 0, 0}} \right] & \left[H_{{ 1, 0}\atop{ 0, 1}}\right] & \left[H_{{ 1, 0}\atop{ 0, 2}}\right] & \left[H_{{ 1, 1}\atop{ 0, 0}}\right] & \left[H_{{ 1, 1}\atop{ 0, 1}}\right] & \left[H_{{ 1, 1}\atop{ 0, 2}}\right] \\  
\noalign{\medskip}
   \left[H_{{ 1, 0}\atop{ 1, 0}} \right] & \left[H_{{ 1, 0}\atop{ 1, 1}}\right] & \left[H_{{ 1, 0}\atop{ 1, 2}}\right] & \left[H_{{ 1, 1}\atop{ 1, 0}}\right] & \left[H_{{ 1, 1}\atop{ 1, 1}}\right] & \left[H_{{ 1, 1}\atop{ 1, 2}}\right] \\  
\noalign{\medskip}
   \left[H_{{ 1, 0}\atop{ 2, 0}} \right] & \left[H_{{ 1, 0}\atop{ 2, 1}}\right] & \left[H_{{ 1, 0}\atop{ 2, 2}}\right] & \left[H_{{ 1, 1}\atop{ 2, 0}}\right] & \left[H_{{ 1, 1}\atop{ 2, 1}}\right] & \left[H_{{ 1, 1}\atop{ 2, 2}}\right] \\  
\noalign{\medskip}
   \end{array} \right]
\left[\begin{array}{l} 
\noalign{\medskip}
		       \vec{I}^{sky}_{{ 0}\atop{ 0}} \\ 
\noalign{\medskip}
                       \vec{I}^{sky}_{{ 0}\atop{ 1}}\\ 
\noalign{\medskip}
		       \vec{I}^{sky}_{{ 0}\atop{ 2}}\\ 
\noalign{\medskip}
%\hline
\noalign{\medskip}
		       \vec{I}^{sky}_{{ 1}\atop{ 0}} \\ 
\noalign{\medskip}
		       \vec{I}^{sky}_{{ 1}\atop{ 1}}\\ 
\noalign{\medskip}
		       \vec{I}^{sky}_{{ 1}\atop{ 2}}\\
\noalign{\medskip}
			       \end{array}\right] =
\left[\begin{array}{l} 
\noalign{\medskip}
                       \vec{I}^{dirty}_{{ 0}\atop{ 0}}  \\ 
\noalign{\medskip}
                       \vec{I}^{dirty}_{{ 0}\atop{ 1}} \\ 
\noalign{\medskip}
		       \vec{I}^{dirty}_{{ 0}\atop{ 2}}\\ 
\noalign{\medskip}
%\hline
\noalign{\medskip}
		       \vec{I}^{dirty}_{{ 1}\atop{ 0}} \\ 
\noalign{\medskip}
		       \vec{I}^{dirty}_{{ 1}\atop{ 1}} \\ 
\noalign{\medskip}
		       \vec{I}^{dirty}_{{ 1}\atop{ 2}} \\
\noalign{\medskip}
			       \end{array}\right] 
\label{Eq:msmfs_neqn_matrix}
\end{equation}

This is the system of equations to be solved to obtain estimates of the model parameters $I^{sky}_{p\atop q}$.
The spatial-frequency sampling  of a real interfetometer is always incomplete 
($[\Sa]$ is rank-deficient).
Therefore, each Hessian block, and the entire Hessian matrix is singular, an exact inverse
does not exist. An accurate reconstruction can be obtained only via successive approximation
(iterative numerical optimization).
%An approximate solution of the normal equations is performed in each minor cycle
%to choose flux components,  

\subsection{Principal Solution}
An approximate solution of the normal equations can be computed via diagonal
approximations of all Hessian blocks.
%Even in the ideal case of complete sampling where it may not be singular, 
%its dimensions make a brute-force inversion and application intractable.
Each Hessian block is a convolution operator with a shifted version of 
$I^{psf}_{{s,p}\atop{t,q}}$ in each row, and 
elements on the diagonal within each Hessian block are the same. 
The Hessian matrix with a diagonal approximation for each block $\left[\He_{{s,p}\atop{t,q}}\right]$ 
can be written as one $N_tN_s \times N_tN_s$ element matrix ($[H^{peak}]$), 
that applied to all pixels independently.  An approximate solution of the normal equations
can be obtained by inverting $[H^{peak}]$ once, and applying it to all pixels of the dirty
images, one pixel at a time.

This process of doing a pixel-by-pixel inversion of the diagonal-approximate 
Hessian matrix gives the
principal solution\footnote{The principal solution (as defined in \citet{BRACEWELL_ROBERTS} and used in 
\citet{DECONV_LECTURE}) is a term specific to radio interferometry
and represents the dirty image normalized by the sum of weights.  
It is the image formed purely from the measured data, with no contribution from the
invisible distribution of images (unmeasured spatial frequencies).
It is also an approximate solution of the normal equations, calculated using a diagonal
approximation of the Hessian (elements on the diagonal are the sum of weights).
For isolated sources, the values measured at the peaks of the
principal solution images are the true sky values as represented in the 
image model.
} of the system. 
%For the simplest case of $N_s=1, N_r=1$, the principal solution is obtained
%by normalizing the Fourier-inverted weighted visibilities by the sum of weights (values on the
%diagonal of the Hessian matrix).
%
%
%
%The principal solution is found by using a diagonal approximation of each
Such a solution will be correct only at the locations of the centers
of flux components (source peaks), and must be augmented with an iterative optimization 
approach to ensure accuracy.  In the case of perfect sampling (where the Hessian blocks
are truly diagonal), the principal solution will directly give images
of series-coefficients.

%When the principal solution is to be used within an iterative joint deconvolution,
%a few simplifying assumptions may be needed to trim computational costs.
%For a source with complicated spatial structure the number of distinct spatial
%scale basis functions is typically $N_s \approx 10$, and for power-law spectra with
%indices around -1.0, $N_t =4$ or $N_t=5$ terms in the series are required to
%accurately model the power law with a polynomial (across a 2:1 bandwidth). 
%Therefore typically, $N_sN_t \approx 50$. Although the inversion of $[H^{peak}]$ 
%may be tractable, the computational cost of a $50\times 50$ matrix multiplication
%applied per pixel to a set of $10^6$ pixels over  a large number of
%iterations may be prohibitive in comparison to the numerical accuracy that this
%exact inversion provides. 
%

\subsubsection{Properties of $[H^{peak}]$}

\begin{enumerate}
\item The elements on the diagonal of $[H^{peak}]$ are a measure of the instrument's sensitivity 
to a flux component of unit total flux whose shape and spectrum is given by each of the $N_sN_t$
possible pairs of spatial and spectral basis functions.
\begin{eqnarray}
H^{peak}_{{s,p}\atop{t,q}} &=& mid\left\{\vec{I}^{psf}_{{s,p}\atop{t,q}}\right\} = tr\left[\sum_{\nu} \wntq [T_s\Snd\Wimn\Sna T_p] \right] \\
	& &\forall~ s,p ~\in ~\{0 ... N_s-1\}~~,~~t,q ~\in ~\{0 ... N_t-1\}\nonumber
\label{Eq:hpeak_msmfs}
\end{eqnarray}
\item The off-diagonal elements measure the orthogonality\footnote
{\label{FN:ortho}The following definition of orthogonality is used here. Two vectors are orthogonal
if their inner product is zero. The orthogonality of a pair of scale functions 
is measured by the integral of the product of their $uv$-taper functions.
To account for $uv$-coverage, this integral is weighted by the sampling function
(see Eqn.~\ref{Eq:hpeak}).
}
 between the various
basis functions, for the given $uv$-coverage and weighting scheme.
They measure the amount of overlap between basis
functions in the measurement domain. 
Smaller values indicate a more orthogonal set of basis functions,
and the instrument is better able to distinguish between the chosen spatial scales.

\item The condition number of this matrix (or of blocks within this matrix)
will indicate if the chosen set of basis functions and spatial-frequency coverage 
provide enough constraints to provide a stable solution, and 
can be used as a metric to choose a suitable basis set.
For a simple example, if a 3-term solution is attempted with data from only two distinct
frequencies, $[H^{peak}]$ will be singular.
Or, for some choice of multi-frequency $uv$-coverage, the visibilities
measured by the instrument for two different spatial scales may become hard to distinguish. 
Then, the cross-term element of $[H^{peak}]$ corresponding to
this combination could have a higher value, indicating that the two parameters are highly coupled,
and there is insufficient information in the data and sampling pattern to distinguish 
between the scales. A similar situation can arise to create ambiguity between 
spatial or spectral structure (an extreme example is multi-frequency measurements from only
one baseline.
%Figure XXX shows a 1D example of
%such a situation, where it can be difficult to 
%distinguish between an extended source with a flat spectrum and a compact source 
%whose flux falls with increasing frequency (steep spectrum).
%\item The convolution kernels from the first row of Hessian blocks
% ($s=0,t=0, p\in\{0,N_s-1\}, q\in\{0,N_t-1\}$) represent the
%instrument's response functions to a flux component of unit total flux whose shape is
%given by the $p^{th}$ scale basis function and whose spectrum is given
%by the $q^{th}$ Taylor function. 
\item In general, $[H^{peak}]$ will be a positive-definite 
symmetric matrix whose inverse can be easily computed {\it via} a 
Cholesky decomposition\footnote
{A Cholesky decomposition is a decomposition of a symmetric positive-definite matrix
into the product of a lower triangular matrix and its conjugate transpose. 
It is used in the solution of system of equations $[A]\vec{x}=\vec{b}$ where
$[A]$ is symmetric positive-definite. 
The normal equations of a linear least-squares problem are usually in this form. 
In our case, this linear least-squares problem corresponds to the 
representation of the sky brightness as a linear combination of basis functions
\citep{NR}.
%[$http://en.wikipedia.org/wiki/Cholesky\_decomposition$].
}.  
Also, the value of $N_s$ is usually $< 10 $, making the inversion 
of $[H^{peak}]$ tractable. 

\item 
Several approximations can be made about the structure
of $[H^{peak}]$ to simplify its inversion, and it is 
important to understand the numerical implications of these trade-offs.
%
One is a block-diagonal approximation of the full 
Hessian ({\it i.e.} using only those blocks of the Hessian in
Eqn.~\ref{Eq:msmfs_neqn_matrix} for which $s=p$; top-left
and bottom-right quadrants).
This approximation ignores the cross-terms between spatial scales and assumes that
the scale basis functions are orthogonal. 
This is never true for a set of tapered truncated paraboloids, but
this approximation works because of the iterative $\chi^2$-minimization
process.
Now, a multi-frequency principal solution 
can be done separately on each remaining $N_t\times N_t$ block,  
one spatial scale at a time.
%The Hessian blocks that form each $[H^{peak}_s]$ are outlined
% in Eqn.~\ref{Eq:msmfs_neqn_matrix}.
%Note that solving the multi-frequency $[H^{peak}]$ for each scale
This automatically does a normalization across scales that corresponds to a diagonal
approximation of the multi-scale Hessian 
(see section \ref{Sec:diff_msclean} for 
alternate ways of computing the multi-scale solution).
%This diagonal approximation across scales  is never accurate because 
%a set of tapered truncated paraboloids cannot form an orthogonal basis set.
%However, this approximation works because an iterative $\chi^2$-minimization
%process tolerates inaccurate steps during each iteration.   
The update  step of the iterative deconvolution still needs to evaluate
the full LHS of the normal equations while subtracting out a flux component

\end{enumerate}

\subsection{MS-MFS algorithm}\label{Sec:MSMFS}

This section describes an iterative joint deconvolution process that 
solves the normal equations (Eqn.\ref{Eq:msmfs_neqn_1}) and produces a
set of $N_t$ series-coefficient images at $N_s$ different spatial scales.
The algorithm presented in this section is listed with more details
in Algorithm \vref{ALGO:MSMFS_1} and Algorithm \vref{ALGO:MSMFS_2}) 


%\subsubsection{Pre-compute Hessian}
\noindent\paragraph{\bf Pre-compute Hessian : }
Convolution kernels for all distinct blocks in the $N_s N_t \times N_s N_t$ Hessian 
are evaluated {\it via} Eqn.~\ref{Eq:msmfs_neqn_2.5}. 
All kernels are normalized by $w_{sum}$ such that the peak of $\vec{I}^{psf}_{{0,0}\atop{0,0}}$
is unity, and the relative weights between Hessian blocks is preserved. 
%This is equivalent to defining the weight image $\vec{I}^{wt}$ as the diagonal of
%the $[H_{{00}\atop{00}}]$ Hessian block, and normalizing all the RHS vectors by it. 
A set of $N_s$ matrices each of shape $N_t\times N_t$ and denoted as $[H^{peak}_s]$
are constructed from the diagonal blocks of the full Hessian (blocks for which $s=p$ 
in Eqn.~\ref{Eq:msmfs_neqn_matrix}).
Their inverses are computed and stored in $[{H^{peak}_s}^{-1}]$.

%\subsubsection{Iterative deconvolution}

\noindent\paragraph{\bf Initialization :} 
All $N_sN_t$ model images are initialized to zero (or an {\it a priori} model).

\noindent\paragraph{\bf Major and minor cycles : }
The normal equations are solved iteratively by repeating steps 
\ref{stepRHS_msmfs} to \ref{stepPredict_msmfs} until some termination criterion is reached.
Steps \ref{stepRHS_msmfs} and \ref{stepPredict_msmfs} form one major cycle, 
and repetitions of Steps \ref{stepFind_msmfs} to \ref{stepUpdate_2_msmfs} form the minor cycle.
\begin{enumerate}
%\item {\bf Initialization :} 
%All $N_sN_t$ model images are initialized to zero (or an {\it a priori} model).
\item\label{stepRHS_msmfs} {\bf Compute residual images :} The RHS vectors 
(residual or dirty images) $\vec{I}^{dirty}_{{s}\atop{t}} ~\forall~t\in\{0,N_t$-$1\}$
of the normal equations are computed {\it via} Eqn.~\ref{Eq:msmfs_neqn_3} by first
computing the multi-frequency dirty images and then convolving them by the
scale basis functions.

\item\label{stepFind_msmfs} {\bf Find a Flux Component :}
The principal solution is computed for all pixels, one scale at a time.
\begin{equation}
I^{pix,psol}_s = [{H^{peak}_s}^{-1}] I^{pix,dirty}_s ~~~~~~~\mathrm{for~each~pixel,~and~scale} ~s
\label{Eq:msmfs_psol}
\end{equation}
Here,  $[H^{peak}_s]$ is the $s^{th}$ block (of size $N_t\times N_t$)
in the list of diagonal-blocks of $[H^{peak}]$, 
and $I^{pix,dirty}_s$ is the $N_t\times 1$ vector constructed 
from $\vec{I}^{dirty}_{{s}\atop{t}} ~\forall~t\in\{0,N_t$-$1\}$.

The principal solution consists of $N_s$ sets of 
$N_t$ Taylor-coefficient images.
For iteration $i$, the $N_t$ element solution set 
with the dominant $q=0$ component across all scales and pixel locations
is chosen the current flux component. Other heuristics may also be employed
to make this choice, for example, pick the set of components that makes the
largest impact on the value of $\chi^2$. 
Let the scale size of this chosen subset be $p$. 

The result of this step is a set of $N_t$ model images, each containing one 
$\delta$-function that marks the location of the center of a flux component
of shape $\I^{shp}_{p,(i)}$. The amplitudes of these $N_t$ $\delta$-functions
are the Taylor coefficients that model the spectrum of the total flux of 
this component. Let these model images be denoted as 
$\left\{\vec{I}^{model}_{{{p}\atop{q}},(i)}\right\};q\in[0,N_t]$.

\item\label{stepUpdate_msmfs} {\bf Update model images :}
A single multi-scale model image is accumulated for each Taylor coefficient.
\begin{equation}
\I^{model}_{q} = \I^{model}_{q} + g \left( \I^{model}_{{{p}\atop{q}},(i)} \star \I^{shp}_{p} \right) ~~~~~~~~ \forall q\in[0,N_t]
\label{Eq:msmfs_updatemodel}
\end{equation}
where $g$ is a loop-gain that takes on values between 0 and 1 and controls the 
step size for each iteration in the $\chi^2$-minimization process.

\item\label{stepUpdate_2_msmfs} {\bf Update RHS :}
The RHS residual images are updated by evaluating and subtracting out the entire
LHS of the normal equations. Since the chosen flux component corresponds
to just one scale, the evaluation of the LHS is a summation over only Taylor terms.
\begin{equation}
\I^{res}_{{s}\atop{t}} = \I^{res}_{{s}\atop{t}} - g\left( \sum_{q_i=0}^{N_t-1} \left[  \I^{psf}_{{s,p}\atop{t,q}}  \star \I^{model}_{{{p}\atop{q}},(i)} \right] \right)
\label{Eq:msmfs_updaterhs}
\end{equation}
{\bf Repeat from Step \ref{stepFind_msmfs}} until the minor-cycle flux limit is reached.

\item\label{stepPredict_msmfs} {\bf Predict }:
Model visibilities are computed from each Taylor-coefficient image, 
in the same way as in Eqn.~\ref{Eq:mfs_predict} for multi-frequency imaging.
Residual visibilities are computed as $\vec{V}^{res}_{\nu}= \vec{V}^{corr}_{\nu} - \vec{V}^{model}_{\nu}$.

{\bf Repeat from Step \ref{stepRHS_msmfs}} until a global convergence criterion is satisfied.
\end{enumerate}
\enlargethispage{\baselineskip}

\noindent\paragraph{\bf Restoration :}
After convergence, the model spectral coefficient images can be interpreted in
different ways. If required, the final image products can be smoothed
with the restoring beam and the residuals are added back in. 
%Some forms applicable for radio astronomy are described below as additional
%operations that need to be performed on the model images. 
\begin{enumerate}
\item The most obvious data products are the spectral-coefficient images themselves, 
which can be directly smoothed by the restoring beam. The residual images that are added
back in should be the principal solution computed from the final residuals, to
ensure that any undeconvolved flux has the correct flux values.
\item  For the study of broad-band radio emission, the spectral coefficients can be
interpreted in terms of a power law in frequency with varying index
(as described in Section \ref{Sec:freqmodel}).
The data products are images of the reference-frequency flux $\vec{I}^{sky}_{\nu_0}$,
the spectral-index $\vec{I}^{\alpha}$ and the spectral curvature $\vec{I}^{\beta}$.
%Section \ref{Sec:ErrPowerLaw} describes the errors involved in this calculation.
\begin{eqnarray}
\label{Eq:calcab_1}
\vec{I}^{sky}_{\nu_0} &=& \vec{I}^{model}_0 \\
\label{Eq:calcab_2}
\vec{I}^{\alpha} &=& {\vec{I}^{model}_1}/{\vec{I}^{model}_0}  \\
\label{Eq:calcab_3}
\vec{I}^{\beta} &=& \left[{\vec{I}^{model}_2}/{\vec{I}^{model}_0}\right] - \left[{{\vec{I}^{\alpha}(\vec{I}^{\alpha}-1)}}{/2}\right]
\end{eqnarray}
Spectral index and curvature images can be calculated only in regions where
the values in $\vec{I}^{model}_0$ are above a chosen threshold.
%In this case, it is appropriate to smooth the final $\vec{I}^{sky}_{\nu_0}$ image
%with a restoring beam, but not the spectral index or curvature images. 
\item An image cube can be constructed by evaluating the spectral polynomial
{\it via} Eqn.~\ref{Eq:tfunc_1} for each frequency. This form of data product is
useful for sources whose emission is not well modeled by a power law,
but is a smooth polynomial in frequency. Band-limited signals that taper off
smoothly in frequency are one example.
\item An image of the continuum flux can be constructed by evaluating and adding up
the flux at all frequencies. This continuum image is different from the 
reference-frequency image which represents the flux measured at only one frequency.
\end{enumerate}

\noindent \paragraph{\bf Primary-beam correction : } 
A correction for the average primary-beam and its frequency dependence can be
done as a post-deconvolution step. The primary-beams are first evaluated or measured 
as a function of frequency, and the frequency-dependence per pixel modeled by a
power-law or a polynomial (perferably the same spectral polynomial used for the image
reconstruction). Primary-beam correction can be done as follows.
%\begin{equation}
%\vec{I}^{new}_{\nu_0}=\vec{I}^m_{\nu_0}/\vec{\Pb}_{\nu_0}, \vec{I}^{new}_{\alpha}=\vec{I}^m_{\alpha}-\vec{\Pb}_{\alpha}, \vec{I}^{new}_{\beta}=\vec{I}^m_{\beta}-\vec{\Pb}_{\beta}
%\end{equation}
\begin{eqnarray}
\vec{I}^{new}_{\nu_0}&=&\vec{I}^m_{\nu_0}/\vec{\Pb}_{\nu_0}\\
 \vec{I}^{new}_{\alpha}&=&\vec{I}^m_{\alpha}-\vec{\Pb}_{\alpha}\\
 \vec{I}^{new}_{\beta}&=&\vec{I}^m_{\beta}-\vec{\Pb}_{\beta}
\end{eqnarray}
Note that if a polynomial is fit for the frequency-dependence of the primary beam, 
and $\vec{\Pb}_{\nu_0}, \vec{\Pb}_{\alpha}, \vec{\Pb}_{\beta}$ computed from it, the above 
operation is numerically identical to doing a polynomial division in terms of two sets of
coefficients (for $N_t<=3$). A brute-force polynomial-division using more series coefficients
will yield a more accurate solution.   Note however, that such a correction will not be
accurate if there are time-dependent variations in the primary-beam, and will require
integration with the AW-Projection algorithm discussed in \citep{AW-Projection}.



\subsubsection{Software Implementation : }
The MS-MFS algorithm described in section \ref{Sec:MSMFS} has been implemented and 
released {\it via} the CASA\footnote{\href{http://casa.nrao.edu}
{Common Astronomy Software Applications} is being developed at the
\href{http://www.nrao.edu}{National Radio Astronomy Observatory}} 
software package (version 2.4 onwards). The data products are $N_t$
spectral-coefficient images, a spectral index image, and a curvature images (if $N_t>2$).
The parameters that control the algorithm are (a) $\nu_0$ : a reference frequency chosen near the middle of the sampled
frequency range, about which the Taylor expansion is performed, 
(b) $N_t$ : the number of coefficients of the Taylor polynomial to solve for, and 
(c) $N_s$ and $\I^{shp}_s$ : a set of scale sizes in units of image pixels to use
for the multi-scale representation of the image.
The resulting wide-band image model can then be used within a standard 
self-calibration loop. 

%\noindent{\bf Data Products : }
%The basic products of the MS-MFS algorithm are a set of 
%$N+1$ multi-scale coefficient images that describe the spectrum of the
%sky brightness at each pixel (coefficients of an $N^{th}$-order polynomial).
%%(see section \ref{Sec:mfs_model}). 
%The $0^{th}$-order coefficient image is the Stokes I intensity image at the reference
%frequency (not the continuum image defined as the
%integrated flux across the full sampled bandwidth). To create the continuum
%image, the polynomial has to be evaluated and summed over all frequency channels.
%Derived quantities such as the spectral index and spectral curvature are 
%computed from the coefficient images (see Eqns.~\ref{Eq:calcab_1} 
%to \ref{Eq:calcab_3}).
%
%\noindent{\bf Control parameters :} 
%The three main parameters that control the operation of the MS-MFS algorithm
%are (a) $\nu_0$ : a reference frequency chosen near the middle of the sampled
%frequency range, about which the Taylor expansion is performed, 
% (b) $N_t$ : the number of coefficients of the Taylor polynomial to solve for, and 
% (c) $N_s$ and $\I^{shp}_s$ : a set of scale sizes in units of image pixels to use
%for the multi-scale representation of the image.
%
%
%\noindent{\bf Wide-band self-calibration : }
%The broad-band flux model generated by the MS-MFS algorithm can be used within 
%a self-calibration loop in exactly the same manner as standard self-calibration.
%The purpose of such a self-calibration would be to 
%improve the accuracy of the calibration.
%


%\newpage
\begin{algorithm}[Ht!]
\label{ALGO:MSMFS_1}
  \SetLine
  \linesnumbered
  \dontprintsemicolon
%  \SetKwRepeat{Repeat}{Repeat}{Until}
%  \SetKwFor{ForEach}{ForEach}{Do}{End}
%\KwData{ $\vec{V}^{corr}_{\nu}, \vec{I}^{shp}_s~\forall~s\in\{0,N_s$-$1\}, ~ [\Sna]~ \forall \nu$ }
%\KwResult{ $\vec{I}^{psf}_{{sp}\atop{tq}},[{H^{peak}_s}], \vec{I}^m_{\nu_0}, \vec{I}^{\alpha}, \vec{I}^{\beta}$ }
  \KwData{calibrated visibilities : $\vec{V}^{corr}_{\nu}~~\forall \nu$}
  \KwData{$uv$-sampling function : $[\Sa_{\nu}]$}
  \KwData{image noise threshold and loop gain $\sigma_{thr}, g_s$}
  \KwData{scale basis functions : $\I^{shp}_s ~ \forall s\in\{0,N_s-1\}$}
  \KwResult{model coefficient images : $\I^{m}_{q}~ \forall q\in\{0,N_t-1\}$}
  \KwResult{spectral index and curvature : $\I^{m}_{\alpha},\I^{m}_{\beta}$}

  \vspace{0.5cm} 
%{Compute the dirty image $\vec{I}^{dirty}$ and psf $\vec{I}^{psf}$}\;
  \For{ $t \in \{0,N_t-1\}, q \in \{t,N_t-1\}$}
  {
        { Compute the spectral PSF } $\vec{I}^{psf}_{tq}$\;
        \For{ $s \in \{0,N_s-1\}, p \in \{s,N_s-1\}$}
	{
		{Compute the scale-spectral PSF} $\vec{I}^{psf}_{{sp}\atop{tq}} = \vec{I}^{shp}_s \star \vec{I}^{shp}_p \star \vec{I}^{psf}_{tq} $\;
	}
  }
  \For { $s \in \{0,N_s-1\}$}
  {
     Construct $[{H^{peak}_s}]$ from $mid(I^{psf}_{{s,s}\atop{t,q}})$ and compute $[{H^{peak}_s}^{-1}]$\;
  }
%%%%%%%%%%%%%%%%%%%%%%
%%%\vspace{0.5cm}
%%  \caption[MS-MFS CLEAN : Pre-Deconvolution Setup]
%%          {MS-MFS CLEAN : Pre-Deconvolution Setup}
%%\end{algorithm}
%%%\newpage
%%
%%\begin{algorithm}[H]
%%  \SetLine
%%  \linesnumbered
%%  \dontprintsemicolon
%%%%%%%%%%%%%%%%%%%%%%%  
%%\KwData{ $\vec{V}^{corr}_{\nu}, \vec{I}^{shp}_s, \vec{I}^{psf}_{{sp}\atop{tq}},[{H^{peak}_s}]~~~\forall~~s \in \{0,N_s-1\}, p \in \{s,N_s-1\}, \sigma_{thr}$ }
%%\KwResult{$I^{m}_{q}~ \forall q \in \{0,N_t-1\}$}
%%%%%%%%%%%%%%%%%%%%%  
  Initialize the model $\vec{I}^{m}_t$ for all $t \in \{0,N_t-1\}$ and compute $f_{sidelobe}$ \;
  \Repeat (\tcc*[f]{Major Cycle}) { Peak residual in $\vec{I}^{res}_0 < \sigma_{thr}$ }
  {
    \For{$t \in \{0,N_t$-$1\}$}
    {
      Compute the residual image $\vec{I}^{res}_t$ \;
      \For{$s \in \{0,N_s$-$1\}$}
      {
	      Compute $\vec{I}^{res}_{{s},{t}} = \vec{I}^{shp}_s \star \vec{I}^{res}_t$
      }
    }
    Calculate $f_{limit}$ from $\vec{I}^{res}_{0,0}$\;
    \Repeat (\tcc*[f]{Minor Cycle}){ Peak residual in $\vec{I}^{res}_{0,0} < f_{limit} $ } 
    {
%     Compute $I^{m}_q~\forall q\in \{0.N_t-1\}$ and update $\vec{I}^{res}_{{s},{t}}~\forall s,t$ (Algorithm \vref{ALGO:MSMFS_2})\;
     \For{$s \in \{0,N_s$-$1\}$}
     {
      \uIf{Peak of $\vec{I}^{res}_{s,0} > 10~\sigma_{thr} $}
      {
       \ForEach{pixel}
       {
          Construct $I^{rhs}_s$, an $N_t\times 1$ vector from $I^{res}_{s,t} ~~\forall ~ t \in \{0,N_t$-$1\}$\;
          Compute principal solution $I^{sol}_s = [{H^{peak}_s}^{-1}] I^{rhs}_s$\;
       }
       Choose $I^{sol} = max\{I^{sol}_{t=0},~\forall~s\in\{0,N_s$-$1\}\}$ \;
      }
      \Else
      {
       Find the location of the peak in $\vec{I}^{res}_{s,0},~\forall~s\in\{0,N_s$-$1\}$\;
       Construct $I^{rhs}_s$, from $I^{res}_{s,t}$ for the chosen $s$, at this location\;
       Compute $I^{sol} = [{H^{peak}_s}^{-1}] I^{rhs}_s$ at this location\;
      }
     }
       \For{$t \in \{0,N_t-1\}$}
       {
        Update the model image : $I^{m}_t = I^{m}_t + g_s ~ I^{shp}_{s_i} \star I^{sol}_t$ \;
        \For{$s \in \{0,N_s$-$1\}$}
	{
          Update the residual image : $I^{res}_{s,t} = I^{res}_{s,t} - g ~\sum_{p=0}^{N_s-1}\sum_{q=0}^{N_t-1}[I^{psf}_{{sp}\atop{tq}} \star I^{sol}_q]$\;
	}
       }
    }
   Compute model visibilities $V^{m}_{\nu}$ from  $I^{m}_t~\forall t\in \{0.N_t-1\}$\;
   Compute a new residual image $I^{res}$ from residual visibilities $V^{corr}_{\nu}-V^{m}_{\nu}$\;
  }
Calculate $\vec{I}^m_{\nu_0}, \vec{I}^{\alpha}, \vec{I}^{\beta}$ from $I^{m}_t~\forall t\in \{0.N_t-1\}$ and restore the results \;
If required, remove average primary beam : $\vec{I}^{new}_{\nu_0}=\vec{I}^m_{\nu_0}/\vec{\Pb}_{\nu_0}, \vec{I}^{new}_{\alpha}=\vec{I}^m_{\alpha}-\vec{\Pb}_{\alpha}, \vec{I}^{new}_{\beta}=\vec{I}^m_{\beta}-\vec{\Pb}_{\beta}$\;
\vspace{0.5cm}

  \caption[MS-MFS Algorithm]
          {MS-MFS Algorithm : }
\end{algorithm}

%\newpage
%\begin{algorithm}[t!]
%\label{ALGO:MSMFS_2}
%  \SetLine
%  \linesnumbered
%  \dontprintsemicolon
%%%%%%%%%%%%%%%%%%%%%%  
%\KwData{ residual images : $\vec{I}^{res}_{s,t}$}
%\KwData{ scale basis functions : $\vec{I}^{shp}_s$}
%\KwData{ scale-Spectral PSFs : $\vec{I}^{psf}_{{sp}\atop{tq}}~~~\forall~~s \in \{0,N_s-1\}, p \in \{s,N_s-1\}$}
%\KwData{ Hessian for each scale : $[{H^{peak}_s}]~~~\forall~~s \in \{0,N_s-1\}$ }
%\KwResult{model coefficient images : $I^{m}_{q}~ \forall q \in \{0,N_t-1\}$}
%\KwResult{updated residual images : $I^{res}_{s,t}~ \forall s \in \{0,N_s-1\},~ t \in \{0,N_t-1\}$}
%\vspace{0.5cm}
%%%%%%%%%%%%%%%%%%%%%%  
%
%%    \Repeat (\tcc*[f]{Minor Cycle}){ Peak residual in $\vec{I}^{res}_{0,0} < f_{limit} $ } 
%%    {
%     \For{$s \in \{0,N_s$-$1\}$}
%     {
%      \uIf{Peak of $\vec{I}^{res}_{s,0} > 10~\sigma_{thr} $}
%      {
%       \ForEach{pixel}
%       {
%          Construct $I^{rhs}_s$, an $N_t\times 1$ vector from $I^{res}_{s,t} ~~\forall ~ t \in \{0,N_t$-$1\}$\;
%          Compute principal solution $I^{sol}_s = [{H^{peak}_s}^{-1}] I^{rhs}_s$\;
%       }
%       Choose $I^{sol} = max\{I^{sol}_{t=0},~\forall~s\in\{0,N_s$-$1\}\}$ \;
%      }
%      \Else
%      {
%       Find the location of the peak in $\vec{I}^{res}_{s,0},~\forall~s\in\{0,N_s$-$1\}$\;
%       Construct $I^{rhs}_s$, from $I^{res}_{s,t}$ for the chosen $s$, at this location\;
%       Compute $I^{sol} = [{H^{peak}_s}^{-1}] I^{rhs}_s$ at this location\;
%      }
%     }
%       \For{$t \in \{0,N_t-1\}$}
%       {
%        Update the model image : $I^{m}_t = I^{m}_t + g_s ~ I^{shp}_{s_i} \star I^{sol}_t$ \;
%        \For{$s \in \{0,N_s$-$1\}$}
%	{
%          Update the residual image : $I^{res}_{s,t} = I^{res}_{s,t} - g ~\sum_{p=0}^{N_s-1}\sum_{q=0}^{N_t-1}[I^{psf}_{{sp}\atop{tq}} \star I^{sol}_q]$\;
%	}
%       }
%%    }
%
%\vspace{0.5cm}
%  \caption[MF-MFS CLEAN : minor cycle steps]
%          {MF-MFS CLEAN : minor cycle steps}
%\end{algorithm}


\subsection{Relation to MF-Clean}
A point-source multi-frequency deconvolution algorithm can be derived by setting  $N_s=1$ and
using $I^{shp}_0 = \delta$-function in the derivations in sections \ref{Sec:meqn} and \ref{Sec:neqn}.
This section discusses the difference between the resulting multi-frequency Hessian and RHS vectors
and those described in the MF-Clean algorithm \citep{MFCLEAN}, and shows that the MF-Clean
approach will incur errors for arrays with dense and irregular spatial-frequency coverage where
tracks from different baselines intersect.

\noindent The normal equations can be written in block matrix form (for example, for $N_t=3$).
\begin{equation}
\left[\begin{array}{lll} 
   [H_{0,0}] & [H_{0,1}] & [H_{0,2}]\\  
\noalign{\medskip}
   [H_{1,0}] & [H_{1,1}] & [H_{1,2}] \\  
\noalign{\medskip}
   [H_{2,0}] & [H_{2,1}] & [H_{2,2}] \\  
   \end{array} \right]
\left[\begin{array}{l} \vec{I}^{sky}_{0} \\ 
\noalign{\medskip}
                       \vec{I}^{sky}_{1} \\ 
\noalign{\medskip}
		       \vec{I}^{sky}_{2}\end{array}\right] =
\left[\begin{array}{l} \vec{I}^{dirty}_{0}\\
\noalign{\medskip}
		       \vec{I}^{dirty}_{1} \\
\noalign{\medskip}
		       \vec{I}^{dirty}_{2}\end{array}\right] 
\label{Eq:mfs_neqn3}
\end{equation}
where each block $[\He_{t,q}]$ is a convolution operator with $\vec{I}^{psf}_{t,q}$ as its kernel.
\begin{eqnarray}
%\label{Eq:mfs_neqn_1}
%\sum_{q=0}^{N_t-1} [H_{t,q}] \vec{I}^{sky}_{q} &=& \vec{I}^{dirty}_{t} ~~~~~~~~~~~~~ \forall~ t ~\in ~\{0 ... N_t-1\} \\
%\label{Eq:mfs_neqn_2a}
% [H_{t,q}] &=& \sum_{\nu} \wntq [\He_{\nu}]\\
\label{Eq:mf_neqn_2b}
\vec{I}^{psf}_{t,q}  &=& \sum_{\nu} \wntq \vec{I}^{psf}_{\nu}\\
\label{Eq:mf_neqn_3a}
\vec{I}^{dirty}_{t}  &=& \sum_{\nu} \wnt \vec{I}^{dirty}_{\nu}
%\label{Eq:specPSF}
%\vec{I}^{psf}_{t,q}&=& \sum_{\nu} \wntq [\Fd\Snd\Wimn] \vec{1} = \sum_{\nu} \wntq \vec{I}^{psf}_{\nu}
\end{eqnarray}
%
The MF-Clean algorithm described in \citep{MFCLEAN} follows a
matched-filtering approach using functions called spectral-psfs, 
%\begin{equation}
%\vec{I}^{psf}_t = \sum_{\nu}\wnt \vec{I}^{psf}_{\nu}
%\label{Eq:mfclean_1}
%\end{equation}
which are equivalent to the convolution kernels from the first row
of Hessian blocks ($q=0$) in Eqn.\ref{Eq:mf_neqn_2b}.
%\item The convolution kernels from the first row of Hessian blocks
% ($s=0,t=0, p\in\{0,N_s-1\}, q\in\{0,N_t-1\}$) represent the
%instrument's response functions to a flux component of unit total flux whose shape is
%given by the $p^{th}$ scale basis function and whose spectrum is given
%by the $q^{th}$ Taylor function. 
Hessian elements and RHS vectors are calculated by convolving spectral-psfs with 
themselves and the residual images.
\begin{eqnarray}
\label{Eq:mfclean_2}
\vec{I}^{psf}_{t,q}  &=& \left\{ \sum_{\nu}\wnt \vec{I}^{psf}_{\nu}  \right\} \star \left\{ \sum_{\nu}\wnq \vec{I}^{psf}_{\nu}  \right\}\\
\label{Eq:mfclean_3}
\vec{I}^{dirty}_{t}  &=& \left\{ \sum_{\nu}\wnt \vec{I}^{psf}_{\nu}  \right\} \star \left\{\sum_{\nu} \vec{I}^{dirty}_{\nu}  \right\}
\end{eqnarray}
%Note the difference between Eqns \ref{Eq:mfclean_2} and \ref{Eq:mfclean_3}
% with Eqns.\ref{Eq:mf_neqn_2b} and \ref{Eq:mf_neqn_3a}.
Formally, this matched filtering approach is exactly equal to the
calculations shown in 
Eqns.\ref{Eq:mf_neqn_2b} and \ref{Eq:mf_neqn_3a} only under the conditions that there
is no overlap on the spatial frequency plane between measurements from different
observing frequencies, and all measurements are weighted equally across the spatial-frequency
plane (uniform weighting).
Appendix \ref{App:A} contains a derivation that demonstrates this.

In practice this difference manifests itself as follows.
Consider a spatial-frequency grid cell onto which measurements from
two different baselines and frequencies map. Let $V_1,V_2$ be the measured visibilities
at two frequencies $1,2$ and let $w_1,w_2$ be their Taylor-weights.
A matched-filtering approach 
calculates $(w_1+w_2)(V_1+V_2)$, whereas Eqns.\ref{Eq:mf_neqn_2b} and \ref{Eq:mf_neqn_3a}
require the computation of $(w_1 V_1) + (w_2 V_2)$.  The two are equivalent only for
flat spectrum sources where $V_1=V_2$ or when there is no overlap between the spatial
frequency grid cells measured 
from different observing frequencies ($V_1$ and $V_2$ map to different spatial frequencies).

The MF-Clean algorithm was initially developed for the ATCA telescope,
an East-West array of antennas with circular $uv$-coverage patterns and minimal
spatial-frequency overlap across channels. This matched filtering approach therefore
worked well.  However, when applied to data from the VLA (where $uv$-tracks intersect
each other and there is considerable spatial-frequency overlap), numerical
instabilities limited the fidelity of the final image, especially with
extended emission.  Changing the computations
to those in Eqns.\ref{Eq:mf_neqn_2b} and \ref{Eq:mf_neqn_3a}
eliminated this instability (determined using simulated VLA data).




%%%%%%%%%%%%%%%%%%%%%%

\subsection{Relation to MS-Clean}
A narrow-band (or flat-spectrum) multi-scale deconvolution algorithm can be derived by
setting $N_t=1$ in the derivations in sections \ref{Sec:meqn} and \ref{Sec:neqn}.
%The main difference between this method and those described in  \citep{MSCLEAN} and \citep{ERIC_MSCLEAN}
%is in the calculation of the multi-scale principal solution before searching for the dominant flux component. 
%

\noindent The normal equations can be written in block matrix form (for example, for $N_s=2$).
\begin{equation}
\left[\begin{array}{ll} 
\noalign{\medskip}
   [H_{0,0}] & [H_{0,1}] \\  
\noalign{\medskip}
   [H_{1,0}] & [H_{1,1}]  \\  
\noalign{\medskip}
   \end{array} \right]
\left[\begin{array}{l} \vec{I}^{sky,\delta}_{0} \\ 
\noalign{\medskip}
                       \vec{I}^{sky,\delta}_{1} \end{array}\right] =
\left[\begin{array}{l} \vec{I}^{dirty}_{0}\\
\noalign{\medskip}
		       \vec{I}^{dirty}_{1} \end{array}\right] 
\label{neqn_ms_math}
\end{equation}
%where each block $[\He_{s,p}]$ is a convolution operator with $\vec{I}^{psf}_{s,p}$ as its kernel.
%\begin{eqnarray}
%%\mathrm{where}~~~ [H_{s,p}] &=& [\Fd T_{s}\F][B][\Fd T_{p}\F] \\%~~~~~~~~\mathrm{where}~~~~~~ [B]=[\Fd\Sd\Wim\Sa\F]\\
%\vec{I}^{psf}_{s,p} &=& \vec{I}^{shp}_s \star \vec{I}^{psf} \star \vec{I}^{shp}_p \\
%		\vec{I}^{dirty}_{s}  &=& \vec{I}^{shp} \star \vec{I}^{dirty} %~~~~~~~~\mathrm{where}~~~~~~ \vec{I}^{dirty} = [\Fd\Sd\Wim] V^{corr}
%\label{Eq:ms_neqn_3}
%\end{eqnarray}
%$[B]$ is the Beam matrix ($[H]$ in Eqn.~\ref{Eq:Hessian}) 
%and $\I^{dirty}$ is the standard dirty image (Eqn.~\ref{Eq:Dirty}).
%
The elements of $[\He^{peak}]$, the $N_s\times N_s$ matrix formed from the multi-scale Hessian blocks
are given by 
\begin{equation}
H^{peak}_{s,p} = mid\left\{\vec{I}^{psf}_{s,p}\right\} = tr([T_s][\Sd\W\Sa][T_p])~~~~\forall~ s,p ~\in ~\{0 ... N_s-1\}
\label{Eq:hpeak}
\end{equation}
The elements on the diagonal of $[H^{peak}]$ correspond to $s=p$ and 
are a measure of the sensitivity of the instrument to a particular spatial scale. 
With uniform weighting, the kernels $\vec{I}^{psf}_{s,s}$ on the diagonal blocks are 
the autocorrelations of the different scale functions $\vec{I}^{shp}_s$, as measured by the
interferometer, and this gives the area under the main beam of the PSF for each spatial scale.
%The off-diagonal elements given by $s\ne p$
%are a measure of the orthogonality  of the basis set,
%for the given $uv$-coverage and weighting scheme.
The principal solution can be computed by inverting $[\He^{peak}]$ and applying it to the
RHS vectors. 

There are two differences between this approach, and those described in \citet{MSCLEAN} and \citet{ERIC_MSCLEAN}.
\begin{enumerate}
\item {\bf Finding a flux component : }
In both forms of MS-Clean, 
the amplitude and scale of a
flux component are chosen by searching for the peak in the list of dirty
images after having applied a scale bias, 
an empirical term that de-emphasises large spatial scales.
The scale bias $b_s = 1-0.6~s/s_{max}$  used in \citet{MSCLEAN}
(where $s_{max}$ is the width of the largest scale basis function)
is a linear approximation of how the inverse of the area under each
scale function changes with scale size\footnote
{g
When $s/s_{max} = 1.0$ the bias term is $1.0 - 0.6 = 0.4$ which is approximately equal
 to the inverse of the area under a Gaussian of unit peak and 
 width, given by ${1.0}/{\sqrt{2\pi}} = 0.398$.}.
%The smoothed residual images at each scale are normalized by this estimate
%before peak values are chosen.
It is meant to be used to normalize residual images that have been smoothed with 
scale functions that have unit peak, before flux components are chosen.
The algorithm described in \citet{ERIC_MSCLEAN} uses 
$b_s \approx 1.0/s^{2x}$ where $x\in\{0.2,0.7\}$, to approximate
a normalization by the area under a Gaussian,
for the case when images are smoothed by applying a $uv$-taper 
that tends to unity for the zero spatial frequency.
%
Both these normalization schemes are 
approximations of using a diagonal approximation of $[H^{peak}]$ 
%and discarding all cross-terms 
when computing the principal solution before picking out flux components. 
%A diagonal approximation will however not give correct values
%for sources composed of overlapping flux components of different spatial scales.

Once we have this understanding, we can see that the full Hessian $[H^{peak}]$ 
(and not just a diagonal approximation)
can be inverted to get the normalization exactly right, especially for sources
that contain overlapping flux components of different spatial scales. 
%(i.e. taking into account
%the non-orthogonality between the different spatial scales).
It can be shown 
that by applying the inverse of the full $[H^{peak}]$ to the RHS vectors,
 we are able to 
get a more accurate estimate of the total-flux of the component at each scale 
than by just reading off a peak from a series of dirty images biased by the MS-Clean $b_s$. 
%This difference has been demonstrated on simulations (Section \ref{Sec:MS-pc})
%where the inverse of
%$[H^{peak}]$ was applied to all pixels of a series of smoothed dirty-images,
%but the relative performance of this approach (compared to the existing methods)
%is yet to be analysed within the complete iterative deconvolution
%framework. It is likely that the technique described
%in Section \ref{MS-deconv} would get more accurate minor cycle estimates 
%and therefore converge in fewer iterations.
However, this solution gives correct values only at the locations of the
centers of the flux-components, and introduces large errors in the psf sidelobes.
Therefore, for reasons of stability, a diagonal approximation is a more 
appropriate choice (demonstrated on simulated VLA data). 

\item{\bf Minor cycle updates :}
The update steps in \citet{MSCLEAN} and Section \ref{MS-MFS}
evaluate the full LHS of the normal equations
to update the smoothed residual images and subtract out flux components 
within the image domain.
This allows each minor cycle iteration to search for the 
optimal flux component across all scales  
without having to recompute smoothed residual images by going to the visibility domain
in each iteration.
\citet{ERIC_MSCLEAN} ignores the cross-terms, performs
a full set of minor cycle iterations on one scale at a time, and recomputes
smoothed residual images {\it via} the visibility domain after every full set of
minor cycle iterations.
\end{enumerate}

\noindent A choice among these three methods
will depend on trade-offs between the accuracy within each minor cycle iteration, 
the computational cost per step, and optimized global convergence
patterns to control the total number of iterations.

%%%%%%%%%%%%%%%%%%%%%%%%%%%%%%%%%%%%%%%%%%%%%%%%%%%%%%%%%%%%%%%%%%%%%%%%5

\section{Hybrids of Narrow-Band and Continuum Techniques}

The preceeding sections discussed multi-frequency solutions that used data from all measured
frequencies together, so as to take advantage of the combined spatial-frequency coverage.
However, there are some situations where single-channel methods used in combination
with multi-frequency-synthesis (and no built-in spectral model) will be able to deliver
scientifically useful wide-band reconstructions at the continuum sensitivity.

%There are many hybrid approaches where the advantage of arbitrary spectral
%reconstructions allowed by single-channel imaging is combined with MFS
%below a certain flux level to make use of continuum sensitivity. 
%These approaches can be very effective for simple fields of point sources
% (ref TechRep/thesis.)

%\noindent{\bf Single-channel imaging : }

%


The basic idea of a hybrid wide-band method is to combine 
%narrow-band channel imaging
%with one or more of the continuum methods in a multi-stage approach. 
%The goal is to combine 
the advantages of single-channel imaging (simplicity and 
insensitivity to source spectra) with those of continuum imaging 
(deconvolution with full continuum sensitivity).
%
\begin{enumerate}
\item Deconvolve each channel separately upto the single-channel sensitivity limit.
At this stage, only sources brighter than $\sigma_{chan}$ will be detected and deconvolved.
\item Remove the contribution of bright (spectrally varying) sources by subtracting out
visibilities predicted from the model image cube.
At this stage, the peak residual brightness is at the level of the single-channel
noise limit $\sigma_{chan}$. 
\item Perform regular multi-frequency-synthesis imaging (flat-spectrum assumption)
on the continuum residuals to extract flux that lies between $\sigma_{chan}$ and $\sigma_{cont}$.
Accordiing to \citep{CCW_MFCLEAN}, and as shown in section XX, 
errors due to this flat-spectrum assumption become visible only above a dynamic
range of 1000 (for $\alpha=-1.0$ and a 2:1 bandwidth ratio). 
Therefore, as long as the sensitivity improvement between a single-channel
the full band is less than 1000, this second step will incur no errors even if the
remaining flux has spectral structure. 
This requirement translates to $N_{chan}<10^6$, which is usually satisfied\footnote
{Even if visibilities are measured at a very high frequency resolution, they can be
averaged across frequency ranges upto the bandwidth-smearing limit for the
desired image field-of-view. 
%For the EVLA in A-configuration between 1 and 2 GHz, this 
%corresponds to few-hundred channels for a single-primary-beam field-of-view.
}
\item Add model images from both steps, and restore the results. 
For unresolved sources, it may be appropriate to use a clean-beam fitted for
the highest frequency, but in general, to not bias spectral information, all channels
should be restored using a clean beam fitted to the lowest frequency in the range.
\end{enumerate}
The advantages of this approach are its simplicity, and that it can handle wide-band
reconstructions with band-limited signals and spectral-lines.
The disadvantages are that the angular resolution of the images and spectral information 
will be restricted to that given by the lowest frequency (a factor of two worse than what is
possible), and high-fidelity deconvolution of resolved structure
requires sufficient single-frequency spatial-frequency
coverage to unambiguously reconstruct all the spatial structure of interest, at all frequencies.

%
%However, all hybrid methods are 
%inherently single-channel methods that depend on having
%sufficient spatial-frequency coverage per channel to  reconstruct all
%the structure of interest. Also, any spectral information can be obtained
%only at the angular resolution of the lowest frequency.


%__________________________________________________________________

\section{Imaging results}\label{Sec:IMAGINGRESULTS}
\subsection{Simulation}

\paragraph{Objective : }
The goals of this test are to assess the ability of the MS-MFS algorithm 
to reconstruct both spatial and spectral 
information about a source in terms of a linear combination of compact and 
extended flux components with 
polynomial spectra (flux model described in section \ref{Sec:MSMFS-model})
as well as to test how appropriate this flux model is when the 
true sky brightness is a complex extended source whose spectral
characteristics vary smoothly across its surface.


\paragraph{Sky brightness : }
Wide-band EVLA observations were simulated for a sky brightness distribution 
consisting of one point source with spectral index of $-$2.0 and 
two overlapping Gaussians with spectral indices of $-$1.0 and $+$1.0. 
Fig.\ref{Fig:msmfs_sim_1} shows the reference
frequency image of this simulated source, 
plots of the spectrum at different locations on the source, and the resulting
spectral index and curvature maps. 
The spectral
index across the resulting extended source varies smoothly between $-$1.0 and $+$1.0,
with a spectral turnover in the central region corresponding to a spectral 
curvature of approximately 0.5. 
Fig.\ref{Fig:msmfs_sim_11} shows the first
three Taylor coefficient maps that describe this source.

\paragraph{MS-MFS Imaging : }
Two wide-band imaging runs were done using the MS-MFS algorithm and the results
compared. The first used a multi-scale flux model (section \ref{Sec:MSMFS-model}) 
in which $N_t=3$ and $N_s=4$ with scale sizes
defined by widths of $0,6,18,24$ pixels and the second used a point-source 
flux model in which $N_t=3$ and $N_s=1$ with one scale
function given by the $\delta$-function (to emulate the MF-CLEAN algorithm described 
in section \ref{Sec:mfs_model}).
A $5\sigma$ flux threshold of about $20\muup$Jy was used as the
termination criterion.

\paragraph{Results : }
The results from these imaging runs are shown in %Fig.\ref{Fig:msmfs_sim_2} 
Fig.~\ref{Fig:msmfs_sim_12} (three Taylor coefficients), 
Figure \ref{Fig:msmfs_sim_3} shows residual images 
over a larger region of the sky, and Fig.~\ref{Fig:msmfs_sim_2} shows the
intensity at the reference frequency, spectral index and spectral curvature.
All figures show the results with both MS-MFS and MF-CLEAN. 
%as images of 
%the intensity at the reference frequency (top), the 
%spectral index (middle) and spectral curvature (bottom).
%The left column shows the results with a multi-scale flux model 
%and the right column shows the results with a point-source 
%flux model.


%
\begin{figure}[t!]
%%\epsfig{figure=algomsmfs/msmfs_model.eps,scale=0.78}
\epsfig{figure=pics/fig_msmfs_1.eps,scale=0.3}
\caption[Example : Simulated wide-band sky brightness distribution]
{\small Simulated wide-band sky brightness distribution : 
These images represent the wide-band sky brightness distribution that was
used to simulate EVLA data to test the MS-MFS algorithm. 
%This true sky brightness is
%shown in terms of the final data products that the MS-MFS algorithm produces (total intensity
%spectral index and spectral curvature). in order to be able to evalua imaging
%performance. 
The image on the top left shows the
total intensity image of the source at the reference frequency $\I_{\nu_0}$. The plots
on the bottom left show spectra (and their power law parameters) at 4 different locations.
%different parts of the field extended source and for one isolated point source.
The spectral index varies smoothly between about $+$1 and $-$1 across the extended source 
and is $-$2.5 for the point source. 
The spectral curvature has significant values only in the central region of the extended source
where the spectrum turns over within the sampled range.
The images on the right show these trends in the form of spectral index (top) and
spectral curvature (bottom) maps. 
}
\label{Fig:msmfs_sim_1}
%\end{figure}
\vspace*{0.5cm}
%\begin{figure}
\epsfig{figure= pics/fig_msmfs_2.eps,scale=0.3}
\caption[Example : True Taylor coefficient images]
{\small True Taylor coefficient images :
These images show the first three Taylor coefficients for the polynomial expansion
of the wide-band flux distribution shown in Fig.~\ref{Fig:msmfs_sim_1}.
These images are the (left) intensity at the reference frequency $I_{0} = I_{\nu_0}$,
(middle) first-order Taylor-coefficient $I_1 = \alpha I_{\nu_0}$
and (right) second-order Taylor-coefficient 
$I_2 = \left({\alpha(\alpha-1)}/{2} + \beta \right) I_{\nu_0}$ 
(see Eqn.~\ref{Eq:taylorexpansion}). All images are displayed at the same flux scale.
}
\label{Fig:msmfs_sim_11}
\end{figure}

\begin{figure}
\epsfig{figure=pics/fig_msmfs_12.eps,scale=0.3}
\caption[Example : Reconstructed Taylor coefficient images]
{\small Reconstructed Taylor coefficient images :
These images show the first three Taylor coefficients (similar to Fig.~\ref{Fig:msmfs_sim_11})
obtained using two different wide-band flux models.
The top row shows the results of using a multi-scale wide-band 
flux model (MS-MFS) and the bottom row shows the results of using a point-source wide-band flux model
(MF-CLEAN, or MS-MFS with only one spatial scale given by a $\delta$-function).
All images are displayed at the same flux scale.
}
\label{Fig:msmfs_sim_12}
\end{figure}

\begin{figure}[t!]
\begin{center}
\epsfig{figure=pics/bigpic_multiscale_residual.eps,scale=0.15}
\epsfig{figure=pics/bigpic_pointsource_residual.eps,scale=0.15}
\end{center}
\caption[Example : Residual images]
{\small Residual images : This figure shows the residual images obtained after
applying MS-MFS to wide-band EVLA data simulated for the
sky brightness distribution shown in Fig.\ref{Fig:msmfs_sim_2}.
The residual image on the left is obtained when a multi-scale flux model was used (MS-MFS). 
The RMS noise on source is about 20 $\muup$Jy and off source is 5 $\muup$Jy.  
Compare this with the residual image on the right from a point-source deconvolution (MF-CLEAN) where
the on source RMS is about 0.2 mJy and off source is 50 $\muup$Jy. 
(Note that the displayed data ranges are different for these two images.
 The flux scale for the image on the left is $\pm0.3 \times 10^{-4}$ and 
 for the right is $\pm0.3\times 10^{-3}$.)
This clearly demonstrates the advantage of using a
multi-scale flux model.
}
\label{Fig:msmfs_sim_3}
\end{figure}

\begin{figure}[t!]
\begin{center}
\epsfig{figure=pics/fig_msmfs_3.eps,scale=0.3}
\end{center}
\caption[Example : MS-MFS final imaging data products]
{\small MS-MFS final imaging data products : 
These images show the results of applying MS-MFS to wide-band 
EVLA data simulated for the
sky brightness distribution described in Fig.\ref{Fig:msmfs_sim_1}.
The left column shows the results of using a multi-scale wide-band 
flux model (MS-MFS) and the right
column shows the results of using a point-source wide-band flux model
(MF-CLEAN, or MS-MFS with only one spatial scale given by a $\delta$-function). 
The top, middle and
bottom rows correspond to the intensity image at the reference frequency
$I_{\nu_0}$, the spectral index $\alpha$ and spectral curvature $\beta$ maps respectively.
The flux scale for each left/right pair of images is the same, and
the sharp source boundaries in the spectral index and curvature maps are because 
of a flux threshold used to compute them.
With a multi-scale flux model (MS-MFS, left), the reconstructions of $\alpha$
and $\beta$ are accurate to within 0.1 in high signal-to-noise regions.
With a point-source flux model (MF-CLEAN, right), deconvolution errors break
extended emission into flux components of the size of the resolution element
and these errors transfer non-linearly to the spectral index and curvature maps.
Table \ref{Tab:evlasim_err} compares the true and reconstructed values of $I_{\nu_0},\alpha,\beta$
for three regions of this sky brightness distribution.
}
\label{Fig:msmfs_sim_2}
\end{figure}

\clearpage
\noindent The main points to note from these images are listed below.
\begin{enumerate}
%\item Multi-scale vs point source : residuals and image dynamic range, fidelity, error bars on a,b.
\item With a multi-scale multi-frequency flux model (MS-MFS) 
the spectral index across the extended source was reconstructed 
to an accuracy of $\delta\alpha < 0.05$ with the maximum error being in the
central region where the spectral index goes to zero and $N_t=3$ is too high 
for an accurate fit (section \ref{Sec:Errs} describes how the choice of
$N_t$ affects the solution process). 
The spectral curvature across the extended source was estimated to an accuracy of
$\delta\beta<0.1$ in the central region with the maximum error of 
$\delta\beta \approx 0.2$ in the regions 
where the curvature signal goes to zero and the source surface brightness
is also minimum (the outer edges of the source).

\item With a multi-frequency point-source model (MF-CLEAN)
the accuracy of the spectral index and curvature maps was limited to
$\delta\alpha\approx 0.1,\delta\beta\approx 0.5$.
This is because the use of a point source model will break any 
extended emission into components the size of the resolution element
and this leads to deconvolution errors well above the
off-source noise level (note the difference between the
intensity images $I(\nu_0)$ produced with MS-MFS {\it vs} MF-CLEAN).
%a multiscale and point-source
%(right) model \cite{CORNWELL_CLEAN_1983,DANS_THESIS}). 
Error propagation during the computation of spectral index and 
curvature as ratios of these noisy reconstructed images leads to
high error levels in the result.

%\item For the isolated point source, both multi-frequency flux models were able to
%reconstruct the spectral index and curvature 
%to accuracies of $< 0.002$ {\tt CHECK!!!!!!!!!!!!!!!!}. 
%These numbers coincide with
%$\delta\alpha$ calculated for an image-domain error of 0.001. 
%
%Figure \ref{Fig:msmfs_sim_3} shows the residual images 
%over a larger region of the sky for the two test runs described above.
%(multi-scale image model (left) and a point-source model (right)).  
\item The imaging run that used a multi-scale image model was terminated at a $5\sigma$
noise threshold. The peak residual is about 20 $\muup$Jy and the off-source RMS is $5 \muup$Jy
(close to the theoretical RMS of 3 $\muup$Jy as listed in Table \ref{Tab:evlasim}).
The imaging run that used a point-source model was terminated after at least
four successive major cycles failed to reduce the peak residual below 200$\muup$Jy despite
an apparant decrease in the residuals during the minor cycle iterations.
The off source RMS in the result is about 50 $\muup$Jy.

\end{enumerate}



\subsection{Multi-frequency VLA observations of Cygnus-A}

\paragraph{Objective : }
Wide-band VLA observations of the bright radio galaxy Cygnus~A were used to
test the MS-MFS algorithm on real data as well as to test standard
calibration methods on wide-band data.
Most of the images so far made of Cygnus~A and
its spectral structure have been from large amounts of multi-configuration
narrow-band VLA data \cite{CYGA_1991} designed so as to measure the spatial structure as
completely as possible at two widely separated frequencies.
The goal of this test was to use multi-frequency
snapshot observations of Cygnus~A to evaluate how well the MS-MFS algorithm is 
able to simultaneously reconstruct its spatial and spectral structure
from measurements in which the single-frequency $uv$-coverage was insufficient
to accurately reconstruct all the spatial structure at that frequency.


\begin{figure}[t!]
\epsfig{figure=pics/cyga_uvcoverage.eps,scale=0.4}
\caption[VLA multi-frequency $uv$-coverage]
{\small VLA multi-frequency $uv$-coverage : 
	This figure shows the multi-frequency $uv$-coverage of VLA observations of Cygnus~A,
taken as a series of narrow-band snapshot observations. The plots on the left show the
$uv$-coverage from one frequency channel (20 snapshots at 1.7 GHz).
By zooming into the central region (bottom left) and comparing the spacing between the
measurements to the size of the $uv$ grid cells being used for imaging we can show that
the single-frequency measurements are incomplete. 
The plot on the right shows the multi-frequency $uv$-coverage using nine frequency tunings.
A zoom-in of the same central region (bottom right) shows that for the chosen $uv$ grid cell
size (or image field of view over which the image is to be reconstructed) the combined
sampling leaves no unmeasured grid cells. The imaging results from these observations will
test our ability to reconstruct both spatial and spectral information from
incomplete spatial frequency samples at a discrete set of frequencies.}
\label{Fig:cyga_uvcoverage}
\end{figure}

\paragraph{Cygnus~A } 
%Wide-band VLA observations of the bright radio galaxy Cygnus~A were used to
%test the MS-MFS algorithm on real data as well as to test standard
%calibration methods on wide-band data.
Cygnus~A an extremely bright (1000 Jy) radio galaxy with 
a pair of bright compact hotspots about 1 arcmin away from each other on either
side of a very compact core, 
and extended radio lobes associated with the hotspots that have  
broad-band synchrotron emission at multiple spatial scales. 
From many existing measurements \cite{CYGA_1996}, this radio source is
known to have a spatially varying spectral
index ranging from near zero at the core, -0.5 at the bright hotspots and up to
-1.0 or more in the radio lobes. 

\paragraph{Observations : } Wide-band data were taken as described in 
Table \ref{Tab:vladata_cyg} using the
VLA 4-IF mode which allowed four simultaneous data streams containing
RR and LL correlations at two independent frequency tunings.
A set of 18 frequencies were chosen such that they spanned
the entire frequency range allowed by the new EVLA receivers (1--2 GHz).
Visibilities that used antennas with the older receivers were flagged for regions of 
the band not covered by the receivers (below 1.2 GHz and above 1.8 GHz). 
The $uv$-coverage for this dataset for the RR correlations is shown
in Fig.\ref{Fig:cyga_uvcoverage}.
The data were inspected visually and visibilities that were affected by 
strong radio frequency interference were flagged (masked).

\paragraph{Calibration :} 
Standard techniques were used to calibrate these data.
Flux calibration at each frequency was done {\it via} observations of 3C286.
Phase calibration was done using 
an existing narrow-band image of Cygnus~A at 1.4 GHz 
\cite{CYGA_1991} as a model. 

At the time of these observations, the VLA correlator was getting inputs from
a combination of VLA and EVLA antennas. A gain control system that was temporarily
put in place to accomodate the use of new EVLA antennas with the VLA correlator
treated the two independent frequency tunings
in the 4-IF mode differently\footnote
{
To allow the use to new EVLA antennas with
the old VLA correlator, an automatic gain control had to be used at each EVLA antenna
to mimic the old VLA antennas and ensure that the input power levels to the VLA correlator 
were within the range over which it has a linear response. The type of gain control 
was being done differently for the two simultaneous
frequency tunings in the VLA 4-IF mode. The A/C IF stream used an automatic
gain controller based on power levels measured in 1 second and the B/D IF stream
used a static look-up table to decide attenuation levels.
This resulted in a difference in power levels
for the A/C and B/D data streams for all baselines that involved EVLA antennas
when the source being observed was bright enough to contribute to increasing the
overall system temperature.
}. This caused errors in the correlator input for very strong sources (Cygnus~A)
that increased the input power level beyond the linear power range of the VLA correlator.
%and as a result of this effect, 
Observations of the calibrator source 3C286 were not affected by this problem.
We were therefore able to calibrate all the frequency tunings for Cygnus~A and use 
the resulting wide-band spectrum along with the known integrated flux and spectral index of
Cygnus~A to idenfity which of the frequency tunings of Cygnus~A 
were affected. It was found that every alternate frequency (the second of each pair 
of simultaneous frequency tunings (B/D) in the VLA 4-IF mode) was affected.
Therefore to safely eliminate the effect of this problem for our tests, 
one of the two simultaneous frequency tunings were flagged from the recorded 
visibilities reducing the number of spectral windows from 18 to 9.
The final dataset used for imaging consisted of nine spectral windows
each of a width of about 4 MHz and separated by about 100 MHz.

\paragraph{Imaging : }
These data were imaged using two methods, the  
MS-MFS algorithm and a hybrid method consisting of STACK + MFS on residuals
(see section \ref{Sec:hybrids} for a description of this method).
Their results were compared
to evaluate the merits of the MS-MFS algorithm over the much simpler hybrid
method that used a combination of existing standard methods.
The data products evaluated were the total-intensity image, the continuum residual image
and the spectral index map.
The effect of the primary beam was ignored in these imaging runs because 
the angular size of Cygnus~A is about 2 arcmin, which at L-band is within
a few percent of the HPBW of the primary beam, 
a region where the antenna primary beam and its spectral effects can be ignored.

\begin{enumerate}
\item {\bf MS-MFS : }
The MS-MFS algorithm was run with a $2^{nd}$-order polynomial to model the source
spectrum and a set of 10 scale basis functions of different spatial scales
to model the spatial structure ($N_t=3, N_s=10$).
Iterations were terminated
using a 30 mJy stopping threshold. A theoretical continuum point-source 
sensitivity of 0.38 mJy was 
calculated for this dataset using an increased system temperature of $T_{sys}=250$
(due to the high total power of Cygnus~A).

\item {\bf Hybrid : }
The second approach was a hybrid algorithm
in which the MS-CLEAN algorithm was run separately on the data from each 
spectral window and then a single MS-CLEAN run was performed on the continuum
residuals (the STACK + MFS on residuals hybrid algorithm described in 
\ref{Sec:hybrids}). The total intensity image was constructed as an average of the
single channel image plus the result of the second stage on the continuum residuals.
This method is the same as that used in section \ref{Sec:hybrid} to test
the hybrid algorithm for the case of dense single-frequency $uv$-coverage. Note however
that the observations being described in this section do not have dense single-frequency
$uv$-coverage, and the purpose of applying this hybrid method is to emphasize the
errors that can occur if this method is used inappropriately.
\end{enumerate}

\paragraph{Results : }

Figure \ref{Fig:cyga_intres} shows the reconstructed total-intensity images (top row)
and  the residual images (bottom row) obtained from these two methods. 
Figure \ref{Fig:cyga_spx} shows the spectral maps constructed 
{\it via} the two methods described above as well as from existing images at 
1.4 and 4.8 GHz.

\begin{enumerate}
\item {\bf Intensity and Residuals :}  
Both methods gave a peak brightness of 77 Jy/beam at the hotspots 
and a peak brightness of about 400 mJy/beam for the
fainter extended parts of the halo. 
The residual images for both methods  
showed correlated residuals due to the use of a multi-scale
flux model composed of a discrete set of scales (small-scale correlated structure within 
the area covered by the source, but no visible large-scale deconvolution errors due to
missing large-scale flux). 

The off-source noise level 
achieved in the continuum image with MS-MFS was about 25 mJy,
giving a maximum dynamic range of
about 3000. The peak on-source residuals were at the level of 30 mJy.
Further iterations did not reduce these
residuals, and the use of a higher-order polynomial $N_t>3$ introduced more errors in
the spectral index map (see section \ref{Sec:ErrPowerLaw} for a discussion about errors
on the spectral index as a function
of $N_t$ and the SNR of the measurements). 
%
The off source RMS reached by the hybrid method was about 30 mJy, with the peak residuals
in the region of the source of 50 mJy. Deeper imaging in either stage did not
reduce these residuals.

Note also that both methods were almost two orders of magnitude above the theoretical point-source
sensitivity shown in Table~\ref{Tab:vladata_cyg} (calculated for
an equivalent wide-band observation).
However, the achieved RMS levels were consistent with the best RMS levels previously achieved
with the VLA at 1.4 GHz for this particular source at 
L-band ($\sim$20 mJy, [Perley, R. (private communication)]).

%Therefore, for these images, the limiting error looks to be due to the
%multi-scale representation, and not residual spectral effects...

\item {\bf Spectral Index : }
The image on the top left is the result of the MS-MFS algorithm and shows 
spectral structure at multiple scales across the source.  
For comparison, the image at the bottom is a spectral-index map constructed
from existing narrow-band images at 1.4 and 4.8 GHz, each constructed from 
a combination of VLA A, B, C and D configuration data \cite{CYGA_1991}.
These two images (top-left and bottom) show a very similar spatial distribution
of spectral structure. This shows that despite having a comparatively small amount
of data (20 VLA snapshots at 9 frequencies) the use of an algorithm that models
the sky brightness distribution appropriately is able to extract the same information
from the data as standard methods applied to large amounts of data.
The estimated errors on the spectral index map are $<0.1$ for the brighter regions
of the source (near the hotspots) and $\ge 0.2$ for the fainter parts of the lobes
and the core.

The image on the top right shows the spectral index map constructed from a
spectral cube (a set of 9 single-channel images) containing the results of
running the MS-CLEAN algorithm separately on each frequency and then smoothing
the results down to the angular resolution at the lowest frequency in the range.
Note that the single-frequency observations consisted of 20 snapshots of Cygnus~A.
This $uv$-coverage is too sparse to have measured all the spatial structure
present in the source, and the non-uniqueness of the single-frequency reconstructions
caused the images at the different frequencies to differ from each
other enough to adversely affect the spectra derived from these images.

\item {\bf Spectral Curvature : }
Note that although Cygnus~A itself has more than sufficient signal-to-noise to measure
any spectral curvature, very low level deconvolution errors (3 orders of magnitude below
the bright 77 Jy/beam hotspot) dominate the region around the 
very bright hotspots and this is sufficient to destroy the spectral curvature images.
That is, the signal-to-error ratio of the higher-order coefficient images
is too low to measure a physically plausible curvature
term (corresponding to a change in $\alpha$ of $<0.2$ across 700 MHz at 1.4 GHz).
\end{enumerate}

\paragraph{Wide-band Self Calibration : }

A few tests were done to test whether a self-calibration process that used 
wide-band flux models would yield any improvement on the gain solutions or
imaging results.

Two sets of calibration solutions were computed and compared.
For the first set of solutions, several rounds of amplitude and phase self-calibration were run,
beginning with a point-source model 
and using the MS-MFS algorithm to iteratively build up a wide-band flux model.
Self-calibration was terminated after new gain solutions were indistinguishable
from that of the previous run.
The second set of solutions was found by using a single 
1.4 GHz model for amplitude and phase self-calibration (with gain amplitudes
normalized to unity to preserve the source spectrum). 
No significant difference was found
and the second set of solutions were chosen for imaging. 

As an additional test, the final wide-band flux model generated {\it via} the
MS-MFS algorithm was used to predict model visibilities for a 
wide-band self-calibration step (amplitude and phase) to test
if this process yielded any different gain solutions.
Again, on these data, there was no noticeable improvement in the continuum residuals 
or on the stability of the spectral-index solution in low signal-to-noise regions.

This suggests that either the use of a common 1.4 GHz model image for 
all individual frequencies did not introduce much error, or that the residual
errors are dominated by the effects of multi-scale wide-band deconvolution
and the flux model assumed by the MS-MFS algorithm.  Further tests are required with
much simpler sky brightness distributions and real wide-band data, in order to clearly
ascertain when wide-band self-calibration will be required for high-dynamic range
imaging.


%In an unrelated test (done by SB), these data were used to test wide-band primary-beam
%correction {\it via} looking at Stokes V images with and without primary beam correction.
%A Stokes V signal of XXX was measured before correction - reduced to YYY afterwards.
%Cygnus~A hotspots are bright enough that a 1\% instrumental polarization so close
%to the center of the beam (1 arcmin out of a FWHM of 28 arcmin) is a 100mJy signal !!


\begin{figure}[t!]
\begin{center}
\epsfig{figure=pics/cyga_msmfs_stokesi.eps,scale=0.2}
\epsfig{figure=pics/cyga_hybrid_stokesi.eps,scale=0.2}
\epsfig{figure=pics/cyga_msmfs_residual.eps,scale=0.2}
\epsfig{figure=pics/cyga_hybrid_residual.eps,scale=0.2}
\end{center}
\caption[Cygnus~A : Intensity and residual images]
{\small Cygnus~A : Intensity and residual images : 
These images show the total intensity (top row) and residual images (bottom row)
obtained by applying two wide-band imaging methods to Cygnus~A data taken
as described in Table \ref{Tab:vladata_cyg}.
The images on the left are the result of the MS-MFS algorithm and those on 
the right are with the STACK + MFS hybrid in which MS-CLEAN was used for all the
deconvolutions (single-channel deconvolutions followed by 
second deconvolution on the continuum residuals. 
The total intensity images show no significant differences. 
Both residual images show correlated residuals of the type expected for 
the MS-CLEAN algorithm that uses a discrete set of scale sizes
(the error pattern obtained by choosing a nearby but not exact spatial scale
 for a flux component will be a ridge running along the edge of each flux component).
The peak and off source residuals for the MS-MFS algorithm are
30 mJy and 25 mJy and with the hybrid algorithm are 50 mJy and 30 mJy 
respectively, showing a very mild improvement in continuum sensitivity
with the MS-MFS algorithm.
}
\label{Fig:cyga_intres}
\end{figure}

\begin{figure}[t!]
\begin{center}
\epsfig{figure=pics/cyga_msmfs_spx.eps,scale=0.2}
\epsfig{figure=pics/cyga_hybrid_spx.eps,scale=0.2}
\end{center}
\begin{center}
\epsfig{figure=pics/cyga_ccarrili_spx.eps,scale=0.2}
\end{center}
\caption[Cygnus~A : Spectral Index image]
{\small Cygnus~A : Spectral Index image : 
These images show spectral index maps of Cygnus~A constructed {\it via}
the MS-MFS algorithm (top left) and the hybrid algorithm (top right)
applied to the data described in Table \ref{Tab:vladata_cyg}.
The image at the bottom is a spectral index map constructed from 
two narrow-band images at 1.4 and 4.8 GHz obtained
from VLA A,B,C and D configuration data at these two frequencies \cite{CYGA_1991}. 
The spatial structure seen in the MS-MFS spectral index image is very similar
to that seen in the bottom image. For comparison, the spectral index
map on the top-right clearly shows errors arising due to non-unique 
solutions at each separate frequency as well as smoothing to the
angular resolution at the lowest frequency.
}
\label{Fig:cyga_spx}
\end{figure}

\subsection{M87}

M87 is a bright (200 Jy) radio galaxy located at the center of the Virgo cluster.
The spatial distribution of broad-band synchrotron emission from this source 
consists of a bright central region (spanning a few arcmin) containing a 
flat-spectrum core, a jet (with known spectral index of $-$0.55) and two radio lobes
with steeper spectra ($-$0.5 $> \alpha > -$0.8) 
\citep{Rottmann_1996_AA,Owen_2000}.
This central region is 
surrounded by a large diffuse radio
halo (7 to 14 arcmin) with many bright narrow filaments ($\approx 10''\times 3'$).

Multi-frequency VLA data were taken in a similar way as described for
Cygnus-A, with a series of 10 snapshots at 16 different frequencies 
within the sensitivity range of the EVLA L-band receivers. 

Fig.\ref{Fig:m87_lobes} shows the intensity, spectral
index and spectral curvature maps of the bright central region at an
angular resolution of 3 arcsec (C+B-configuration).
Fig.\ref{Fig:m87_spectrum} shows a plot of the spectrum formed from the 
integrated flux in the central bright region.


The peak brightness at the center of the final restored intensity image
was 15 Jy with an off-source RMS of 1.8 mJy and an
on-source RMS of about between 3 and 10 mJy.
The residual images show low-level correlated residuals at the location of 
the source but deconvolution errors are almost absent from the rest of the 
image, indicating that the best off-source RMS noise level for these data has
almost been reached.
The maximum dynamic range (ratio of peak brightness to off-source RMS)
is about 8000, with the on-source dynamic range (ratio of peak brightness to
on-source RMS) of about 1000.

The spectral index map\footnote
{The spectral index between two frequency bands $A$ and $B$ will be denoted 
as $\alpha_{AB}$. For example, the symbol $\alpha_{PL}$ corresponds to the
frequency range between P-band (327 MHz) and L-band (1.4 GHz), and
$\alpha_{LL}$ corresponds to two frequencies within L-band (here, 1.1 and 1.8 GHz).
A similar convention will be used for spectral curvature $\beta$.
} of the bright central region (at 3 arcsec resolution)
shows a near flat-spectrum core with $\alpha_{LL} = -0.25$,
a jet with $\alpha_{LL} = -0.5$ and lobes with $-0.6 > \alpha_{LL} > -0.7$.

These numbers show that in the bright central region and in the halo 
there is sufficient signal-to-noise to measure the spectral index but any 
realistic spectral curvature (for broad-band synchrotron emission)
is detectable only within the central bright region.


This bright central region had sufficient
($>$100) signal-to-noise to be able to detect spectral curvature. 
The third panel in Fig.\ref{Fig:m87_lobes} shows the spectral curvature
measured within this region.
Note that the error bars on the spectral curvature are at the same level as 
the measurement itself. Therefore, a reliable estimate can only be obtained as
an average over this entire bright region. The average curvature is
measured to be $\beta_{LL} = -0.5$ which corresponds to a change in $\alpha$ 
across L-band by
$\triangle \alpha = \beta \frac{\triangle\nu}{\nu_0} \approx -0.2$.

These numbers were compared with two-point spectral indices computed between
327 MHz (P-band), 1.4 GHz (L-band), and 4.8 GHz (C-band) from existing images
\cite{Owen_2000},[Owen, F. (private communication)]. 
Across the bright central region, $-0.36 > \alpha_{PL} > -0.45$ and 
$-0.5 > \alpha_{LC} > -0.7$. The measured values ($-0.5 > \alpha_{LL} > -0.7$
and $\triangle\alpha \approx 0.2$) are consistent with these
independent calculations.
%($-0.5 > \alpha_{LL} > -0.7$ and $\triangle\alpha \approx 0.2$. 

The points in Fig.\ref{Fig:m87_spectrum} shows the integrated flux over the central
bright region of M87
(shown in $\log(I)$ vs $\log(\nu/\nu_0)$ space)
from the 16 single-spectral-window images. The curved line passing through these points
is the average spectrum that the MS-MFS algorithm automatically fit for this region.
It corresponds to $\alpha \approx -0.52$ and $\triangle \alpha \approx 0.2$ across
the source. The straight dashed lines correspond to constant spectral indices of $-$0.42 and
$-$0.62 and show that the change in $\alpha$ across the band is approximately 0.2 (as also
calculated from $\beta_{LL} = -0.5$ that the MS-MFS algorithm produced).
Note that the scatter seen on the points in the plot is at the 1\% level of the
values of the points (signal-to-noise of 100). Also evident from the plot is the
fact that the curvature signal is at a signal-to-noise ratio of 1.
These results show that a signal-to-noise of $>100$ is required to measure 
a change in spectral index of 0.2 across 700 MHz at 1.4 GHz.


\begin{figure}[t!]
\begin{center}
%\epsfig{figure=algomsmfs/m87_centre_stokesi.eps,scale=0.255}
%\epsfig{figure=algomsmfs/m87_centre_alpha.eps,scale=0.255}
%\epsfig{figure=algomsmfs/m87_centre_beta.eps,scale=0.255}
\epsfig{figure=pics/picM87_center_restored.eps,scale=0.15}
%\epsfig{figure=pics/picM87_center_residual.eps,scale=0.15}
\epsfig{figure=pics/picM87_center_alpha.eps,scale=0.15}
\epsfig{figure=pics/picM87_center_beta.eps,scale=0.15}
\end{center}
\caption[M87 core/jet/lobe : Intensity, Spectral index, Curvature]
{\small M87 core/jet/lobe : Intensity, Spectral index, Curvature : 
These images show 3-arcsec resolution maps of the central bright
region of M87 (core+jet and inner lobes), where the signal-to-noise was
sufficient for the MS-MFS algorithm to detect spectral curvature.
%The quantities displayed are the intensity at 1.5 GHz (left),
%the spectral index (middle) and the spectral curvature (right).
The quantities displayed are the intensity at 1.5 GHz (top left),
the residual image (top right),
%(peak on-source residual is 80 mJy/beam), 
the spectral index (bottom left) and the spectral curvature (bottom right).
The spectral index is near zero at the core, varies between $-$0.36 and
$-$0.6 along the jet and out into the lobes. The spectral curvature is
on average 0.5 which translates to $\triangle\alpha=0.2$ across L-band.
The peak of the source is 4.6 Jy, the on-source RMS is 40 mJy/beam 
and this gives an on-source signal-to-error ratio of about 100.
Note that the flux scale on the residual image (top right) is about 2 orders of magnitude
lower than the total-intensity image (top left).
}
\label{Fig:m87_lobes}
\end{figure}

\begin{figure}[t!]
\begin{center}
\epsfig{figure=pics/m87_specplot.eps,scale=0.3}
\end{center}
\caption[M87 core/jet/lobe : L-band spectrum]
{\small M87 core/jet/lobe : L-band spectrum : 
This plot shows the spectrum formed from the integrated flux within the central 
bright region between 1.1 and 1.8 GHz. The points are the integrated flux measured from single-spectral-window
model images, the curved line is the average spectrum that the MS-MFS algorithm
automatically fit to these data in this region. This spectrum corresponds to an average
$\alpha_{LL}=-0.52$ and a change of $\triangle\alpha\approx 0.2$ across the band (1.1 to 1.8 GHz).
The straight dashed lines represent pure power-law spectra with indices $-$0.42 and $-$0.62
and are another way of showing that the change in $\alpha$ across the band is about 0.2.
These numbers are consistent with two-point spectral indices computed between
327 MHz (P-band), 1.4 GHz (L-band), and 4.8 GHz (C-band) 
($-0.36 > \alpha_{PL} > -0.45$ and $-0.5 > \alpha_{LC} > -0.7$)
from existing images \cite{Owen_2000},[Owen, F. (private communication)]. 
}
\label{Fig:m87_spectrum}
\end{figure}


\section{MS-MFS error estimation and feasibility}\label{Sec:ErrorFeas}


\subsection{Dynamic-range vs $N_t$}

If continuum imaging is done with only MFS gridding and source spectra are ignored,
spectral structure will masquerade as spurious spatial structure. 
These errors will affect regions of the image both on-source and off-source
and their magnitudes depend on the available $uv$-coverage, the frequency range being covered,
the choice of reference frequency, and the intensity and spectral index of the source.
%
A rough rule of thumb for an EVLA-type $uv$-coverages (see section \ref{Sec:Err_order})
is that for a point source
of with spectral index $\alpha=-1.0$ measured between 1 and 2 GHz,
the peak error obtained if the spectrum is ignored is at a dynamic range of $<10^3$. 
Note that when all sources in the observed region of the sky have similar spectral indices, 
these errors can be reduced by dividing out an average spectral index
(one single number over the entire sky) from the visibilities before imaging them\footnote
{Note that such a division will reduce the signal-to-noise ratio of the higher-order
terms of the series (for the remaining spectral structure).
Therefore, although the removal of an average spectral index
could reduce the level of imaging artifacts obtained when source spectra are
ignored, the lower signal-to-noise ratio of the spectral signature 
could increase the error on the derived spectral index when MS-MFS is used. 
Note also that this point is not specific to the MS-MFS algorithm,
but is a general statement about how the accuracy of a fit depends on the SNR of the signal
being fitted.}.

%If this spectrum is fit by a straight line ($N_t=2$), then there still is an 
%approximation, and in addition to incurring an error on the value of $\alpha$, 
%errors will appear in the intensity image beyond a dynamic range of about $10^4$.

\subsubsection{3C286 field (EVLA)}

\begin{figure}[t!]
\begin{center}
\epsfig{figure=pics/pic1_3c286_nt1.eps,scale=0.2}
\epsfig{figure=pics/pic1_3c286_nt2.eps,scale=0.2}
\epsfig{figure=pics/pic1_3c286_nt3.eps,scale=0.2}
\epsfig{figure=pics/pic1_3c286_nt4.eps,scale=0.2}
\end{center}
\end{figure}

- frequency dependence of the primary-beam of the EVLA.....
- -1.4 at the HPBW (from SW paper and from EVLA sims). 

- spectral indices of surrounding sources : xx yy zz..... corrected to xxxxx. Confirmed
by a second test observation pointed at those background sources.

\subsubsection{Error as a function of bandwidth ratio}

ADD FIGURE

ALSO REFER TO CCW formula.
Magnitude of diff Taylor terms :
Are they visible in image as errors (vs) SNR needed to solve.



This section shows an example of the errors obtained when the order of the polynomial
chosen for imaging is not sufficient to model the power-law spectrum of the
source. 
EVLA datasets (8 hour synthesis) were simulated for 5 different frequency ranges
around 2.0 GHz. The sky brightness distribution used for the simulation
was one point source whose flux is 1.0 Jy and spectral index is -1.0 with
no spectral curvature.
The bandwidth ratios\footnote
{There are two definitions of bandwidth ratio that are used in radio 
interferometry. One is the ratio of the highest to the lowest frequency
in the band, and is denoted as $\nu_{high}:\nu_{low}$. Another definition
is the ratio of the total bandwidth to the central frequency 
$(\nu_{high}-\nu_{low})/\nu_{mid}$ espressed as a percentage.
For example, the bandwidth ratio for
$\nu_{low}=1.0$ GHz, $\nu_{high}=2.0$ GHz is $2:1$ and $66\%$.
}for these 5 datasets were
100\%(3:1), 66\%(2:1), 50\%(1.67:1), 25\%(1.28:1), 10\%(1.1:1).


Figure.\ref{Fig:taylor_error} shows the measured peak residuals and absolute
measured errors on $I_{\nu_0},\alpha,\beta$ when these datasets were
imaged using multi-frequency deconvolution with 
$N_t=1$ to $N_t=7$ and a linear spectral basis (Eqn.~\ref{Eq:tfunc_1}). 
%
All these datasets were imaged using a maximum of 10 iterations, a 
loop-gain of 1.0, natural weighting and a flux threshold of $1.0 \mu Jy$. No noise
was added to these simulations (in order to isolate and measure 
numerical errors due to the spectral fits).
Peak residuals were measured over the entire $0^{th}$ order
residual image, and errors on $I_{\nu_0},\alpha,\beta$ were computed
at the location of the point source by taking differences with
the ideal values of $I_{\nu_0}=1.0,\alpha=-1.0,\beta=0.0$.

\noindent Noticeable trends from these plots are listed below.
\begin{enumerate}
\item All errors appear to decrease exponentially (linearly in log-space)
as a function of increasing order of the polynomial, and as a function
of decreasing total bandwidth. For very narrow bandwidths, the use of
high-order polynomials increases the error.
\item The peak residuals are much smaller than the error incurred on the 
peak source flux at the reference frequency $I_{\nu_0}$ and the
errors on $\alpha$ and $\beta$. 
\item As an example, for a 2:1 bandwidth ratio, 
a source with spectral index = -1.0, and $N_t=4$,
the achievable dynamic range (measured as the ratio of the peak flux to the 
off-source peak residual) is about $10^5$, the error on the peak flux
at the reference frequency is 1 part in $10^3$, and the absolute
errors on $\alpha$ anre $\beta$ are $~10^{-2}$ and $~10^{-1}$ respectively.
\end{enumerate}
\noindent Note that these trends are based on one simple example, 
and further
analysis is required to understand the source of these errors and
assess how they vary as a function 
of $\alpha$ and $\beta$.
\citet{MFCLEAN_CCW} suggest that for an $(N_t-1)$-order
polynomial, the peak residuals proportional to the product of
$\alpha$ and the peak sidelobe level of the next higher order
$N_t^{th}$ spectral PSF. However, the results of the above tests 
do not follow this rule for all bandwidth ratios.  
Further work is required to (a) understand these errors in terms of
signal-to-noise and in the presence of deconvolution errors and
(b) be able to predict limiting dynamic ranges and error-bars on
$\alpha$ and $\beta$.

Note that all the code implementations for this dissertation use the 
linear expansion given by Eqn.~\ref{Eq:tfunc_1} 
(a polynomial in $I ~ vs ~ (\nu-\nu_0)/\nu_0$ space) to model an 
arbitrary spectrum.
However, in the case of a power-law, a logarithmic expansion
given by Eqn.~\ref{Eq:tfunc_2} 
(a polynomial in $I ~ vs ~ \log(\nu/\nu_0)$ space)
might need fewer terms than the linear expansion to model a power-law spectrum
and yield better results.
\citet{MFCLEAN_CCW} state that the logarithmic expansion has better convergence
properties than the linear expansion when $\alpha << 1$, but
this is yet to be tested for arbitrary values of $\alpha$.
Further, for given values of $\alpha$ and $\beta$, the radius of 
convergence of each series expansion defines a maximum bandwidth 
that it can be used with.
Further work is required to
do a formal comparison between these two sets of spectral basis functions
and their convergence properties when applied to arbitrary spectral shapes.


\begin{figure}
\begin{center}
\epsfig{figure=pics/plot_point_errors.eps,scale=0.3}
\end{center}
\caption[Peak Residuals and Errors for MFS with different values of $N_t$]
{\small Peak Residuals and Errors for MFS with different values of $N_t$ :
These plots show the measured peak residuals (top left) and the errors
on $I_{\nu_0}$(top right), $\alpha$ (bottom left), and $\beta$ (bottom right)
when a point-source
of flux 1.0 Jy and $\alpha$=-1.0 was imaged using Taylor polynomials of different
orders ($N_t=1-7$) and a linear spectral basis (Eqn.~\ref{Eq:tfunc_2}). 
This simulation was done with EVLA $uv$-coverages
(for an 8 hour synthesis run) and 100\%,66\%,50\%,25\% and 10\% fractional 
bandwidths, with a reference frequency of 2.0 GHz. No noise was added to 
these simulations. All runs used a loop-gain of 1.0, used natural weighting,
and were terminated after either 10 iterations or a flux threshold of $1\mu Jy$.
The x-axis of all these plots show the value of $N_t$ used for the simulation.
Plots for $\alpha$ and $\beta$ begin from $N_t=2$ and $N_t=3$ respectively
because at least that many terms are required to calculate these derived
quantities.
Noticeable trends from these plots are
(a) The peak residuals decrease by about a factor of 8 with each increase
of 1 more polynomial coefficient. (b) The peak residuals are larger for 
larger fractional bandwidths. (c) The errors on $I_{\nu_0},\alpha,\beta$
are larger than the peak residuals, but they too decrease with increasing
$N_t$. For very narrow bandwidths, the use of a very high-order polynomial
increases the error.
For a 2:1 bandwidth ratio, a spectral index of -1.0 and very 
high signal-to-noise, a $5^{th}$ or $6^{th}$ order Taylor expansion
is most appropriate (when a linear spectral basis is used).
}
\label{Fig:taylor_error}
\end{figure}


\subsection{Error on Spectral Index}

\subsubsection{Propagation of deconvolution errors}
%Error bars on $\alpha$ can be estimated by 
%error propagation for the expression $\alpha = I_1/I_0$ using measured or estimated
%values for the error on $I_0$ and $I_1$ (see section \ref{Sec:error_prop}).
%For an isolated point source, ${\delta\I^{\alpha}}/{\I^{\alpha}} = \sqrt{\left( {\delta\I_1}/{\I_1} \right)^2 + \left( {\delta\I_0}/{\I_0} \right)^2}$.
Deconvolution errors contribute to the on-source error in the Taylor 
coefficient images, and these errors propagate to the spectral index map which  
is computed as a ratio of two coefficient images.
%and it is not just the stddev of the pixel values as measured from the image. 
Table~\ref{Tab:evlasim_err} lists the estimated and observed errors in spectral index
and curvature for a simulated example and shows that the deconvolution errors that
result when a point-source flux model is used to deconvolve extended emission, 
can increase the error bars on the spectral index and curvature by an order of magnitude.
%

\subsubsection{Overlapping sources}

\subsubsection{Errors as a function of SNR}
The accuracy to which $\alpha$ and $\beta$ can be determined also
depends on the noise per spectral data point, the number of sampled frequencies,
the total frequency range of the samples, and the number of 
spectral parameters $N_t$ in the fit. Section \ref{Sec:ErrPowerLaw} discusses empirically 
derived error bars for the spectral index based on these factors.
These results show that to measure $\alpha=-1.0$ with less than 100\% error bars, we
need a single-channel signal-to-noise ratio greater than about 6, and to measure
$\beta$ with less than 100\% error, we need a single-channel signal-to-noise of 
greater than about 100.  
%
Note that these numbers dictate the required accuracy of wide-band flux calibration.
%\end{enumerate}


REFER TO CCW BIAS : Error due to using too few Taylor terms to fit the exp with a poly.
CCW comment on a bias that occurs with a 2-term T-exp. This is just the use of insufficient terms of a polynomial to model an exponential.


%%%%%%%%%%%%%%%%%%%%%%
The errors on the polynomial coefficients and quantities derived from them
will depend on the number of measurements of the spectrum, 
the signal-to-noise ratio
of the measurements, and their distribution across a frequency range.
%\item 
They will also depend on the order of the polynomial used in the approximation.
Although the physical parameters $I_{\nu_0},\alpha$ and $\beta$ can be obtained from 
the first three coefficients of a Taylor expansion of a power-law with varying index
(Eqns.\ref{Eq:taylorexpansion} and \ref{Eq:series_ab}),
a higher order polynomial may be required during the fitting process to 
improve the accuracy of the first three coefficients\footnote
{\citet{MFCLEAN_CCW} comment on a bias that occurs with a 2-term Taylor expansion, due to the 
use of a polynomial of insufficient order to model an exponential.}.
%
%Empirical estimates for some of these errors are shown below.
In the case of very noisy spectra, 
errors can also arise from attempting to use too many terms in the polynomial fit.

\enlargethispage{\baselineskip}
Figure \ref{Fig:SpecError} illustrates the above trends for the value of $\alpha$
derived from a polynomial fit to a spectrum
constructed {\it via} Eqn.~\ref{EQN_POWERLAW1} 
($I^{true}_0=10.0,\alpha^{true}=-1.5,\beta^{true}=-0.5,\nu_0 = 2.4$GHz)
and evaluated between 1-4 GHz. 
Gaussian random noise was added to give measurement signal-to-noise
ratios of 100, 10 and 1 for three such spectra. 
These spectra were fitted using a linear least-squares method
on two series expansions (Eqns.\ref{Eq:taylorexpansion},\ref{Eq:series_ab})
for different numbers of
terms in the series $N=2,3,4,5$, and also by a non-linear least-squares method to
fit $\alpha$ and $\beta$ directly.
The plots show the error on the derived spectral index
$\delta\alpha=\alpha^{fitted}-\alpha^{true}$ for each case.
%
%%The top,middle and bottom rows correspond to data  with signal-to-noise
%%ratios of 100, 10, 1 respectively. 
%The LEFT column shows the error between the true and fitted values
%of $\alpha$, computed as $|\alpha_{fitted} - \alpha_{true}|$
%and averaged over an ensemble of 50 repetitions with different instances of noise.
%The four blocks represent polynomial approximations with $N_{t}=2,3,4,5$ terms
%in the series. The three bars in each block represent different spectral models 
%(Red : Fit for coefficients of a polynomial in $I$ vs $\dnuno$ space, 
%Blue : Fit for coefficients of a polynomial in $I$ vs $log\nuno$ space,
%Green : Direct fits to the power-law parameters $I_0,\alpha,\beta$).
%%For each case, the error (Y-axis) is computed as 
%%$Err = \sqrt{(\delta I_0/I_0)^2 + (\delta\alpha/\alpha)^2 + (\delta\beta/\beta)^2}$
%%and averaged over an ensemble of 50 repetitions with different instances of noise.
%The RIGHT column shows the corresponding spectra (Noisy Data and Spectral Fits) 
%for one instance of the noise, and for $N_{taylor}=3$.

\noindent Noticeable trends (based on $\delta\alpha$) are listed below.
\begin{enumerate}
\item For high SNR, higher order fits give better results.
For low SNR, higher order fits give larger errors.
\item In most cases, a Taylor expansion of a power law about $\alpha=0,\beta=0$ is a better 
choice than a Taylor expansion about $\nu=\nu_0$.
\item For spectra between 1 and 4 GHz with $\alpha \approx -1.5 \pm 0.4$, a 3rd or
4th order Taylor expansion (either form) is most appropriate.
\end{enumerate}
These trends can be used to choose the spectral basis function and number of terms $N_t$ 
to be used in the multi-frequency deconvolution algorithm (Section \ref{Sec:MFS-deconv}),
based on {\it a priori} knowledge of the average spectral index
and the signal-to-noise ratio of the measurements. 
When there are both high and low signal-to-noise sources, a multi-stage approach 
using different values of $N_t$ might be required.
For example, deconvolution runs can begin with $N_t>3$ but once the peak residual
reaches $10\sigma$, a switch to $N_t=2$ might be beneficial 
(note that this situation has not yet been tested).



\begin{figure}
\begin{center}
\epsfig{figure=pics/plot_aberror_snr100_N100_alpha.eps,scale=0.2}
\epsfig{figure=pics/plot_aberror_snr10_N100_alpha.eps,scale=0.2}
\epsfig{figure=pics/plot_aberror_snr1_N100_alpha.eps,scale=0.2}
\caption[Error Estimates for Spectral Index]
{\small These plots show the average error on the fitted spectral index
	($\delta\alpha = \alpha^{fitted}-\alpha^{true}$) from 
100 noisy measurements of a power-law spectrum defined by
$I^{true}_0=10.0,\alpha^{true}=-1.5,\beta^{true}=-0.5$.
The rows represent different signal-to-noise ratios (Top : 100, Middle : 10, Bottom : 1). 
The left column shows the average $\delta\alpha$ with $N_{t}$ = 2,3,4,5 terms in 
the series, for three different functional forms (Red/Left : $T(\nu=\nu_0)$ : Taylor expansion
of $I_{\nu}$ about $\nu_0$, Blue/Middle : $T(\alpha=0,\beta=0)$ : Taylor expansion of $I_{\nu}$ about
$\alpha=0,\beta=0$, Green/Right : Power Law with varying index).  
The right column shows the corresponding spectra for $N_{t}=3$.
Noticeable trends are 
(a) For high SNR, higher order fits give better results.
(b) For low SNR, higher order fits give larger errors.
(c) In most cases, a Taylor expansion about $\alpha=0,\beta=0$ is a better choice
than an expansion about $\nu_0$.
(d) For spectra between 1 and 4 GHz with $\alpha \approx -1.5 \pm 0.4$, a $3^{rd}$ or $4^{th}$ order
Taylor expansion is most appropriate.}
\label{Fig:SpecError}
\end{center}
\end{figure}


\subsection{Moderately resolved sources}

Consider a source with broad-band continuum emission and  
spatial structure that is either unresolved at all sampled frequencies
or unresolved at the low-frequency end of the band and resolved
at the high-frequency end. The intensity distribution as well as
the spectral index of such emission can be imaged
at the angular resolution allowed by the highest frequency in the band.
%This is true for the intensity image at the reference frequency as well as
%for the spectral index and curvature maps. 
%For sources with smooth broad-band spectra, it is possible to reconstruct spectral
%structure at this highest angular resolution.
This is because compact emission has a signature all across the spatial frequency plane
and its spectrum is well sampled by the measurements. The highest frequencies constrain
the spatial structure and the flux model (in which a spectrum is associated with each
flux component) naturally fits a spectrum at the angular resolution at which the spatial 
structure is modeled. 
%This reconstruction will be accurate for sources with
%smooth broad-band emission associated with each location on the source (as opposed to
Note that such a reconstruction is model-dependent and may require extra information
in order to distinguish between sources whose observed spectra are due to genuine 
changes in the shape of the source with frequency and those with broad-band (power-law)
emission emanating from each location on the source.
%Note that such a reconstruction is model-dependent and may not be accurate for 
%sources whose observed spectra are due to genuine changes in the shape of the source
%with frequency (as opposed to broad-band emission emanating from
%each location on the source).



\begin{figure}[t!]
\begin{center}
%%\epsfig{figure=feasibilit/mod_chanimages.eps,scale=0.8}
\epsfig{figure=pics/pic2_modres_chan0.eps,scale=0.15}
\epsfig{figure=pics/pic2_modres_chan1.eps,scale=0.15}
\epsfig{figure=pics/pic2_modres_chan2.eps,scale=0.15}
\epsfig{figure=pics/pic2_modres_chan3.eps,scale=0.15}
\epsfig{figure=pics/pic2_modres_chan4.eps,scale=0.15}
\epsfig{figure=pics/pic2_modres_chan5.eps,scale=0.15}
\end{center}
\caption[Moderately Resolved Sources : Single-Channel Images]
{\small Moderately Resolved Sources -- Single-Channel Images : 
These figures show the 6 single-channel images generated from simulated
EVLA data between 1 and 4 GHz in the EVLA D-configuration. 
The angular resolution at 1 GHz is 60 arcsec, and at 4 GHz is 15 arcsec
and the white circles in the lower left corner shows the resolution
element decreaseing in size as frequency increases.
The sky brightness consists of two point sources, each of flux 1.0 Jy at
a reference frequency of 2.5 GHz and separated by 18 arcsec.
%making it a moderately resolved source. 
The pixel size used in these images is 4.0 arcsec.
From these single-channel images we can see that the sources begin to 
be resolved only at the higher end of this frequency range, and at the
lower end of the band is barely distinguishable from a single point source
centered on the bottom point source. 
The top point source has a 
spectral index of $+$1.0  %(0.4 Jy to 1.5 Jy across the band) 
and the bottom one has a spectral index of $-$1.0.
%(2.5 Jy to 0.5 Jy across the band) 
%as shown in Fig.\ref{Fig:modres_spectra}.
}
\label{Fig:modres_channel}
\end{figure}
\paragraph{EVLA Simulation : }
Wide-band EVLA data were simulated for the D-configuration
across a frequency range of 3.0 GHz 
with 6 frequency channels between 1 and 4 GHz (600 MHz apart).
This wide
frequency range was chosen to emphasize the difference in angular resolution
at the two ends of the band (60 arcsec at 1 GHz, and 15 arcsec at 4.0 GHz).
The sky brightness chosen for this test consists of a pair of
point sources separated by a distance of 18 arcsec (about one resolution
element at the highest frequency), making this a moderately resolved
source. These point sources were given different spectral indices
($+$1.0 for the top source and $-$1.0 for the bottom one).
Figure \ref{Fig:modres_channel} shows the 6 single-channel images of this
source.
%and Fig.\ref{Fig:modres_spectra} shows the spectra of the two point 
%sources. 
At the low frequency end, the source is almost indistinguishable 
from a single flux component centered at the location of the bottom source
whose flux peaks at the low-frequency end. The source structure becomes
apparant only in the higher frequencies where the top source (with a positive
spectral index) is brighter.
%

%\enlargethispage{\baselineskip}
\paragraph{MS-MFS Imaging Results : }
\begin{enumerate}
\item These data were imaged using the MS-MFS algorithm with $N_t=3$ and 
$N_s=1$ with only one spatial scale (a $\delta$-function).
Figure \ref{Fig:modres_msmfs} shows the results of this imaging run.
The intensity distribution, spectral index and curvature of this source were
recovered at the angular resolution allowed by the 3.6 GHz samples
(18 arcsec). 
These results show that for a source that can be modeled as a set
of flux components (in this case point-sources) with polynomial spectra,
even partial spectral measurements at the highest angular resolution
are sufficient to reconstruct the full spectral structure.
%(for example a synchrotron source where the spectrum at 
%each location can be described by a power law).
%Section \ref{Sec:overlap_modres} later shows an example of imaging
%a moderately resolved source that also contains extended emission.

%Iterations were terminated using a stopping threshold of 1.0 mJy
%which was much higher than the theoretical because the goal of this
%test was to 
\item A second imaging run was performed using only 
the first and last channels (1.0 GHz and 4.0 GHz).
The source is almost completely 
unresolved at 1 GHz (point sources separated by 18 arcsec 
within a 60 arcsec resolution element), and just resolved at 4 GHz
(with an 15 arcsec resolution element).
The goal of this exercise was to test the limits of this algorithm
and the ability of the flux model to constrain the solution when
the data provide insufficient constraints.
The MS-MFS algorithm was run with $N_t=2$ and $N_s=1$ and used the
same number of iterations as the previous example.
Fig.\ref{Fig:modres_msmfs_ends} contains the resulting 
intensity image and spectral index map and shows that it
is still possible to resolve the source and measure its spectral
index at the resolution of the highest frequency. However, the
deconvolution errors are considerably higher.
The obtained peak residual of 5 mJy is not much larger than the
3 mJy level obtained when all 6 channels were used while imaging, 
indicating that this reconstruction is not well constrained by
the data and the model plays a very significant role.
\end{enumerate}

%\noindent These results show that for wide-band imaging in which
%visibilities are measured at a series of approximately regularly-spaced
%frequencies within a large range, the flux model used in the
%MS-MFS algorithm is able to constrain the solution well enough to
%reconstruct the spatial distribution of the spectral structure at
%the angular resolution allowed by the high-frequency measurements.


%\paragraph{Results : }


\begin{figure}[t!]
%\epsfig{figure=pics/mod_msmfsresults.eps,scale=0.8}
\begin{center}
\epsfig{figure=pics/pic2_modres_restored.eps,scale=0.2}
\epsfig{figure=pics/pic2_modres_residual.eps,scale=0.2}
\end{center}
\begin{center}
\epsfig{figure=pics/pic2_modres_alpha.eps,scale=0.2}
\epsfig{figure=pics/pic2_modres_beta.eps,scale=0.2}
\end{center}
\caption[Moderately Resolved Sources : MSMFS Images]
{\small Moderately Resolved Sources -- MSMFS Images : 
These images show the results of running MS-MFS
on EVLA data that was simulated to test the 
algorithm on moderately resolved sources. 
The test sky brightness distribution consists of two point sources 
with spectral indices +1.0 (North) and $-$1.0 (South) separated by one
resolution element at the highest frequency.
The four images shown here are the intensity at 2.5 GHz (top left),
the residual image with a peak residual of 3 mJy (top right),    
the spectral index showing a gradient between $-$1 and +1 (bottom left)
and the spectral curvature which peaks between the
two sources and falls off on either side (bottom right). These results
demonstrate that an appropriate flux model will constrain the solution to a physically
realistic one even when the spectral measurements are incomplete
at the highest resolution. 
}
\label{Fig:modres_msmfs}
\end{figure}




\subsection{Emission at large spatial scales}


%At the lower end of the sampled spatial-frequency range, the size of the central $uv$-hole
%increases with observing frequency. For very large spatial scales whose visibility
%functions are adequately sampled (more than 80\% of the integrated flux) only at the lower
%end of the frequency range, an ambiguity between spatial scale and spectrum can arise
%during the reconstruction. This is because the spectrum of this source is not well-sampled
%by the measurements. 
%A flat-spectrum extended source can be mistaken for a 
%steep spectrum less extended source, and {\it vice-versa}.  This problem can be 
%avoided by providing short-spacing flux constraints (from single-dish observations) 
%to bias the solution, or by flagging all 
%spatial frequencies below $u_{min}$ at $\nu_{max}$ (the smallest spatial frequency
%sampled by the highest observed frequency) to filter out these large spatial scales.
%



%The reconstruction of the visibility function at spatial frequencies
%within the central hole in the $uv$-coverage at each frequency
%involves an extrapolation of the measurements down to the zero $uv$-spacing. 
Consider a very large (extended) flat-spectrum source whose  
visibility function falls mainly within the
central hole in the $uv$-coverage at the highest observing frequency.
With multi-frequency measurements, the size of the central hole in the
$uv$-coverage increases with observing frequency,
and for this source the minimum spatial frequency sampled per 
channel will measure a decreasing peak flux level as frequency increases.
%the fraction of the total flux measured decreases 
%
%When a very large extended source with a flat spectrum is measured with a wide-band 
%interferometer, the minimum spatial frequency per channel
%measures a decreasing peak flux level as frequency increases.  
Since the reconstruction below the minimum spatial frequency
involves an extrapolation of the measurements % down to the zero $uv$-spacing
and is un-constrained by
the data, these decreasing peak visibility levels can be mistakenly interpreted as
the result of a source whose amplitude itself is decreasing with frequency
(a less-extended source with a steep spectrum).
%It can therefore be mistaken for a less-extended source with a steep spectrum.
Usually, a physically realistic flux model is used to 
apply constraints in these unsampled regions of the $uv$-plane
%For example, multi-scale imaging effectively
%fits the Fourier template of a scale function to measurements at the edge of the
%$uv$-hole to reconstruct it.  
and MS-MFS models the sky brightness with polynomial spectra associated
with a set of extended 2D symmetric flux components. 
However, with this model
a large flat-spectrum source and a smaller steep-spectrum source are both
allowed and considered equally probable.
This creates an ambiguity between the reconstructed scale and spectrum that 
cannot always be resolved directly from the data, and 
requires additional information (perhaps 
a low-frequency narrow-band image to constrain the spatial structure, 
low-resolution spectral information, or total-flux constraints).

\paragraph{EVLA Simulation : }
Wide-band EVLA data were simulated for the D-configuration
across a frequency range of 3.0 GHz centred at 2.5 GHz.
(6 frequency channels located 600 MHz apart between 1.0 and 4.0 GHz).
The size of the central hole in the $uv$-coverage was increased
by flagging all baselines shorter than 100 m 
and the wide frequency range was chosen to emphasize the difference
between the largest spatial scale measured at each frequency.
%in the size of the central hole in the $uv$-coverage.
%corresponding largest spatial scale measured at each frequency
%($ 0.8~k\lambda$ or $5~arcmin$ at 1.3 GHz, 
% and $1.3~k\lambda$ or $2.6~arcmin$at 3.9 GHz).
(0.3~k$\lambda$ or 10.3 arcmin at 1.0 GHz, 
 and 1.3~k$\lambda$ or 2.5 arcmin at 4.0 GHz).

The sky brightness chosen for this test consists of one large
flat-spectrum 2D Gaussian whose FWHM is 2.0 arcmin (corresponding
to 1.6~k$\lambda$ at the reference frequency of 2.5 GHz), and one
steep spectrum point-source ($\alpha$=-1.0) located on top of
this extended source at 30 arcsec away from its peak.
%Although the single extended source would suffice for the purpose of
%this test, the point-source is included as a control source.

\paragraph{MS-MFS Imaging Results : }
These data were imaged using the MS-MFS algorithm with $N_t=3$ and
$N_s=3$ with scale sizes given by [0,10,30] pixels. 
Two imaging runs were performed with these parameters and both
were terminated after 100 iterations in order to be able to
compare their performance in terms of the peak residuals.

\noindent Fig.~\ref{Fig:zero_visplot} shows the visibility amplitudes 
present in the simulated data (left column) as well as in the reconstructed model 
(right column) at each of the 6 frequencies for these two imaging runs (top,bottom).
Fig.~\ref{Fig:zero_images} shows images of the intensity, spectral
index and residuals for these runs and compares them to the true sky brightness
reconstructed when all frequencies sample at least 95\% of the total
flux of the source.

\begin{enumerate}
\item The first imaging run applied the MS-MFS algorithm to the 
simulated data after flagging all baselines below 200m. 
No additional constraints were used on the reconstruction.
The visibility plots and imaging results show that 
%
%Figure~\ref{Fig:zero_visplot} shows plots of the visibility amplitudes 
%in the simulated data (left) as well as in the reconstructed model 
%(right) at each of the 6 frequencies. 
%Fig.~\ref{Fig:zero_images} shows images of the intensity, spectral
%index and residuals and show that 
%
from these data it is not possible to distinguish large flat-spectrum 
source from a slightly less-extended steep spectrum source.
This occurs because the visibility function is unconstrained by the data
within the central $uv$ hole and 
%the spectrum is un-constrained by the data and 
given the MS-MFS flux model,
both source structures are equally probable.
Note that the spectrum of the point-source was correctly estimated as $-$1.0.
%
This run was repeated a few times with slightly different input scale
sizes, and the results changed between a flat-spectrum source and
a source with a steep spectrum. If a scale size corresponding
to the exact size of the source was present in the set, the algorithm
was able to reconstruct the correct flux and spectrum. 

\item A second imaging run was performed on the same dataset, but
this time with additional information in the form of total-flux 
constraints at each observing frequency. These constraints were
added in by retaining a small number of very short-baseline 
measurements at each frequency in order to approximate the
presence of total-flux (or integrated flux) estimates 
(only baselines between 25 m and 100 m were flagged from the original
 EVLA D-configuration simulated data).
In practice, these constraints could be provided by single-dish 
measurements or estimates from existing low-resolution information
about the structure and spectrum of the source.
%
%Fig.~\ref{Fig:zero_dc_visplot} shows plots of the visibility amplitude
%used in the imaging run (left) and the corresponding reconstructed
%visibility function (right). Fig.~\ref{Fig:zero_dc_images} show
%the intensity, spectral index and residual images. 
%
The visibility plots and imaging results with this dataset 
show that the short-spacing flux estimates were sufficient to
bias the solution towards the correct solution in which the large
extended source has a flat spectrum and the point source has a
spectral index of $-$1.0.  Note that the residuals are at the same
level as in the previous run. This demonstrates that without 
the additional information about total-flux per frequency, both flux
models are equally poorly constrained by the data themselves.

\end{enumerate}

\noindent These results show that in the central unsampled region
of the $uv$-plane where there are no constraints from the data,
the MS-MFS flux model can produce ambiguous results and additional
information about the flux at low spatial-frequencies is required
(perhaps in the form of total-flux constraints per frequency).
%are required.  
%Note that this has been tested only for a simple
%spatial structure that can be fit with a single symmetric 2D Gaussian
%that the flux model supports. 
For complex spatial structure on these very large scales, the additional 
constraints may need to
come from existing low-resolution images of this field and the
associated spectra.
One way to avoid this problem altogether (but lose some information)
is to flag all spatial-frequencies smaller than $u_{min}$ at $\nu_{max}$
and not attempt to reconstruct any spatial scales larger than what
$\nu_{max}$ allows. 

\begin{figure}[h!]
\begin{center}
\epsfig{figure=pics/plot_uvplot_space_data.eps,scale=0.2}
\epsfig{figure=pics/plot_uvplot_space_data_model.eps,scale=0.2}
\epsfig{figure=pics/plot_uvplot_withdc_data.eps,scale=0.2}
\epsfig{figure=pics/plot_uvplot_withdc_data_model.eps,scale=0.2}
\end{center}
\caption[Very Large Spatial Scales : Visibility plots]% ($>$100m)]
{\small Very Large Spatial Scales - Visibility plots : 
These plots show the observed (left) and 
 reconstructed (right) visibility functions for a
simulation in which a large extended flat-spectrum source is observed
with an interferometer with a large central hole in its $uv$-coverage.
%
The different colours/shades in these plot represent
6 frequency channels spread between 1 and 4 GHz.
These data were imaged in two runs. 
%
The first imaging run (top row) used only
baselines $b>$100 m to emphasize the changing size of the central
hole in the $uv$-coverage across the broad frequency range. 
The plot on the top left shows 
%the visibility function  dominated
%by the large flat-spectrum source and 
how the different frequencies measure 
very different fractions of the integrated flux of the large flat-spectrum source.
The plot on the right shows
that these data can be mistakenly fit using a less-extended source with a steep
spectrum (instead of the large single source with a flat spectrum).
This is possible because within the central $uv$ hole
the spectrum is un-constrained by the data and given the MS-MFS flux model,
both source structures are equally probable.
%
The second imaging run (bottom row) used baselines $b<$25m in addition to $b>$100m to
approximate the addition of nearly total-flux measurements to the first
dataset to attempt to constrain the solution.
The plot on the bottom right shows that this additional information in the
form of short-spacing
constraints (or very low-spatial frequency measurements) is sufficient to
be able to reconstruct the correct sky brightness distribution.
Figure \ref{Fig:zero_images} shows the images that resulted from these tests.
}
\label{Fig:zero_visplot}
\end{figure}


\begin{figure}[h!]
\begin{center}
\epsfig{figure=pics/piczero_restored_mfs.eps,scale=0.15}
\epsfig{figure=pics/piczero_alpha_mfs.eps,scale=0.15}
\epsfig{figure=pics/piczero_residual_mfs.eps,scale=0.15}
\epsfig{figure=pics/piczero_restored_zero.eps,scale=0.15}
\epsfig{figure=pics/piczero_alpha_zero.eps,scale=0.15}
\epsfig{figure=pics/piczero_residual_zero.eps,scale=0.15}
\epsfig{figure=pics/piczero_restored_dc.eps,scale=0.15}
\epsfig{figure=pics/piczero_alpha_dc.eps,scale=0.15}
\epsfig{figure=pics/piczero_residual_dc.eps,scale=0.15}
\end{center}
\caption[Very Large Spatial Scales : Intensity, Spectral Index, Residuals ]
{\small Very Large Spatial Scales - Intensity, Spectral Index, Residuals : 
These images show the intensity distribution (left), spectral index
(middle) and the residuals (right) for three different imaging runs that 
	applied the MS-MFS 
algorithm to the simulated EVLA D-configuration data described in this section
(note that the flux scale used for the residual images in the 
 right column is 3 orders of
 magnitude smaller than the scale used for the intensity image in the left column).
The true sky flux consists of one large flat-spectrum symmetric flux 
component and one steep-spectrum ($\alpha=-1.0$) point source. \\
Top Row : When all baselines are used for imaging,
each frequency samples more than 95\% of the integrated flux. This is
sufficient to reconstruct the true brightness distribution and spectrum.\\
Middle Row : When the central $uv$-hole is increased in size by using
only baseline $b>100m$, the reconstructed model is a slightly smaller flux
component (compare the left column of images) 
with a steep spectrum (compare the middle column of images).\\
Bottom Row : When very short spacing (approximately total-flux) estimates are
included during imaging (using spacings $b<25m$ and $b>100m$), the true sky
brightness distribution is again recovered.
Note that the large-scale residuals in
all three runs are at the same level (2 mJy).
These results show that the spectra are unconstrained by the data 
for very large spatial scales whose visibility functions
fall within the central $uv$-hole at the highest frequency in the band,
and additional information is required. 
}
\label{Fig:zero_images}
\end{figure}


\subsection{Band-limited signals}
spectral lines, continuum subtraction....

\subsection{Frequency dependence of the Primary Beam}

When wide-band imaging is done across wide fields-of-view, sources away from the
pointing center will be attenuated by the value of the primary beam at each
frequency. Wide-band imaging results from such data ignoring the primary beam
will contain spurious spectral structure.
For the EVLA primary beams between 1 and 2 GHz, this extra spectral index at the
half-power point is about -1.4 and about -0.6 at the 70\% point
(see Figs.\ref{PLPB2} and \ref{PLPB3}).
Note that even if the source has a flat spectrum, this artificial spectral index
can cause errors at the levels described for ignoring source spectra in the
restored intensity image.

%The accuracy with which the primary beam and its spectral variations can be
%corrected depends on several factors, the most influential being the fact that
%all image-domain corrections involve an approximate time-averaged primary beam. 
%Tests on 
%simulated and real data show that up to the 70\% point of the primary beam (at the
%reference frequency), the spectral index can be corrected to within 0.05 for point sources
%with signal-to-noise ratios of greater than 100, and to within 0.1 for point sources
%with signal-to-noise ratios of about 10. 
%For extended emission, the errors are dominated by the
%effects of multi-scale deconvolution errors and not primary-beam correction.
%On high signal-to-noise simulations (SNR$>$100) with extended sources located at 
%the 60\% point of the primary beam at the reference frequency, the spectral index 
%was recovered to within an error of 0.2.
%

%__________________________________________________________________



%__________________________________________________________________

\section{Discussion}\label{Sec:CONCLUSIONS}
\begin{enumerate}
\item { Summary : Can do this'n'that (image and spectrum reconstruction + astrophysical parameters)}
...at the angular resolution defined at the highest measured frequency.
\item Single-Channel Imaging Hybrid vs MS-MFS
\item Astrophysics : A variety of astrophysical observations could gain from these new instruments due to the 
increased sensitivity and large instantaneous bandwidth. Synchrotron spectra can be measured
within the instantaneous bandwidth, to detect/measure turnovers/breaks.
Snapshot imaging of highly variable sources 
(sunspots, supernovae, AGN) can be improved with the increased uv-coverage of mfs while
deriving the instantaneous spectra at multiple spatial scales.
Weak supernova remnants could be detected against a continuum background by their spectral signature.
Spectral information can be obtained at the highest angular resolution. This is especially 
significant for moderately resolved sources. 
The atmospheres of stars have an angular size of xxx and are resolvable at xxx GHz in the middle
of the EVLA frequency coverage. MFS with data between xx and yy GHz could yield wideband spectra
at the angular resolution defined by yy GHz.

Sunspot flares - magnetic loops - frequency probes different depths - band-limited signals.

VLBI position offsets as a function of frequency - can be resolved (defined) via MFS.

Multiple arrays can be combined by matching spatial frequency coverage across wide frequency ranges.
For example, EVLA-D between 5 and 50 GHz would match with EVLA-C in L and S bands... (as has always
been done, but it can be imaged together - not sure what final advantage this will bring though..
Snapshot imaging with EVLA subarrays could also achieve this.

\item Software (CASA and ASKAPSOFT), Cost and Parallelization (Hybrid vs MSMFS)
\item { Limitations }
- moderately resolved sources will need simultaneous deconvolution of multipel components (ASP)
- unresolved at lower freq, resolved at higher - 1D simulations show that it is possible to reconstruct, if at least XXX fraction of the bw has resolved info.
\item { Future Directions }
Out into the null of the PB (alternate parameterisations).
Full-Stokes imaging (ref SW)... parameterize differently in lm and freq but same idea.
\end{enumerate}


%__________________________________________________________________

\begin{acknowledgements}
The authors would like to thank the NRAO and NMT for support during
the PhD thesis project that resulted in this algorithm and implementation
in a usable software package. Many thanks to XX, YY, ZZ,... for useful 
discussions.
\end{acknowledgements}



\bibliographystyle{stylefiles/aa} % style aa.bst
\bibliography{msmfs_aa} % your references Yourfile.bib

\begin{appendix}



\subsubsection{Difference with SW-MF-Clean}

%\noindent{\bf Differences :}
\paragraph{Differences}
In the SW-MFCLEAN algorithm, the Hessian block kernels and dirty
images are computed {\it via} FFT-based convolutions in which 
gridded Taylor-weights
are multiplied with gridded visibilities : $(w_1+w_2)(V_1+V_2)$.
\begin{eqnarray}
\label{Eq:sw_psf}
\vec{I}^{psf,sw}_{t,q} &=& \vec{I}^{psf}_{t} \star \vec{I}^{psf}_{q} ~~~\mathrm{where}~~~~\vec{I}^{psf}_x = [\Fd\Sd\W^{mfs}_x\Wim] \vec{1} ~~\mathrm{for}~~x=t,q\\
\label{Eq:sw_dirty}
\vec{I}^{dirty,sw}_t &=& \vec{I}^{psf}_t \star \vec{I}^{dirty} ~~~\mathrm{where}~~~
\vec{I}^{dirty}=[\Fd\Sd\Wim] \vec{V}^{corr} 
\end{eqnarray}
According to Eqns.\ref{Eq:mfs_neqn_1} to \ref{Eq:mf_neqn_3} (MFCLEAN algorithm), 
the Hessian block kernels and dirty images are to be computed by 
multiplying the visibility measurments with the Taylor-weights
before gridding the result : $(w_1 V_1) + (w_2 V_2)$.
\begin{eqnarray}
%%\vec{I}^{psf}_{tq} &=& [\Fd\Sd\W^{mfs}_{t+q}\Wim] \vec{\bf 1}\\ 
\label{Eq:mfs_psf}
\vec{I}^{psf}_{t,q} &=& [\Fd\Sd\Wnt\Wnq\Wim] \vec{1}\\ 
\label{Eq:mfs_dirty}
\vec{I}^{dirty}_t &=& [\Fd\Sd\Wnt\Wim] \vec{V}^{corr}
\end{eqnarray}

%\noindent{\bf Conditions for equality :}
\paragraph{Conditions for equality}
The two methods listed above (Eqns.\ref{Eq:sw_psf},\ref{Eq:sw_dirty} and
Eqns.\ref{Eq:mfs_psf},\ref{Eq:mfs_dirty})
are equivalent only under certain conditions.
Consider Eqn.~\ref{Eq:mfs_neqn_2} for $[H_{t,q}]$. Iff $[\Sd] = [\Sd\Sa\Sd]$ and $[\Sa] = [\Sa\Sd\Sa]$, 
%(i.e. $[\Sd]$ is the 1-inverse\footnote
% {Definition of 1-inverse...
% }of $[\Sa]$ and vice-versa), 
then $[\Sd]$ can be
replaced by $[\Sd\Sa\Sd]$. Further, $[\Sa\Sd]$ and $[\Wntd]$ are both diagonal
matrices of size $nN_c\times nN_c$ and therefore commute.
In this case, $[H_{t,q}]$ becomes
\begin{eqnarray}
[H_{t,q}] &=& [\Fd\Sd\Wntd\Wim\Wnp\Sa\F]\\
	  &=& [\Fd\Sd\Wntd\Sa\F][\Fd\Sd\Wim\Sa\F] [\Fd\Sd\Wnp\Sa\F]\nonumber\\
\Rightarrow~~~~ \vec{I}^{psf}_{tq} &=& I^{psf}_t \star I^{psf} \star I^{psf}_q\\ 
& & \mathrm{where}~~~ I^{psf} = [\Fd\Sd\Wim]\vec{1} ~~~\mathrm{and}~~~ 
I^{psf}_t = [\Fd\Sd\Wnt]\vec{1} \nonumber
\end{eqnarray}
This is still not the same as Eqn.~\ref{Eq:sw_psf} which has two instances of $[\Wim]$.
Therefore, only when $[\Wim]$ is an identity matrix (equally weighted visibilities)
will the kernel functions  from both methods be identical 
$\vec{I}^{psf}_{t,q} = \vec{I}^{psf,sw}_{t,q}$.
A similar argument holds for the dirty images.
%
The restriction of $[\Sd] = [\Sd\Sa\Sd]$ and $[\Sa] = [\Sa\Sd\Sa]$ implies that 
each row and column in $[\Sa]$ has only one 1, with the rest being 0. Since $[\Sa]$ has dimensions
$nN_c\times m$, the maximum number of non-zero elements must be $m$.  
Therefore, any of the $m$ discrete spatial frequencies cannot be measured at more than one
baseline or frequency channel.  
However, consider the $m\times m$ diagonal matrix of gridded imaging weights
$[W^G_{\nu}] = [\Snd\Wimn\Sna]$ per frequency channel.
A projection operator $[\Sna^G]$ of shape ($m\times m$) can be constructed for each frequency
channel, with each diagonal element corresponding to one spatial frequency grid cell.
Measurements from multiple baselines that map onto the same spatial frequency grid cell are
treated as a single measurement in $[\Sna^G]$, with an increased weight in $[W^G_{\nu}]$.  
The use of uniform weighting will flatten out $[W^G_{\nu}]$ as required for equality with
Eqn.~\ref{Eq:sw_psf}. 
Written this way, with multiple frequencies, $[\Sna^G]$ has dimensions $mN_c \times m$, 
and the restriction of $[\Sd] = [\Sd\Sa\Sd]$ and $[\Sa] = [\Sa\Sd\Sa]$ means that any spatial 
frequency must not be measured in more than one frequency channel.
Therefore, a pure matched-filtering approach is strictly valid only for uniform 
weighting and when all filled spatial frequency grid cells contain 
measurements from only one frequency channel.

\end{appendix}


\end{document}


